\section{A Logical Relation in $\presheaves{\omega}$ with Syntactic Minimal Invariance}\label{sec:guarded}

The final proposed approach to handling logical relations for state is
to instead study a language which allows us to merely encode
state. This is not sufficient to solve many of the original goals of
this work but it's a simpler problem and a natural stop along the way
to a proper logical relation. The natural candidate for a logical
relation is to consider (guarded) recursive kinds as described by
\citet{Pottier:11}. The motivation for doing this is that a language
with guarded recursive kinds is quite complex but in a very different
way than a language with state. It does not require Kripke worlds in
particular, the recursion is instead present in the kind structure. By
being present in the structure of types itself it is easier to handle
with type-based methods.

Recursive kinds require us to switch to a richer language where the
types themselves support a form of computation. For our purposes, the
language is given by the following grammar.
\[
  \begin{array}{lcl}
    e & ::= & x \mid \lam{x}{\tau}{e} \mid \ap{e}{e} \mid \Lam{\alpha}{\kappa}{e} \mid \Ap{e}{c}\\
    c, \tau & ::= & \alpha \mid \fn{\tau}{\tau} \mid \all{\alpha}{\kappa}{\tau} \\
      & \mid &  \lam{\alpha}{\kappa}{c} \mid \ap{c}{c} \mid \into{c}
               \mid \out{c}\\
    \kappa & ::= & j \mid \tp \mid \fn{\kappa}{\kappa} \mid \rec{j}{\kappa}
  \end{array}
\]
This language features a simple term language but a much richer type
language, here called \emph{constructors}. Constructors allow for
computation which produces types. This is formally captured by two
judgments: $\hasTJ{\Omega}{\Delta}{c}{\kappa}$ and
$\hasTJ{\Omega}{\Delta}{c_1 \equiv c_2}{\kappa}$. The first judgment
specifies that a constructor $c$ has a kind $\kappa$ in a context of
kind variables $\Omega$ and constructor variables $\Delta$. The second
specifies when two constructors are equal. This latter judgment is
related to the actual expressions of the language by the following
rule.
\[
  \inferrule{
    \hasEJ{\Omega}{\Delta}{\Gamma}{e}{\tau_1}\\
    \hasTJ{\Omega}{\Delta}{\tau_1 \equiv \tau_2}{\tp}
  }{\hasEJ{\Omega}{\Delta}{\Gamma}{e}{\tau_2}}
\]
The constructor language typing judgments are standard for F$\omega$
(see \citet{Barendregt:13} for an explanation) except for the
following rules.
\begin{mathpar}
  \inferrule{
    \hasKJ{\Omega, j}{\kappa}
  }{\hasKJ{\Omega}{\rec{j}{\kappa}}}\and
  \inferrule{
    \hasTJ{\Omega}{\Delta}{c}{\rec{j}{\kappa}}
  }{\hasTJ{\Omega}{\Delta}{\out{c}}{[\rec{j}{\kappa}/j]\kappa}}\and
  \inferrule{
    \hasTJ{\Omega}{\Delta}{c}{\rec{j}{\kappa}}
  }{\hasTJ{\Omega}{\Delta}{\out{c}}{[\rec{j}{\kappa}/j]\kappa}}
\end{mathpar}
The calculus described so far is a version of F$\omega$ with recursive
kinds. This language is quite expressive because recursive kinds allow
for recursive constructor definitions of arbitrary sort through
encoding the Y-combinator at the constructor level. It is
exceptionally poorly behaved, however, because recursive kinds allow
for constructors which do not have a weak head normal form. This
forces choices in the language that seem to have no good answers. For
instance, what terms belong to a constructor of kind $\tp$ that has no
weak head normal form.

There is a technique for potentially approaching a logical relation
for even this language with general recursive types. It draws on two
key ideas, one from Section~\ref{sec:domains} and one from
Section~\ref{sec:smi}. We define a pair of logical relations
$\definitely{-}$ and $\possibly{-}$ as in Section~\ref{sec:smi}. This
logical relation is defined on the \emph{normal forms} of
constructors however. In order to handle the cases where a term has an
infinite normal form or simply diverges, we actually work of finite
fragments of the B\"ohm tree of a term~\citep{Barendregt:13}. Two terms
are related at a type with an infinite normal form if and only if they
are in all finite approximations of that type. In this case, there is
actually a meaningful difference between $\definitely{-}$ and
$\possibly{-}$: if we reach a node in the B\"ohm tree which indicates
that more information exists but we are not allowed access to it in
this approximation, say ?, $\definitely{?} = \{(\bot, \bot)\}$ while
$\possibly{?} = \term \times \term$.

This approach for logical relations works in that the logical relation
is well-defined and if both sides of the birelation $\eta$ are
well-defined, then $\definitely{\tau}_\eta =
\possibly{\tau}_\eta$. The issue is that this fact does not scale to
general case of $c : \kappa$. In particular, it is not the case that
for an $\eta = \op{\eta}$ that
$\definitely{\lam{\alpha}{\kappa}{c}}_\eta = \possibly{\lam{\alpha}{x}{\kappa}}_\eta$.
The issue is that these must be equal as functions on pairs of
relations $(R_1, R_2)$ and not merely on a single relation $R$. If
$R_1 = R_2$ they are equal but this information is not sufficient to
get the case for $\forall$ to go through in the fundamental
theorem. This failure will be instructive for the coming issue:
$\forall$ is the only case where the higher-typed constructors factor
in to the expression language. This interaction is crucial for
\citet{Pottier:11} so it cannot be ignored, but it will be at the
heart of the failure of the next attempt.

Instead of considering full recursive kinds, we restrict our attention
to a particular class of them: the class of guarded recursive
kinds. Guarded recursion in general originates with
\citet{Nakano:00}. The idea is to isolate which recursive
definitions are \emph{productive}. Productivity is informally the idea
that after a finite number of steps something must be produced by a
program. The program need not compute to a normal form but it should
produce some observable output. The classic example of productivity is
a stream: an infinite stream should never evaluate to a normal form
but we would like to be able to calculate the $n$th element of the
stream in finite time. Productive programming is an extremely natural
paradigm for many programs which are never expected to terminate but
still expected to produce work as they go.

The question is how to ensure that a definition is productive. One
line of work is to consider a syntactic condition on programs. This is
used by some proof assistants~\citep{coq} and is simple and
sound. The idea is to check that no recursive calls occur without
being underneath a constructor. In the stream example, for instance,
this means that an element of the stream has to be produced before the
function is entitled to recurse. The issue is that a syntactic check
is too brittle. It does not, for instance, work if a productive
definition has been factored into two different parts since one of
these parts may not be syntactically unguarded. It is also ill-suited
for handling higher-order functions where the productivity may rely on
the supplied function being productive.

A more robust solution is to follow the approach of
\citet{Nakano:00} and equip the type theory with a modality
indicating that something is only available to use ``at a later point
in time''. Then, for instance, in a language with streams the
constructor for a stream would have the type
$\fn{\tau}{\later \mathrm{Stream}_\tau}{\mathrm{Stream}_\tau}$
indicating that it is sufficient to provide the tail of a list so that
it is only available later. In order to make this modality useful,
$\later$ comes equipped with a number of operations. Two of the basic
ones ensure that $\later$ is an applicative functor:
\begin{align*}
  \delay{-} &: \fn{\tau}{\later \tau}\\
  - \zap - &: \later (\fn{\tau_1}{\tau_2}) \to \fn{\later \tau_1}{\later \tau_2}
\end{align*}
Finally, $\later$ can be used to construct fixed points.
\begin{align*}
  \fix{} : \fn{(\fn{\later \tau}{\tau})}{\tau}
\end{align*}
All of this is extremely similar to the operations available in
$\presheaves{\omega}$. The first two operations were facts about the
functor $\later$ and the last one is L\"ob induction. Nakano's work on
guarded recursion provides a syntactic account of this modality. In
the original presentation the operations written above were presented
as subtyping relations in the calculus rendering them all silent. In
more recent
presentations~\citep{Birkedal:steps:11,Bizjak:16,Birkedal:16,Bahr:17}
these operations are all made explicit again as they are mediated by
nontrivial isomorphisms. For our purposes, we will continue the
subtyping presentation for consistency with \citet{Pottier:11} which
describes the encoding motivating this construction.

Guarded recursive kinds are slightly different than guarded recursion
in general because the term language contains divergent and
nonproductive terms. The guarantees provided by guardedness apply only
at the level of constructors where it is the case that all
constructors are productive (have a weak head normal form). In this
sense, the language of guarded recursive kinds can be seen as a
particular simply-typed calculus with guarded recursion and a
distinguished base type $\tp$. The term-formers of this type are the
type-formers of our expression language: $\fn{-}{-}$ and
$\all{\alpha}{\kappa}{-}$. A further distinction between this calculus
and other type theories is that our type formers are all
contractive. This means that they take arguments available only later
and produce a result available now. In type theories with a universe,
for instance, the operations on the universe are all nonexpansive:
they preserve laters by taking arguments available later and producing
a result later (the inverse of the $- \zap -$ operation would do this
for instance). This extra strength provided by contractiveness is
exactly what allows for the construction of divergent terms in our
language. This would present a soundness issue for a type theory but
from our perspective it is a natural convenience of a programming
language unconcerned with serving as a logic.

To make this discussion formal, the (selected) rules of our language are the
following.
\begin{mathpar}
  \declareJudgement{\hasKJ{\Omega}{\kappa}}\\
  \inferrule{
    j \in \Omega
  }{\hasKJ{\Omega}{j}} \and
  \inferrule{
    \hasKJ{\Omega}{\kappa} \\
    \hasKJ{\Omega}{\kappa'}
  }{\hasKJ{\Omega}{\fn{\kappa}{\kappa'}}}\and
  \inferrule{
    \hasKJ{\Omega}{\kappa}
  }{\hasKJ{\Omega}{\later \kappa}} \and
  \inferrule{ }
  {\hasKJ{\Omega}{\tp}} \and
  \inferrule{
    \hasKJ{\Omega, j : \kind}{\kappa}\\
    \guardJ{j}{\kappa}
  }{\hasKJ{\Omega}{\rec{j}{\kappa}}}
\end{mathpar}
\begin{mathpar}
  \declareJudgement{\guardJ{j}{\kappa}}\\
  \inferrule{
    \mbox{$j$ not free in $\kappa$}
  }{\guardJ{j}{\kappa}}\and
  \inferrule{
    \guardJ{j}{\kappa_1}\\
    \guardJ{j}{\kappa_2}
  }{\guardJ{j}{\fn{\kappa_1}{\kappa_2}}} \and
  \inferrule{ }{\guardJ{j}{\later \kappa}} \and
  \inferrule{
    \guardJ{j}{\kappa}
  }{\guardJ{j}{(\rec{j'}{\kappa})}}
\end{mathpar}
\begin{mathpar}
  \declareJudgement{\subKJ{}{\kappa}{\kappa'}}\\
  \inferrule{
    \subKJ{}{\kappa_1'}{\kappa_1} \\
    \subKJ{}{\kappa_2}{\kappa_2'}
  }{\subKJ{}{\fn{\kappa_1}{\kappa_2}}{\fn{\kappa_1'}{\kappa_2'}}} \and
  \inferrule{
    \subKJ{}{\kappa}{\kappa'}
  }{\subKJ{}{\later \kappa}{\later \kappa'}} \and
  \inferrule{
  }{\subKJ{}{\kappa}{\later \kappa}} \and
  \inferrule{
  }{\subKJ{}{\later (\fn{\kappa}{\kappa'})}{\fn{\later \kappa}{\later \kappa'}}} \and
  \inferrule{
  }{\subKJ{}{\fn{\later \kappa}{\later \kappa'}}{\later (\fn{\kappa}{\kappa'})}}
\end{mathpar}
\begin{mathpar}
  \declareJudgement{\hasTJ{}{\Delta}{c}{\kappa} \qquad \hasEJ{}{\Delta}{\Gamma}{e}{\tau}}\\
  \inferrule{
    \hasTJ{}{\Delta}{\tau_1}{\later \tp}\\
    \hasTJ{}{\Delta}{\tau_2}{\later \tp}
  }{\hasTJ{}{\Delta}{\fn{\tau_1}{\tau_2}}{\tp}}\and
  \inferrule{
    \hasTJ{}{\Delta, \alpha : \kappa}{\tau}{\later \tp}
  }{\hasTJ{}{\Delta}{\all{\alpha}{\kappa}{\tau}}{\tp}}
  \and
  \inferrule{
    \hasTJ{}{\Delta}{\tau}{\later^n\,\tp} \\
    \hasEJ{}{\Delta}{\Gamma, x : \tau}{e}{\tau'}
  }{\hasEJ{}{\Delta}{\Gamma}{\lam{x}{\tau}{e}}{\fn{\tau}{\tau'}}}
  \and
  \inferrule{
    \hasEJ{}{\Delta}{\Gamma}{e}{\forall \alpha : \kappa . \tau}\\
    \hasTJ{}{\Delta}{c}{\later^n \kappa}
  }{\hasEJ{}{\Delta}{\Gamma}{\Ap{e}{c}}{[c / \alpha] \tau}}
\end{mathpar}
Turning now to the question of how to construct a logical relation for
this language there are two questions to be answered:
\begin{enumerate}
\item We want to interpret kinds as objects of a category, which
  category should be used?
\item Pairs of related terms should then be points of this object,
  what does it mean to be a point of $\den{\tp}$.
\end{enumerate}
This two-layer approach is typical of a logical relation for languages
with a rich kind structure. We need to interpret kinds as some sort of
semantic object so that $\den{\tp}$ is the collection of semantic
types needed for the logical relation. Traditionally the literature
around logical relations for higher-kinded languages has been
concerned with notions of parametricity at higher
kind~\citep{Hasegawa:94,Robinson:94,Dunphy:04,Vytiniotis:10,Atkey:12},
which extends the object-relational interpretation of
Reynolds~\citep{Reynolds:83} to include natural
notions of candidates at higher kind. These candidates include more
complex conditions that also necessitate the use semantic tools to
handle but it is not the goal of our work. In particular,
parametricity arguments will only be interesting for type variables of
kind $\tp$. At all other kinds the candidates are not relations
between semantic types but rather just semantic types. For instance
the interpretation of a constructor function of kind $\fn{\tp}{\tp}$
is a map between relations, not a relation between maps of semantic
types.

This is an undesirable simplification for a general purpose logical
relation but the only motivation for constructing this logical
relation is to study the semantics of guarded recursive kinds and
their impact on the definition. The technical complications introduced
by studying reflexive graphs would be wasted when we plan to move to a
setting with only $\tp$ and Kripke worlds. It is not clear what it
would mean to consider a model where Kripke worlds were subject to a
relational interpretation as well. To my knowledge this has not been
studied explicitly but it is perhaps latent in work like
\citet{Dreyer:10} where the Kripke world is more relational to allow
for relations between heaps that are not merely point-wise. This,
however, is a much loftier goal than merely constructing any logical
relation for state which is what we are after.

Now the answer to these questions is given by the earlier remark that
our language is the guarded simply-typed lambda calculus with guarded
recursive types and a strange base type. Models of this language are
readily available in $\presheaves{\omega}$~\citep{Birkedal:guarded:10,Birkedal:steps:11} and we
make use of these.

Specifically, we interpret kinds as follows as maps a functor from
environments to presheaves. These environments are mixed-variance.
They send a kind variable to two distinct presheaves, one of which is
used covariantly and one contravariantly. Let us signify these
environments by $\theta$ (to distinguish them with from the
environment used by the interpretation of constructors) and write
$\op{\theta}$ for the operation exchanging all the components of
$\theta$. Finally, by convention let $\theta(\alpha)$ project out the
presheaf being used covariantly.
\begin{align*}
  \den{j}_\theta &= \theta(j)\\
  \den{\later \kappa}_\theta &= \later \den{\kappa}_\theta\\
  \den{\fn{\kappa_1}{\kappa_2}}_\theta &= \den{\kappa_2}_\theta^{\den{\kappa_1}_{\op{\theta}}}\\
  \den{\rec{j}{\kappa}}_\theta &= \fix((P_1, P_2) \mapsto \den{\kappa}_{\theta[j \mapsto (P_1, P_2)]})\\
  \den{\tp}_\theta &= \semtypes\\
\end{align*}
It is a straightforward proof to show that this defines a functor. The
only exceptional fact is that it is necessary to show that taking a
fixed point some arguments of a mixed-variance functor results in a
functor in the remaining arguments. This is proven in Section 7 of
\citet{Birkedal:steps:11} for instance.

The interesting part of this definition is the interpretation of
$\tp$, $\den{\semtypes}$. The other clauses are simply induced by the
logical meaning of the connectives. Interpreting $\to$ as the
exponential, for instance, induces a canonical interpretation of
$\lam{\alpha}{\kappa}{c}$ and $ap{c_1}{c_2}$.

The space of semantic types is where we make crucial use of syntactic
minimal invariance. First we need to define a class of relations
appropriate for semantic types.
\begin{defn}\label{def:guarded:urel}
  A uniform relation is a subset
  $R \subseteq \term_{\tau_1} \times \term_{\tau_2}$ for some $\tau_1$
  and $\tau_2$ so that the following hold.
  \begin{itemize}
  \item $R$ is closed on both sides under contextual equivalence.
  \item $R$ is admissible and in particular contains limits of all
    $\pi$-chains.
  \item If $(e_1, e_2) \in R$ then for all $n$,
    $(\ap{\pi_{\tau_1}^n}{e_1}, \ap{\pi_{\tau_2}^n}{e_2}) \in R$.
  \end{itemize}
  The set of uniform relations is denoted $\urel$.
\end{defn}
Uniform relations are a natural setting for our logical relation since
the assumptions imposed on them are quite natural for anything to be
called a semantic type. Items 1 and 3 require that we respect
observation in and item 2 is a technically necessary and quite
reasonable assumption. The definition of projection in this language
is identical to what is was for the System F-like language considered
previously with the slight proviso that instead of
$\pi_{\alpha}^{n + 1} = \lam{x}{\alpha}{x}$ we must generalize it to
$\pi_{K[\alpha]}^{n + 1} = \lam{x}{\alpha}{x}$ so that any
\emph{neutral type} is impossible to truncate. This definition is
forced, there are no other reasonable moves to make at higher-type
since without knowing anything about $\alpha$ no information can be
deduced about $K[\alpha]$

The presheaf structure on uniform relations is instance of a more
general and well-known embedding of complete 1-bounded ultrametric
spaces into $\presheaves{\omega}$ described for instance in \citet{Birkedal:steps:11}:
\[
  \semtypes(n) = \urel / {\equiv_n}
\]
The equivalence relation at $n$ is defined to equate uniform relations
which agree on $n$ truncations. Formally, this is expressed as
follows.
\begin{align*}
  R \equiv_n &S =\\
  &(\forall (e_1, e_2) \in R.\ (\ap{\pi^i_{\tau_1}}{e_1}, \ap{\pi^i_{\tau_2}}{e_2}) \in S) \land {}\\
  &(\forall (e_1, e_2) \in S.\ (\ap{\pi^i_{\tau_1}}{e_1}, \ap{\pi^i_{\tau_2}}{e_2}) \in R)
\end{align*}
Restriction maps simply send equivalence classes of uniform relations
to even coarser equivalence classes of uniform relations.

This definition implies that a global point: $h : 1 \to \semtypes$
precisely in correspondence to a particular uniform relation. It picks
out a family of equivalences classes so that each is increasingly fine
$h(0)(\star) \supseteq h(1)(\star) \supseteq ... \supseteq h(n)(\star)$
Syntactic minimal invariance and admissiblity of uniform relation
tells us that if two uniform relations are equated by $\equiv_n$ for
all $n$ then they are equal on the nose. This means that there can be
at most one uniform relation in all the equivalence classes selected
by $h$ and $h$ is uniquely induced by the uniform relation. This
result is in fact a corollary of the fact that the aforementioned
embedding preserves finite limits.

Now the logical relation (the constructor interpretation) is really a
standard interpretation of the simply-typed lambda calculus with a
base type who's occupants are relations. It is defined as follows.
\begin{align*}
  \den{\alpha}_\eta(n) &= \eta(\alpha)(n)\\
  \den{\lam{\alpha}{\kappa}{c}}_\eta(n) &= (\star, S) \mapsto \den{c}_{\eta[\alpha \mapsto S]}(n)\\
  \den{\ap{c_1}{c_2}}_\eta(n) &= \den{c_1}_\eta(n)(\star)(\den{c_2}_\eta(n))\\
  \den{\out{c}}_\eta &= \mathsf{out}_n(\den{c}_\eta(n))\\
  \den{\into{c}}_\eta &= \mathsf{in}_n(\den{c}_\eta(n))\\
  \den{\fn{\tau_1}{\tau_2}}_\eta(n) &=\\
  \{(e_1, e_2) \mid&\ e_1 \Downarrow \iff {e_2 \Downarrow} \AND \\
  & \forall (a_1, a_2) \in \den{\tau_1}_\eta(n). \ (\ap{e_1}{a_1}, \ap{e_2}{a_2}) \in \den{\tau_2}_\eta\}\\
  \den{\all{\alpha}{\kappa}{\tau}}_\eta &=\\
  \{(e_1, e_2) \mid&\ e_1 \Downarrow \iff {e_2 \Downarrow} \AND\\
  &\forall c_1, c_2 : \kappa,\ S \in \den{\kappa}.
    \ (\Ap{e_1}{c_1}, \Ap{e_2}{c_2}) \in \den{\tau}_{\eta[\alpha \mapsto S]}\}
\end{align*}
In this definition we are using $\mathsf{out}$ and $\mathsf{in}$ as
the isomorphisms mediating a recursive type and its unfolding. There
is a technical issue with this presentation: it properly should be
done on \emph{derivations} of terms and a coherence theorem must be
proven. This is because the while it is the case that if
$\kappa_1 \le kappa_2$ then there is a canonical
$i : \den{\kappa_1} \mono \den{kappa_2}$, it is not the case that $i$
is an identity. This means that when the subtyping rule is applied in
a derivation the resulting semantic object must be adjusted
explicitly. This issue is obliviated when working in ultrametric
spaces because the subtyping coercions are simply identities
there. This technical detail is not relevant to the issue with this
approach and thus we have suppressed it.

The extra applications of $n$ and $\star$ are results of the fact that
the interpretation is a natural transformation from $\den{\Delta}$,
here written $\eta \in \den{\Delta}$ to $\den{\kappa}$. This is
defined by an indexed family of functions which is what has been done
here. Additionally, since constructor functions are exponentials in
the category at stage $n$ they are natural transformations
$\yoneda(n) \times \den{\kappa_1}_\eta \to \den{\kappa_2}_\eta$ so we
must apply them to an element of $\yoneda(n)(n) = \{\star\}$ in
addition to the actual argument.

The issue arises from the fact that this must be made well-typed and
natural. These two requirements are inexorably tied together by the
interpretation of $\lambda$ which is only a valid natural
transformation if $\den{c}$ is natural in $\eta$. It is unfortunately
the case that these properties fail to hold for the definition of
$\den{\all{\alpha}{\kappa}{\tau}}$. In this case, being well-typed
implies that for at $n$, for every pair $\eta$, $\eta'$ which are
equal at stage $n$ it must be that
$\den{\all{\alpha}{\kappa}{\tau}}_\eta \equiv_n \den{\all{\alpha}{\kappa}{\tau}}_{\eta'}$.
Importantly, equality at stage $n$ is \emph{not} necessarily just
point-wise equality because two semantic types are equal at stage $n$
if they are related by $\equiv_n$. This essentially is a statement
that $\den{\all{\alpha}{\kappa}{\tau}}$ is functional with respect to
$\equiv_n$ and this is simply false.

As a counter example, consider the following instance of $\tau$.
\[
  \tau = \fn{\ap{\alpha}{\beta}}{\ap{\alpha}{\gamma}}
\]
For convenience, let us denote
$\all{\alpha}{\fn{\tp}{\tp}}{\fn{\ap{\alpha}{\beta}}\ap{\alpha}{\gamma}}$
as $\tau'$ In this case we must show that show that
$\den{\tau'}_\eta \equiv_n \den{\tau'}_{\eta'}$ provided that
$\eta(\beta) \equiv_n \eta'(\beta)$ and
$\eta(\gamma) \equiv_n \eta'(\gamma)$. Let us suppose that
$\eta(\beta)$ is a uniform relation $R \subseteq \tau_a \times \tau_a$
and $\eta'(\beta)$ is $S \subseteq \tau_a \times \tau_a$ and both
$\eta(\gamma)$ and $\eta'(\gamma)$ are $R$. All that is needed to
cause a problem is to assume that $R \neq S$. If such an $R$ and $S$
cannot be constructed then $\urel$ is trivial and our logical relation
is degenerate.

We are required to show that $\pi^n$ transports from
$\den{\tau'}_\eta(n)$ to $\den{\tau'}_{\eta'}(n)$ and back again. Proving
the first part of this means showing that if
$(e_1, e_2) \in \den{\tau'}_\eta(n)$ then the following holds.
\[
  (\ap{\pi_{[\tau_a/\beta]\tau'}}{e_1}, \ap{\pi_{[\tau_a/\beta]\tau'}}{e_1})
  \in \den{\tau'}_{\eta'}(n)
\]
However, if $n > 2$ and $e_1$ and $e_2$ do not diverge, then this is
equivalent to showing the following.
\[
  (\Lam{\alpha}{\tp}{\Ap{e_1}{\alpha}}, \Lam{\alpha}{\tp}{\Ap{e_1}{\alpha}})
  \in \den{\tau'}_{\eta'}(n)
\]
This can be simplified further since $\den{\tau'}_{\eta'}(n)$ is a uniform
relation to just the following.
\[
  (e_1, e_2) \in \den{\tau'}_{\eta'}(n)
\]
Which is to say, we are required to show that if $\eta$ and $\eta'$
are only related by $\equiv_n$ then we must show that
$\den{\tau'}_{\eta}$ and $\den{\tau'}_{\eta'}$. However, it is not
hard to show that the following two facts are true.
\[
  (\Lam{\alpha}{\fn{\tp}{\tp}}{\lam{x}{\ap{\alpha}{\tau_a}}{x}},
   \Lam{\alpha}{\fn{\tp}{\tp}}{\lam{x}{\ap{\alpha}{\tau_a}}{x}})
   \in \den{\tau'}_\eta
\]
\[
  (\Lam{\alpha}{\fn{\tp}{\tp}}{\lam{x}{\ap{\alpha}{\tau_a}}{x}},
   \Lam{\alpha}{\fn{\tp}{\tp}}{\lam{x}{\ap{\alpha}{\tau_a}}{x}})
   \not\in \den{\tau'}_{\eta'}
\]
This means that $\den{-}$ is not type-correct and not a well-formed
definition. At the root of the issue is the fact that syntactic
minimal invariance does not seem to scale to higher kinds without some
fundamentally new idea. Without something akin to a type-casing
operation it does not seem possible to define a correct version of
$\pi$ at neutral types. It really seems that it must proceed by
induction on the variable the neutral constructor is stuck on. This
issue does not arise in domain theory because in this setting
projections are untyped and case on the structure of the element of
the domain not the ``type'' that this element may later be shown to
inhabit.

Any cure for this issue seems worse than the illness. Adding type-case
for instance allows us to define a correct projection for instance but
it destroys the parametricity of the system. One could consider a
system like LX~\citep{Crary:99} and define projection with respect to
the reification of a type. This solves the issue for all the
types that we have reifications of but there is a diagonalization
issue here: codes are needed for every single constructor but the
codes themselves are constructors.

Finally, as mentioned previously $\urel$ is inadequate for
representation independence results. Since we have required that
uniform relations are closed under truncations any uniform relation
between different types will relate divergent terms on one side to
nondivergent terms on the other. This puts a dent in using this
technique to construct a logical relation for state. Essentially, we
have removed the explicitly need for $\trunc_i$ but all proofs must
act as though $\trunc_i$ is still in the language. All together, this
means that even fixing this particular logical relation is likely
inadequate for our purposes.

%%% Local Variables:
%%% mode: latex
%%% TeX-master: "../main"
%%% End:
