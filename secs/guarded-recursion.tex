\section{A Logical Relation in Ultrametric Spaces}\label{sec:guarded}

The final proposed approach to handling logical relations for state is
to instead study a language which allows us to merely encode
state. This is not sufficient to solve many of the original goals of
this work but it's a simpler problem and a natural stop along the way
to a proper logical relation. The natural candidate for a logical
relation is to consider (guarded) recursive kinds.

Recursive kinds require us to switch to a richer language where the
types themselves support a form of computation. For our purposes, the
language is given by the following grammar.
\[
  \begin{array}{lcl}
    e & ::= & x \mid \lam{x}{\tau}{e} \mid \ap{e}{e} \mid \Lam{\alpha}{\kappa}{e} \mid \Ap{e}{c}\\
    c, \tau & ::= & \alpha \mid \fn{\tau}{\tau} \mid \all{\alpha}{\kappa}{\tau} \\
      & \mid &  \lam{\alpha}{\kappa}{c} \mid \ap{c}{c} \mid \into{c}
               \mid \out{c}\\
    \kappa & ::= & j \mid \tp \mid \later \kappa \mid \fn{\kappa}{\kappa} \mid \rec{j}{\kappa}
  \end{array}
\]
This language features a simple term language but a much richer type
language, here called \emph{constructors}. Constructors allow for
computation which produces types. This is formally captured by two
judgments: $\hasTJ{\Omega}{\Delta}{c}{\kappa}$ and
$\hasTJ{\Omega}{\Delta}{c_1 \equiv c_2}{\kappa}$. The first judgment
specifies that a constructor $c$ has a kind $\kappa$ in a context of
kind variables $\Omega$ and constructor variables $\Delta$. The second
specifies when two constructors are equal. This latter judgment is
related to the actual expressions of the language by the following
rule.
\[
  \inferrule{
    \hasEJ{\Omega}{\Delta}{\Gamma}{e}{\tau_1}\\
    \hasTJ{\Omega}{\Delta}{\tau_1 \equiv \tau_2}{\tp}
  }{\hasEJ{\Omega}{\Delta}{\Gamma}{e}{\tau_2}}
\]
The constructor language typing judgments are standard for
F$\omega$\citep{TODO-FOMEGA} except for the following rules.
\begin{mathpar}
  \inferrule{
    \hasTJ{\Omega}{\Delta}{c}{\rec{j}{\kappa}}
  }{\hasTJ{\Omega}{\Delta}{\out{c}}{[\rec{j}{\kappa}/j]\kappa}}\and
  \inferrule{
    \hasTJ{\Omega}{\Delta}{c}{\rec{j}{\kappa}}
  }{\hasTJ{\Omega}{\Delta}{\out{c}}{[\rec{j}{\kappa}/j]\kappa}}
\end{mathpar}

%%% Local Variables:
%%% mode: latex
%%% TeX-master: "../main"
%%% End:
