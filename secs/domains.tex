\section{Tying the Knot Using Domains}\label{sec:domains}

This section describes the first and most obvious approach to building
a step-index-free logical relation for state. It would seem that the
real crux of the issue was this Kripke world. Nothing past the
construction of the Kripke world in Section~\ref{sec:steps}
was particularly challenging, especially when a logical approach is
taken. So a natural approach is to find a different setting in which
to solve the following equation
\[
  \worlds \cong \assignables \pto (\worlds \mto \pow{\term \times \term})
\]
Traditionally, domains have been used for exactly this purpose. The
impetus of domain theory as an investigation separate from order
theory in general was the observation by Dana Scott that domains
possessed solutions to equations that sets did not. For instance, a
model of the lambda calculus may be constructed by solving the
equation
\[
  D \cong D \to D
\]
This equation is not possible to solve in sets without trivializing
$D$. In fact, Theorem~\ref{thm:steps:fixed-points} uses a construction
derived originally from domain theory itself.

\subsection{Domain Theory, Briefly}

A domain is a particular sort of partial order which comes equipped
with enough structure to accurately describe a notion of continuous
functions upon it.
\begin{defn}\label{def:domains:directed-set}
  In a partial order $P$, a directed set $D \supseteq P$ is a set so
  that $D \neq \emptyset$ and if $a, b \in D$ then there is a $c \in
  D$ so that $a, b \le c$.
\end{defn}
\begin{defn}\label{def:domains:domain}
  A domain\footnote{In this text domains are directed complete partial
    orders as opposed to any of the myriad varierties of domains one
    might have chosen.} is partial order possessing joins (least upper
  bounds) of all directed subsets. Given a directed subset $D$ we
  write this join as $\join D$.
\end{defn}
Intuitively, elements of a domain represent a (partial) piece of
knowledge. Moving to a larger element represents moving to a piece of
knowledge that extends what was already known. The ability to
construct joins of directed sets fits into this intuition as the
ability to take a great many fragments of compatible knowledge and
glue them together.

This intuition primarily comes from the example domain of a partial
function. The function is not everywhere defined so it is not a
complete piece of knowledge, but given a collection of partial
functions which agree on the intersection, they can be glued
together. More precisely, fix our underlying set of $S \pto D$ to be
partial functions from some set $S$ to a domain $D$. The ordering on
these functions is given by the following:
\[
  f_1 \reach f_2 \defs \forall s \in S.\ f_1(s) \Downarrow \implies f_1(s) = f_2(s)
\]
Then, joins are constructed on directed sets as follows:
\[
  (\join_i f_i)(s) = \join \{d \mid \exists i.\ f_i(s) = d\}
\]
Showing that this is a partial function relies crucially on the
directedness of $(f_i)_i$.

The extra structure on domains gives rise to a new class of maps that
preserve this structure. These maps are called (Scott) continuous in
that they preserve joins.
\begin{defn}[Scott continuity]
  A map between domains is (Scott) continuous if it preserves the
  joins of directed sets. That is, if $f : D \to E$ and
  $I \subseteq D$ is directed, then
  $f(\join I) = \join \{f(i) \mid i \in I\}$.
\end{defn}
Calling this definition continuity can be justified by considering the
Scott topology on domains, in which continuous functions are precisely
those which preserve upper bounds. For a complete account of the
theory of domains, see for instance \citet{Girez:03,Abramsky:94}.
\begin{thm}
  Domains form an a category, $\DCPO$, where morphisms are continuous
  functions. This category is bicartesian closed, in particular the
  set of continuous maps between two domains is a domain.
\end{thm}
The interesting portion of domain theory for these purposes is the
ability to solve domain equations. Specifically, locally continuous
functors of domains again have solutions.
\begin{defn}
  A functor, $F : \DCPO \to \DCPO$ is locally continuous if its action
  on morphisms is a continuous function. That is, there is a
  continuous map of domains $B^A \to F(B)^{F(A)}$. This definition
  generalizes readily to a mixed variance n-ary functor.
\end{defn}
\begin{thm}\label{thm:domains:fixed-points}
  Any mixed variance locally continuous functor
  $F : \op{\DCPO} \times \DCPO \to \DCPO$ has an invariant object $I$
  so that $F(I, I) \cong I$.
\end{thm}

\subsection{Domains for State}

The domains involved in expression the desired equation are quite
straightforward. Power sets are always a complete lattice and
therefore a domain. Scott continuous functions are always monotone so
instead of considering monotone functions it's more natural to pick
Scott continuous functions to get a domain. Finally, partial maps form
a domain as shown above with the small complication that they cannot
be limited to finite partial maps.

This gives us a proper functor $F : \op{\DCPO} \to \DCPO$ defined by
\begin{align*}
  D \mapsto& \assignables \pto (D \mto \pow{\term \times \term})\\
  f \mapsto& \lambda x.\ \lambda \alpha.\ \lambda d.\ x(\alpha)(f(d))
\end{align*}
There is only one exceptional feature of all of this, the ordering on
the partial map aspect of this domain is nonstandard. A Kripke world
is supposed to map an assignable to a semantic type. The ordering on
these worlds should let us add new mappings, but old mappings should
stay the same. They must in fact, if a location is allocated so that
it can be either $\littrue$ or $\litfalse$, it would be a real
problem if suddenly it could also be a natural number. This ordering,
explicitly, is then defined by the following formula.
\[
  w_1 \reach w_2 \iff \forall \alpha \in \dom(w_1).\ w_1(\alpha) \simeq w_2(\alpha)
\]
This ordering, however natural, means that $F$ is not locally
continuous and therefore there is no way to construct a solution to
it.

Supposing for a moment that a solution to $F$ did exist. There would
still seemingly be a problem because one of these worlds could have
full support, making modeling allocation impossible. This can be
avoided by defining the logical relation on compact
elements~\citep{Girez:03} of the world domain, precisely those worlds
enjoying only finite support.

By defining the logical relation on these worlds, it is trivial to
extend it to an arbitrary world by taking $\den{\tau}(w)$ on such a
world to be the join of the results of $\den{\tau}$ on all the compact
elements below $w$. This is automatically continuous and recovers
exactly $\den{\tau}$ on compact elements which are, after all, the
worlds of interest.

%%% Local Variables:
%%% mode: latex
%%% TeX-master: "../main"
%%% End:
