\section{Understanding State through (Syntactic) Minimal Invariance}\label{sec:smi}

Now that domains are seen to be unhelpful, there is some question
about where to turn next. There are not many well-studied
algebraically complete categories~\citep{Freyd:70} which support
interesting denotational semantics. The semantic structures for
step-indexing that support solving recursive domain equation require
more than continuity or approximation, they require a notion of
closeness for two semantic types. In $\presheaves{\omega}$ for
instance, there was a natural notion of when two points
$p, q : 1 \to P$ were close by checking at what stage they became
equal, eg when $p_n = q_n : \{\star\} \to P$.

Crucially, in order for this to make sense for semantic types every
equality was judged relative to a step. Two semantic types are equal
at $n$ if they agree on equalities that hold at stage $n$ or earlier,
eg, equalities which hold if the two sides are only allowed to run for
$n$ steps. Without step-indexing what could this be replaced by?

The answer is suggested by an idea in domain theory to resolve a
seemingly unrelated problem. The observant reader may have noticed
that there is a crucial strengthening that
Theorem~\ref{thm:steps:fixed-points} provides over
Theorem~\ref{thm:domains:fixed-points}, the former provides unique
solutions. In fact, there may be many nonisomorphic domains which
satisfy a given equation. This state of affairs can prove troublesome
for semantics in domains because it means that not all the properties
of the domain are determined by the equation. There may be ``exotic''
elements which are not required to satisfy the domain equation but are
present none the less which impedes reasoning about recursive domain
equations by induction or similar. \citet{Pitts:96} proposes a
solution to this by imposing an additional requirement on the solution
to a domain equation: minimal invariance.

Minimal invariance, precisely, is the statement that given a $D$ so
that $D \cong F(D, D)$ for a particular family of functions $\pi_i : D
\to D$ the following equation holds.
\[
  \join_i \pi_i = 1
\]
The function $\pi_i$ is determined by induction on $i$.
\begin{align*}
  \pi_i &: D \to D\\
  \pi_0 &\defs \bot\\
  \pi_{i + 1} &\defs F(\pi_i, \pi_i)
\end{align*}
The intuition behind this definition is that if $\pi_i$ converges to
$1$ then every element in $D$ must have arisen as the limit of
elements computed by the inclusion $F^n(\bot, \bot) \to D$ which can
be viewed as semantic requirement that elements of $D$ are
finite\footnote{Or rather, determined a the limits of ``finite''
  elements. This is not to be confused here with \emph{compact} elements and
  algebraic domains which share a similar intuition but require
  technically different properties since there is no reason to assume
  that $F$ preserves algebraicness}. It is not difficult to prove that
if $D \cong F(D)$ and $E \cong F(E)$ with $D$ and $E$ both being
minimally invariant then $E \cong D$ and the standard $D_\infty$
construction produces a minimally invariant domain. Therefore, domain
equations have a unique minimally invariant solution and these satisfy
a form of induction through minimal invariance.

An interesting observation made originally by
\citet{Birkedal:99} and extend by
\citet{Crary:07} is that for the class of domains arising
naturally in semantics (so called ``universal domains'') the
projection functions, $\pi_i$, are computable
syntactically. Additionally, since contextual equivalence can be
generalized to contextual approximation in a natural way, minimal
invariance can be stated as a syntactic property of the programming
language itself. In this setting, projections are naturally
type-indexed and this is denoted $\pi_\tau^i : \fn{\tau}{\tau}$. The
definition of $\pi_\tau^i$ behaves almost as an $\eta$-expansion where
after $i$ $\eta$ expansion the function simply diverges.

It is well worth it to take a moment to be precise about the
definitions involved.
\begin{defn}\label{def:smi:contextual-approximation}
  Two terms $\hasESigJ{\Delta}{\Gamma}{\Sigma}{e_1, e_2}{\tau}$ are
  said to be contextually approximate at type $\tau$, written
  $\approxESigJ{\Delta}{\Gamma}{\Sigma}{e_1}{e_2}{\tau}$ if for all
  contexts
  $\hasCEJ{\Delta}{\Gamma}{\Sigma}{\tau}{C}{\cdot}{\cdot}{\Sigma'}{\cmd{\tau'}}$,
  $\steps{C[e_1]}{\cmd{m_1}}$ and $\steps{C[e_2]}{\cmd{m_2}}$ such
  that for any heap $h : \Sigma$ then
  $(m_1, h) \Downarrow \implies (m_2, h) \Downarrow$.
\end{defn}
We will adopt the traditional abuse of abandoning the context
annotations and just writing $e_1 \cless e_2$ when the contexts and
$\tau$ are obvious.
\begin{defn}\label{def:smi:contextual-limits}
  A family of terms $(e_i)_{i \in I}$ is said to have a limit $e$ if
  for all $i \in I$ $e_i \cless e$ and $e$ is the smallest element
  with this property.
\end{defn}
Since limits are only unique up to contextual equivalence, it is not
necessarily well defined to write $\join_i e_i$ as if it were a term
rather than an equivalence class of terms. We will instead often make
use of the notation $\join_i e_i = e$ to signify that $e$ belongs to
this equivalence class. The corresponding syntactic version of minimal
invariance can now be precisely stated.
\begin{thm}\label{thm:smi:smi}
  For all $\hasTJ{}{\Gamma}{\tau}{\tp}$ and $e : \tau$,
  $\join_i \ap{\pi_\tau^i}{e} = e$
\end{thm}
The idea behind using syntactic minimal invariance is that instead of
defining a single logical relation $\den{-}_{-}$ a pair of logical
relations is defined instead: $\definitely{-}_{-}$ and
$\possibly{-}_{-}$. The definition of these two logical relations are
mutually recursive and provide a decomposition of the logical relation
two monotone halves. The idea is that one should think of
$\definitely{\tau}_{\eta}$ containing equalities that are certainly
true but it does not contain all of them. On the other hand
$\possibly{\tau}_\eta$ contains all equalities that are true as well
as a few that might not be. This suggests a natural relationship
between these two sets where
$\definitely{\tau}_\eta \subseteq \possibly{\tau}_\eta$. The crucial
move that syntactic minimal invariance provides is to show the
following.
\[
  \forall i.\ (e_1, e_2) \in \definitely{\tau}_\eta \implies
  (\ap{\pi_\tau^i}{e_1}, \ap{\pi_\tau^i}{e_2}) \in \possibly{\tau}_\eta
\]
Then, by showing the comparatively easy lemma that
$\possibly{\tau}_\eta$ is always closed under limits when $\eta$ is it
immediately follows that
$\definitely{\tau}_\eta = \possibly{\tau}_\eta$. This maneuver is
useful because these two definitions are naturally monotone therefore
its easy to express a natural definition of recursive types, as is
done in \citet{Crary:07} for instance. The only complication
is that environments must generalize from relations to birelations, a
map to a pair of relations rather than a map to a single
relation. Define $\op{\eta}$ to be the environment mapping $\alpha$ to
the relations $\eta(\alpha)$ with the components swapped. The crucial
insight is that when the variance in the definition of the logical
relation swaps, we switch from $\definitely{-}$ to $\possibly{-}$ and
vice-versa, for instance:
\begin{align*}
  \definitely{\fn{\tau_1}{\tau_2}}_\eta &=
  \{(e_1, e_2) \mid \forall (a_1, a_2) \in \possibly{\tau_1}_\eta.
  \ (\ap{e_1}{a_1}, \ap{e_2}{a_2}) \in \definitely{\tau_2}_\eta\}
\end{align*}
This approach seems to naturally fit recursive types and one might
wonder then if it can be scaled to support state just as step-indexing
does\footnote{If this sentence leave the reader in suspense I
  encourage them to glance at the title}.

The key insight is that while the distinction of
$\definitely{-}$/$\possibly{-}$ is not useful, projection functions
give us a notion akin to step-indexing. We can say that two semantic
types are equal at stage $i$ if they are equal upon truncating all the
programs in the equalities by $\pi^i$. It remains then to define
$\pi_\tau^i$ for our language. In order to do this we need a
divergence at the level of expressions, this can be easily
accomplished by adding fixed points or just adding a formal term
$\bot$ so that $\step{\bot}{\bot}$.
\begin{align*}
  \pi_\tau^0 &= \lam{x}{\tau}{\bot}\\
  \pi_{\fn{\tau_1}{\tau_2}}^{i + 1} &=
  \lam{f}{\fn{\tau_1}{\tau_2}}{\lam{x}{\tau_1}{\ap{\pi_{\tau_2}^i}{(\ap{f}{(\ap{\pi_{\tau_1}^i}{x})})}}}\\
  \pi_{\allNoKind{\alpha}{\tau}}^{i + 1} &=
  \lam{f}{\allNoKind{\alpha}{\tau}}{\LamNoKind{\alpha}{\ap{\pi_\tau^i}{(\Ap{f}{\alpha})}}}\\
  \pi_{\cmd{\tau}}^{i + 1} &=
  \lam{x}{\cmd{\tau}}{\cmd{\bnd{x'}{x}{\ret{\ap{\pi_\tau^i}{x'}}}}}
\end{align*}
This approach, while effective for recursive types, is not quite what
is needed to handle state. After all, the issue in
Section~\ref{sec:steps} was not the definition of
$\den{\cmd{\tau}}_\eta(w)$, it was the construction of the set in which
$w$ lives. On the other hand, the projections themselves are
interesting, they provide a way to construct an indexing of semantic
types without actually counting steps. This approach was explored in
\citet{Birkedal:domain:10} and successfully used to construct a
logical relation for a similar language. This work, however, made use
of the normal domain theoretic projections rather than their syntactic
counterparts. The question of producing a logical relation for our
language is then reduced to the question of producing a syntactic
account of \citet{Birkedal:domain:10}.

In this account, monadic computations are represented by
store-passing.
\begin{align*}
  \states &= \assignables \pto D\\
  D &\cong (D \to D) \oplus (D \times D) \oplus ... \oplus \states \to (\states \times D)
\end{align*}
Projections, $\pi^i$, work by casing on which branch of the domain
equation the argument is in and proceed accordingly. This is analogous
to how with syntactic minimal invariance the projection function is
split according the type of the argument. Most of the clauses are
standard with the interesting one being the last one for
commands. Denote a value in of the domain in the last branch as
$\cmd{f}$ where $f : \states \to (\states \times D)$. Then
$\pi^{i + 1}$ on this satisfies the following formula.
\begin{align*}
  \pi^{i + 1}(\cmd{f}) &=
  \cmd{\lambda s. (\trunc^i(\pi_1 f(s)), \pi^i(\pi_2 f(s)))}
  \trunc^i(s) = \pi^i \circ s
\end{align*}
In words: truncation on a command runs the command a point-wise
truncated heap and then truncates the return value as well as the
resulting heap.

In order to implement this syntactically, there's an issue. How can it
be possible to replicate in a setting where accessing the whole heap
is forbidden. In fact, this cannot be implemented as is. This presents
a serious issue in scaling up syntactic minimal invariance and there
is no clear solution.

If, for instance, a new command $\trunc_i \div \mathrm{unit}$ is
added to the language it is possible to express the projections. With
the other definition left unchanged the new case,
$\pi_{\cmd{\tau}}^{i + 1}$, is as follows.
\begin{align*}
  \lam{c}{\cmd{\tau}}
  {\cmd{& \bnd{\_}{\cmd{\trunc_i}}{\\
        & \bnd{x}{c}{\\
        & \bnd{\_}{\cmd{\trunc_i}}{\\
        & \ret{\ap{\pi_\tau^i}{x}}}}}}}
\end{align*}
If this is added then Theorem~\ref{thm:smi:smi} holds when we add the
following rule for evaluating $\trunc^i$.
\begin{mathpar}
  \inferrule{
  }{\stepsM{\trunc^0}{h}{\trunc^0}{h}}\and
  \inferrule{
    h(\alpha) = \ap{\pi_\tau^i}{\alpha}
  }{\stepsM{\trunc^{i + 1}}{h}{\ret{()}}{h'}}
\end{mathpar}
It is worth noting the strangeness of the last rule which is morally
``truncating the full heap''. This strangeness is twofold. It bakes in
the $\pi_\tau$ functions into the very operational semantics despite
them just being chunks of user-defined code. It also requires that the
heap be in some way typed so that when projecting at some particular
cell it is possible to determine the type to truncate at.

Beyond the strangeness of the rule, however, it works. A logical
relation can be defined using Kripke worlds which index semantic types
by truncation. This is problematic, however, because it changes the
notion of contextual equivalence. By adding new constructs to the
language new contexts are added and this impacts contextual
equivalence. In this extended language, programs that ought to be
equal are no longer. For instance, the following program should be
equivalent to $\cmd{\cmd{\ret{\lam{x}{\tau}{x}}}}$ but is not.
\begin{align*}
  \cmd{& \dcl{\alpha}{\lam{x}{\tau}{x}}{\\
       & \ret{\cmd{\get{\alpha}}}}}
\end{align*}
The issue is that another part of the program could truncate the heap,
replacing $\lam{x}{\tau}{x}$ with a truncation of itself. This
truncation would diverge strictly more than the original and thus the
second piece of code only contextually approximates the first. This
weakening is discussed more in subsequent work to
\citet{Birkedal:domain:10}, for instance in
\citet{Birkedal:adts:12}. In our case, however, this restriction is
damaging. It makes it impossible to prove any information-hiding
results using our logical relation and this makes it difficult to
prove that certain effects are benevolent\footnote{So actually this is
  a key motivation that I forgot about until now. TODO mention it
  earlier.}.

This puts a stop to naively extending syntactic minimal invariance to
our language with state. An idea with briefly considering is to build
a logical relation on a language with $\trunc_i$ and then attempt to
excise the terms that should not have been in there in the first place
after the fact. This approach has proven to be too complicated to
feasibly manage. Instead, a better approach seems to be to isolate a
more type-theoretic characterization of the difficulty of a logical
relation for state.

%%% Local Variables:
%%% mode: latex
%%% TeX-master: "../main"
%%% End:
