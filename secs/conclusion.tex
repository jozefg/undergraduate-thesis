\section{Conclusions}

This thesis has illustrated the difficulty of constructing a logical
relation for a language with higher-order state. The current set of
techniques which work are complex and despite this they only provide
sound but not complete reasoning principles. Even with this complexity
logical relations are among the best known techniques for handling
effects. They strike a combination between using simple mathematics
and providing useful reasoning principles.

The difficulty and complexity of the attempts to remove step-indexing
from the logical relation raises a question: is it worth it? The
motivations for step-index-free logical relations where basically the
following:
\begin{itemize}
\item Logical relations for step-indexing where complicated and
  painful to use and construct correctly.
\item Step-indexing is ad-hoc and difficult to justify.
\item Step-indexing is ill-suited for capturing properties beyond
  safety.
\end{itemize}
The first two issues are addressed in the same way: a generalization
in the way we view logical relations. If logical relations are just
structure preserving relations, there's no reason to limit ourselves
to traditional set-theoretic relations. Already with Kripke logical
relations there is the same indexing structure observed in Kripke
logical relations without step-indexing. Rather than viewing relations
as holding or not holding it is quite natural to replace them with a
more internsional, local notion of truth as is done in both
step-indexing and Kripke logical relations. If we make use of more
sophisticated tools for working with these relations, either by
working internally to a type theory, a higher-order logic, or a topos
much of the day-to-day issues of step-indexing can be resolved.

The last issue is the real problem. The nature of step-indexing is
inherently limiting because it is only is suited for properties that
are sound with respect to finite prefixes. Consider for instance, the
set of programs producing a stream of natural numbers. The program
outputting $1$, $2$, $3$, .. for any finite step $n$ produces a
maximum but it is not the case that it ever actually outputs a number
globally! This is a classic issue in mathematics: global truth is not
truth in every localization and step-indexing only allows us to
discuss local truth.

In order to fix this, one can either remove the indexing structure and
never work with local truth or add enough locales so that local truth
is strong enough. For instance, if we work with $\omega^2$ in the
example above the problematic program is ruled out: it does not
satisfy the program at $(\omega, 0)$. A fuller theory of logical
relation in this setting is given by \citet{Svendsen:16}.

Further investigation is needed in both directions. There are many
unstudied generalizations of relations beyond merely considering sets
of truth values. On the other hand, there is an undeniable appeal to
just being able to work with sets and to find some new clever
technique to avoid the indexing all together. At the moment though,
there seems to be far more unexplored and accessible space in the
first direction than the second; constructing a step-index-free
logical relation seems to require some new approach beyond
applications of classic tools as was done in this thesis.

%%% Local Variables:
%%% mode: latex
%%% TeX-master: "../main"
%%% End:
