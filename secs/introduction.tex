In the study of programming languages a great number of important
questions hinge on the equality of programs.
\begin{itemize}
\item The verification of a compiler pass is nothing but a
  question of equality of the naive program and its optimized
  version.
\item An optimized data structure may be shown to be correct relative to a
  much simpler reference solution
\item Much of the research in dependent type theory centers around
  what terms should be judged equal.
\end{itemize}
Given the role that equality plays in programming languages it is
unsurprising that a great variety of mathematical tools have been
developed to study it.

Before anything can be said about tools for analyzing equality,
however, a precise characterization of equality must be given. There
are many reasonable notions of equality between programs and all of
them are suitable for different circumstances. For instance, one can
compare programs up to renaming of variables, $\alpha$-equivalence, in
compilers and other applications where it is important that no
information is lost. In type-checkers for dependently typed languages,
it is natural to be more flexible and regard programs as the same if
they share a normal form, some sort of $\beta\eta$-equivalence at
least. In this work, we are interested in studying program behavior
and verifying properties about it so an even more relax notion of
equality is called for: \emph{contextual equivalence}.

Contextual equivalence is formally defined later (see
Definition~\ref{def:language:cxt}) but informally contextual
equivalence of two programs, $e_1 \cong e_2$, expresses that $e_1$ and
$e_2$ are internally indistinguishable. This means that there is no
program containing $e_1$ so that if $e_1$ is replaced with $e_2$ then
the program returns a different result. This notion of equivalence is
appealing because there is no aspect of a program we care about other
than what it computes to. Contextual equivalence allows us to ignore
unimportant differences in the precise way that the answer of a
program is computed and focus instead on the answer itself. This means
that two implementations of a data structure, for instance, are
contextually equivalent even if one is far more complex and efficient
than the other.

For all the appeal of contextual equivalence, it is very difficult to
establish any instances of it. In order to show that
$e_1 \cong e_2$ it is necessary to quantify over all possible programs
using $e_1$ and $e_2$ and reason about their behavior. This includes
programs which do not really make use of $e_1$ or $e_2$ such as
$\ap{(\lam{x}{\fn{\unit}{\tau}}{1})}{(\lam{x}{\unit}{e_1})}$. This lack
of constraints on how $e_1$ or $e_2$ is used is precisely what makes
contextual equivalence so useful but it makes even the most basic
proofs involved affairs. In order to compensate for this, a variety of
tools have been specialized to simplify the process of establishing when
$e_1 \cong e_2$.

Broadly speaking, there are two classes of these tools. One may
consider denotational approaches, where in the equality of programs is
reduced to the question of the equality of normal mathematical
objects. This line of study begins with Dana Scott's investigations of
the lambda calculus~\citep{Scott:76}. Such tools have been
immensely effective when they may be applied but they are difficult to
use. Complex programming languages often make use of extremely
sophisticated mathematical objects and using the model requires
understanding them. This has meant that denotational semantics is
traditionally out of reach for the verification of programs by an
average programmer or anyone besides a domain expert.

On the other hand, there are syntactic approaches to equality. Some of
these date back to the original study of the lambda calculus and its
reduction properties. Syntactic tools tend to be simpler to use,
relying only one elementary mathematics and an understanding of the
syntax itself. For equality, the tool of choice when working
syntactically is a logical
relation~\citep{Tait:67,Girard:72,Pitts:98,Appel:01,Ahmed:04,Ahmed:06,Appel:07,Dreyer:09,Dreyer:10}.
Dating back to~\citet{Tait:67} logical relations generalize
homomorphisms and can be thought of as \emph{structure preserving
  relations}\footnote{This viewpoint is more liberating than the
  traditional explanation that they are a type-indexed inductively
  defined family of relations. Logical relations arise for general
  mathematical structure. A programming language is merely one example
  of such structure.}. Logical relations are used for a wide variety
of tasks but the most important one for us is as a method for
demonstrating contextual equivalences. The study of logical relations
has given rise to a powerful generalization of the theory of data
abstraction called parametricity~\citep{Reynolds:83}. Parametric
reasoning has been so important that there have even been attempts to
reconstruct it in a denotational
sense~\citep{Bainbridge:90,Abadi:90,Ma:91,Birkedal:05,Dunphy:04}. This
work is important for crystallizing the connections between
naturality and parametricity as well as cementing parametricity in the
broader tradition of logical relations. At the present time this
program is still incomplete however. A summary of the state of the art
in 2013 and the fundamental challenges may be found in
\citet{Hermida:14}.

To be precise, a logical relation for us is a family of binary
relations indexed by the types of the language,
$(R_\tau)_{\tau \in \types}$. It is constructed by induction over the
types indexing it so that $R_{\fn{\tau_1}{\tau_2}}$ is defined in
terms of $R_{\tau_1}$ and $R_{\tau_2}$. For each $\tau$, the following
property is expected to hold:
\[
  (e_1, e_2) \in R_\tau \implies
  \hasE{e_1, e_2}{\tau} \land e_1 \cong e_2
\]
This property expresses the soundness of the logical relation with
respect to contextual equivalence. The reverse property, completeness,
is desirable but often unachievable\footnote{Instead a variety of
  techniques for \emph{forcing} completeness to hold are
  employed. These amount to adding all the missing identifications to
  the logical relation. While theoretically desirable, this is useless
  for the problem of actually establishing an
  equivalence.}. Completeness is usually unnecessary for establishing
certain concrete equivalences so in this context it is less
important. Finally, an important property of logical relations which
is difficult to capture formally is that it is much easier to prove
that $(e_1, e_2) \in R_\tau$ than to directly show that
$e_1 \cong e_2$. Typically, $R_\tau$ is meant to capture the logical
action of $\tau$; it expresses the precise set of observations
possible to make of expressions of type $\tau$ without all the
duplication of contextual equivalence. For instance, in order to show
that $(e_1, e_2) \in R_{\fn{\tau_1}{\tau_2}}$ it suffices to show the
following.
\[
  \forall (a_1, a_2) \in R_{\tau_1}.
  \ (\ap{e_1}{a_1}, \ap{e_2}{a_2}) \in R_{\tau_2}
\]
Given the obvious appeal of having a logical relation defined for a
language, it would seem that the construction of a logical relation
characterizing equality is a natural first step in the study of a new
programming language. The central stumbling block to this goal is that
logical relations are difficult to extend to new programming languages
and especially to new types and computational effects.

To see where trouble might arise, consider a language with polymorphic
types: $\allNoKind{\alpha}{\tau}$. What should the definition of
$R_{\allNoKind{\alpha}{\tau}}$ be? Phrased different, when are two
polymorphic functions indistinguishable? One might expect a definition
like the following:
\[
  (e_1, e_2) \in R_{\allNoKind{\alpha}{\tau}} \defs
  \forall \tau'.\ (\Ap{e_1}{\tau'}, \Ap{e_2}{\tau'}) \in R_{[\tau'/\alpha]\tau}
\]
This does correctly characterize contextual equivalence but it is
ill-suited as a definition in a logical relation because it will not
be well-founded. A logical relation is constructed by induction on the
types and with impredicative polymorphism there is absolutely no
reason why $\tau'$ should be smaller than
$\allNoKind{\alpha}{\tau}$. This is not a minor issue, defining a
correct logical relation for a language with impredicative
polymorphism requires the novel method of
candidates~\citep{Girard:71,Girard:72}. This situation has been a
recurring issue: features in programming languages tend to have some
recursive structure which prevents them from easily fitting into the
inductive definition of a logical relation. For many features, clever
constructions have been found which circumvent well-foundedness
issues: for instance parametric
polymorphism~\citep{Girard:71,Girard:72}, first-order
state~\citep{Pitts:98}, simple exceptions, and recursive
types~\citep{Birkedal:99,Crary:07}. Despite this work logical
relations are still a long way away from being able to cope with a
full programming language. The state of the art for dealing with
features like higher-order references, nontrivial control passing, or
concurrency is to use step-indexing.

Step-indexing is a technique first proposed by \citet{Appel:01}. It is
based on a simple but ingenious idea: if you the logical relation is
not well-defined by induction on the type just add a number which
decreases and induct on that. This idea means that a logical relation
is no longer a type-indexed relation but a relation indexed by a
number (called a \emph{step}) and a type, $(R_\tau^i)_{i, \tau}$.

The meaning of two programs related at step $i$ and type $\tau$ they
are indistinguishable at type $\tau$ for $i$ steps. Two programs are
related for $i$ steps intuitively if it will take at least $i$ steps
in order to make an observation which distinguishes them. For
instance, if a program $e_1$ runs to $\tt$ in 10 steps while $e_2$
runs to false in 20, it takes 20 steps to distinguish them since we
must wait for $e_2$ to finish running before they can be compared.

This idea of steps has proven to be incredibly robust and easily
extend to many different
settings~\citep{Appel:01,Ahmed:04,Ahmed:06,Appel:07,Dreyer:09,Dreyer:10,Birkedal:steps:11,Turon:13,Svendsen:16}.
There is a price to be payed for this extra flexibility. Firstly,
steps will now pervade the definition of the logical relation and any
proof now must contain pointless bookkeeping activity in order to keep
track of them. Secondly, the evaluation behavior of the programs under
consideration has begun to matter again. The goal of contextual
equivalence was to erase it from consideration. Now there are
instances where in order to use a logical relation a program must be
modified with spurious NO-OP statements in order to make it take extra
steps, a clear violation of this principle. These issues are even
present when just using the logical relation to validate contextual
equivalences~\citep{Svendsen:16}. In practice, this has meant that
while step-indexing is incredibly widely used it is almost universally
disliked\footnote{Citation: I walked around at POPL and the 5 people I
  asked seemed like REALLY bummed about it.}.

The structure of this thesis as follows. In Section~\ref{sec:language}
an ML-like language is described that will serve as target for the
logical relation. In Section~\ref{sec:steps} a step-indexed logical
relation for this language is constructed and discussed, especially to
highlight its short-comings. In Section~\ref{sec:domains} the first
failed step-index-free logical relation is described, the most na\"ive
approach centered around using domain theory. In
Sections~\ref{sec:smi} and~\ref{sec:guarded} a variety of attempts at
step-index-free logical relations are discussed for languages powerful
enough to encode general references, largely built around (guarded)
recursive kinds.

%%% Local Variables:
%%% mode: latex
%%% TeX-master: "../main"
%%% End:
