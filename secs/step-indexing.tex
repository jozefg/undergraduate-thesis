\section{A Step-Indexed Logical Relation}\label{sec:steps}

Before diving into the various approaches for constructing a logical
relation without step-indexing, it is well worth the time to see how
a logical relation can be done with it. The purpose of this section is
to sketch the complication intrinsic to any logical relation and show
how step-indexing obliterates them, though at a high cost.

Our logical to begin with a mapping from types to semantic types
(merely sets of terms). In order to handle impredicative polymorphism
Girard's method~\citep{Girard:71,Girard:72}, see \citet{TODO-PFPL} for
a comprehensive explanation of the technique. This means that our
logical relation is of the form
\[
  \den{-}_{-} : \types \to \typesEnv \to \pow{\term \times \term}
\]
The central challenge is of course the meaning of $\den{\cmd{\tau}}$:
the action of the logical relation at commands. At an intuitive level,
for two commands are rather like (partial) functions: they map heaps to heaps
and a return value. Drawing inspiration from how logical relations for
functions are defined, we might write the following for the definition
the logical relation.
\begin{align*}
  \den{\cmd{\tau}}_\eta& \triangleq \{(e_1, e_2) \mid\\
  &\exists m_1, m_2.\ \steps{e_i}{\cmd{m_i}} \land{}\\
  &\forall h_1 \sim h_2.
  \ (m_1, h_1) \simeq (m_2, h_2) \land{}\\
  &\quad \forall v_1, h_1', v_2, h_2'.
  \ (\stepsM{m_1}{h_1}{\ret{v_1}}{h_1'} \land \stepsM{m_2}{h_2}{\ret{v_2}}{h_2'})\\
  &\qquad \implies (h_1' \sim h_2' \land (v_1, v_2) \in \den{\tau}_\eta)
\end{align*}
Here left undefined is the definition of $\sim$ between two
heaps. This is in fact a major issue because there appears to be no
good way to identify when two heaps ought to be equal. The first issue
here is that semantic equality of terms (be it contextual or logical)
is type-indexed. This means that in order to compare heaps pointwise
for equality (a reasonable though still wrong idea) requires that we
at least know the types of the entries. Furthermore, we shouldn't
compare these heaps for equality at all locations necessarily, two
heaps should only need to agree on the cells that the programs are
going to use. This is a significant concept if we want to prove
programs to be equivalent which do not use the heap identically. For
instance, consider the two programs:
\[
  \dcl{\alpha}{1}{\ret{\cmd{\get{\alpha}}}} \qquad\qquad
  \ret{\cmd{\ret{1}}}
\]
These are contextually equivalent (the assignable of the first program
is hidden from external manipulation) and yet they allocate in
different ways. So $\sim$ must not be \emph{merely} pointwise equality
in the most general case. Additionally, proving that these two
programs are equal requires showing that $h_1 ~ h_2$ if and only if
$h_1(\alpha) = 1$. That is, this program doesn't merely require that
heap cells contain values of some syntactic type, but they may need to
belong to an arbitrary semantic type. In order to reconcile these
constraints, one thing is clear: the logical relation must somehow
vary depending on the state that the heap is supposed to be in. It is
simply not the case that programs that are equivalent in a heap where
no cells are required to exist if and only if they're equivalent in a
heap where one cell is required to exist.

The solution to this is called a Kripke logical
relation~\citep{TODO-KRIPKE}. Kripke logical relations, more generally
sheaf models, are a recurring phenomenon in computer
science~\citep{TODO-KRIPKE-STUFF}. The recurring theme of Kripke
logical relations, or sheaf models, is to abandon the notion of a
single global truth and judge truth relative to a current state of the
world. Rather than considering $(e_1, e_2) \in \den{\tau}_\eta$, we
should consider at some world $w$: $(e_1, e_2)
\den{\tau}_\eta(w)$. Now the collection of worlds should be
\emph{ordered} by a reachability relation $\reach$. Importantly,
Kripke logical relations should be monotone in relation to $\reach$:
\[
  \forall w_1, w_2.
  \ w_1 \reach w_2 \land (e_1, e_2) \in \den{\tau}_{w_1}
  \implies (e_1, e_2) \in \den{\tau}_{w_2}
\]
Intuitively, if we know that some fact holds at a world, $w_1$, and we
add more knowledge to $w_1$ to reach $w_2 \breach w_1$ it should not
reduce what we know to be true.

In this case, the choice of Kripke world is meant to be express the
current state of the heap that programs are being compared at, or at
least, what is known about it. What is this world concretely however?
As a first cut, one could consider a simple collection of symbols and
types. That is,
$\worlds \triangleq \powfin{\assignables \times \types}$. We can change
our clause for the logical relation to take these worlds into
account.
\begin{align*}
  \den{\cmd{\tau}}_\eta\different{(w_1)}& \triangleq \{(e_1, e_2) \mid\\
  &\exists m_1, m_2.\ \steps{e_i}{\cmd{m_i}} \land{}\\
  &\forall \different{w_2 \breach w_1}.\ \forall h_1 \different{\sim_{w_2}} h_2.
  \ (m_1, h_1) \simeq (m_2, h_2) \land{}\\
  &\quad \forall v_1, h_1', v_2, h_2'.
  \ (\stepsM{m_1}{h_1}{\ret{v_1}}{h_1'} \land \stepsM{m_2}{h_2}{\ret{v_2}}{h_2'})\\
  &\qquad \implies (\different{\exists w_3 \breach w_2}.
    \ h_1' \different{\sim_{w_3}} h_2' \land (v_1, v_2) \in \den{\tau}_\eta\different{(w_3)})
\end{align*}
This addition of the Kripke worlds is largely forced. We must quantify
over all possible $w_2$ extending $w_1$ at the beginning: if this was
elided then $\den{\cmd{\tau}}_\eta$ would not be monotone. The
extension at the end, $w_3 \breach w_2$, is so that the world may be
updated to reflect the changes that were caused by allocating new
cells or updating existing ones. Still unexplained is
$h_1 \sim_w h_2$. At this point it can be defined in a slightly more
refined way since $w$ at least specifies what cells we ought to
compare for equality. The notion of equality that we want is
problematic though.
\begin{itemize}
\item If some stronger notion of equality than logical equivalence,
  such as $\alpha$-equivalence, is used the fundamental theorem will
  fail in the clause for $\set{\alpha}{e}$.
\item If a weaker equality than logical equivalence is used then the
  fundamental theorem will fail in the clause for $\get{\alpha}$.
\item If logical equivalence itself is used, the definition is will
  become ill-founded. This is because the heap may contain cells with
  a type larger than $\cmd{\tau}$.
\end{itemize}
What is needed is an judo throw in the vein of Girard's
method. Instead of attempting to decide what the equality for a
particular heap location should be in the definition of $\sim_w$, it
should be told to us already by $w$. This idea, originating with
\citet{TODO-PITTS-AND-STARK}, means that our Kripke worlds should
instead satisfy the relation:
\[
  \worlds = \assignables \pto \pow{\term \times \term}
\]
Now the world extension relation is defined by the following.
\[
  w_1 \reach w_2 \triangleq
  \dom(w_1) \subseteq \dom(w_2) \land
  \forall \alpha.\ w_1(\alpha) = w_2(\alpha)
\]
This version is much more plausible. With the definition of the clause
of the logical relation described previously together with the
following definition of $\sim_w$ the logical relation is well-defined.
\[
  h_1 \sim_w h_2 \triangleq \forall \alpha \in \dom(w).
  \  (h_1(\alpha), h_2(\alpha)) \in w(\alpha)
\]
The issue here is more subtle and causes the fundamental theorem to
fail in the rule for allocation: what relation should we pick when a
fresh cell is allocated? It seems that the only choice when allocating
a cell of type $\tau$ at world $w$ is to extend our world with
$\den{\tau}_\eta(w)$. The complication arises when we allocate more
cells later and move to a fresh world. The relation at $\alpha$ is now
stale, it refers to an outdated world and doesn't allow for
equivalences which are true at this new world but were previously
false. To concretely see this, consider the program:
\begin{align*}
  \cmd{
    &\dcl{\alpha}{\lam{x}{\nat}{\cmd{\ret{1}}}}{\\
      &\quad \ret{\cmd{\set{\alpha}{\lam{x}{\nat}{
          \bnd{x}{\cmd{\get{\alpha}}}{\\ & \qquad\qquad\qquad\qquad\ret{x + 1}}
        }}}}
    }
  }
\end{align*}
This style of program is used to encode recursion in this language in
general. In this case, however, the central point of interest is that
$\alpha$ is updated to contain a command which mentions $\alpha$. In
logical relation, $\alpha$ could only ever contain terms in
$\den{\fn{\nat}{\cmd{\nat}}}(\emptyset)$. In particular, this never
includes a command which mentions $\alpha$. This prevents this
perfectly type safe program from being included in the logical
relation. This will mean that the fundamental theorem fails.

What is to be done here? The root of the issue is that when we
allocate a cell it is impossible to determine precisely what programs
will occupy it because programs in the cell may mention cells that are
yet to be allocated at the time of the construction. What is needed is
for the semantic type stored in a heap cell to vary according to the
world. This fixes an asymmetry between the Kripke world and the
logical relation: the Kripke world supposedly maps locations to
semantic types but the semantics types (as determined by the logical
relation) vary of the Kripke worlds. This leads us to the final form
of the definition of Kripke worlds.
\[
  \worlds = \assignables \pto (\worlds \to \pow{\term \times \term})
\]
Herein lies the rub, this definition of the set of worlds is precisely
what is required for this logical relation. It is not a set
though. A simple cardinality argument shows that there can be no such
set since it would have to be larger than its own power set. This is
not an easily avoided problem. It is unknown how to simply avoid this
using ingenuity in the choice of the Kripke world.

This is where step-indexing enters the picture. Step-indexing after
all was introduced to break precisely these sort of circularities and
``solve'' recursive equations up to an approximation.

With step-indexing, the space of semantic types becomes indexed by
natural numbers and we will likewise index the world by these same
natural numbers. The idea is that a world at stage $n$ maps to
assignables to semantic types which only vary at the previous $n - 1$
stages, past $n - 1$ they are simply constant. The issue is that this
approach requires a number of complex definitions which obscure the
underlying intent: to solve this recursive equation.

Instead of slogging through the classical step-indexed definitions, we
will make use of a more modern categorical approach. Instead of
working with sets, from the beginning we will work with sets varying
over natural numbers: presheaves over $\omega$.
\begin{defn}
  A presheaf over a category $\Ccat$ is a functor from $\op{\Ccat}$ to
  $\SET$. Presheaves form a category with morphisms being natural
  transformations. This category is written $\presheaves{\Ccat}$.
\end{defn}
\begin{example}
  A presheaf over $\omega$ is a family of sets $(X_i)_{i \in \nat}$
  with a map $r_n : X_{n + 1} \to X_n$ for all $n$ called restriction
  maps. A map between presheaves is then a family of maps
  $f_n : X_n \to Y_n$ so that the following commutes for all $n$.
  \[
    \begin{tikzcd}
      X_{n + 1} \ar[r, "f_{n + 1}"] \ar[d, swap, "r_n"] & Y_{n + 1} \ar[d, "r_n"]\\
      X_n \ar[r, swap, "f_n"] & Y_n
    \end{tikzcd}
  \]
\end{example}
Rather than working directly with sets then, we can instead work with
presheaves synthetically. For instance, we can define the exponential
of two presheaves rather than mucking about to create an implication
which interacts properly with the step.
\begin{lem}
  The (co)limit (a generalized version of products and sums) of
  presheaves is determined pointwise.
\end{lem}
The above theorem tells us, for instance, that if we want to form the
product of two presheaves, $X$ and $Y$, over $\omega$, at time $n$ it's just the
product of the $X(n)$ and $Y(n)$. More generally, this lemma gives a
wide-variety of ways to construct complex presheaves from simple ones
without ever having to deal with the step manually. The main missing
element is the ability to form exponentials, the categorical analog of
functions. For presheaves these are slightly more complicated since
the function must respect the indexing structure.
\begin{lem}
  Given two presheaves $X, Y : \presheaves{\Ccat}$, the exponential
  between them is
  \[
    (Y^X)(c) \triangleq \hom(X \times y(c), Y)
  \]
  where $y$ is the Yoneda embedding, defined by the following.
  \[
    y(c) = hom_\Ccat(-, c)
  \]
\end{lem}
This definition may seem abstract but it is, importantly,
monotone and so determines a valid presheaf. The real power of this
approach is that \emph{it does not matter} that this definition is
complex. The point is that this definition determines a function of
presheaves in that it contains elements we can apply and only elements
we can apply and beyond that its construction is entirely
irrelevant. The categorical methodology of only caring about how an
object relates through morphisms to the rest of the category is
designed to let us avoid having to think too hard about any precise
construction of an object. % TASTE IT.

The constructions so far have given us a wide variety of constructible
presheaves but nothing thus far has increased our expressive power of
what we had in $\SET$. For that we need the ability to solve certain
recursive equations. In fact, it will turn out that we can solve
\emph{guarded} domain equations in $\presheaves{\omega}$. In order to
see this, first let us define a functor on $\presheaves{\omega}$.
\begin{defn}
  The later functor,
  $\later : \presheaves{\omega} \to \presheaves{\omega}$, is defined
  on objects as follows.
  \begin{align*}
    (\later X)(n + 1) &= X(n)\\
    (\later X)(0) &= \{\star\}
  \end{align*}
  This family of sets is a presheaf in the obvious way:
  $r_0 = \lambda x.\ \star$ and $r_{n + 1}$ is the $r_n$ of $X$. The
  action on morphisms is as follows.
  \begin{align*}
    (\later f)(n + 1) &= f_n\\
    (\later f)(0) &= \lambda x.\ \star
  \end{align*}

  The later functor comes equipped with a natural transformation
  $\delay : 1 \to \later$. Explicitly, this means $\delay$ is a family
  of maps $\delay_X : X \to \later X$ so that
  $\delay_Y \circ f = f \circ \delay_X$ for any $f : X \to Y$.
\end{defn}
The later modality is the logical essence of step-indexing. The key
idea of step-indexing, after all, is not merely that it suffices to
ensure that two programs are related for all steps in order to
conclude that they're related. The second, subtler, key idea is that
in order for two programs to be related for $n$ steps, a subcomponent
of these programs needs to be related for only $n - 1$ steps. If this
was not the case, then it would never be possible to decrement the
step-index and the exercise would have been largely moot. The later
modality internalizes the idea of ``$n - 1$ steps'' and so it lets us
talk about (uniformly) constructing an element at stage $n$ from an
element at stage $n - 1$. The $\delay$ operation even crystallizes the
monotonously of the logical relation: if we have an element at stage
$n$ we can always construct one at stage $n - 1$.

A crucial point about using natural numbers for step-indexing is that
natural numbers are well-founded. There's no way to pick a natural
number that we can decrement forever. This justifies reasoning by
induction on the natural numbers. In particular, it justifies
constructing an object at any stage provided we can take a
construction of it at stage $n - 1$ and construct it at stage
$n$. This principle may be expressed with the later modality: it's
precisely the existence of a family of morphisms:
\[
  \fix_A : A^{\later A} \to A
\]
These morphisms allow us to take a particular sort of fixed-point at
any level of maps $\later A \to A$. These morphisms belong to the
class of contractive morphisms: morphisms $X \to Y$ that can be
factored as $X \to \later X \to Y$. In particular, suppose we have
some $f : B \times \later A \to A$. Then the following equality holds:
\[
  \fix_A \circ \transpose{f} = f \circ \pair{1}{\delay_A \circ \fix_A \circ \transpose{f}}
\]
Written out using more type-theoretic notation this is easier to
process.
\[
  \forall a, b.\ \fix_A(f(b, -))(a) = f (b, \delay_A(\fix_A(f(b))(a)))
\]
This fixed-point construction allows us to build many interesting
structures inside this category. Through the inclusion of
universes~\citep{TODO-UNIVERSES}, one can even construct solutions to
certain (small) domain equations entirely internally to the
category~\citep{TODO-QUACK-FIXED-POINTS}. The basic idea is that
$\presheaves{\omega}$ can be easily made to support an object,
$\mathcal{U}$, for which global sections classify other smaller
objects. This is the categorical analog of a Grothendieck
universe~\citep{TODO-GROTHENDIECK} and proves to be an important
concept in the semantics of dependent type theory. Then, by using
$\fix_\univ$ it is possible to construct the recursively defined
presheaves necessary to build logical relations. This approach, while
not explicit, is latent in the work done in constructing logical
relations in guarded type theory and guarded
logics~\citep{TODO-LR-IN-GUARDED}.

We will consider the more traditional approach of using proper domain
equations rather than just small ones. In this case, the traditional
domain theoretic approach is to use functors and to construct fixed
points of them. This approach in domain theory originates from
\citet{TODO-PLOTKIN-SMYTH} and subsumed prior work by Scott
constructing recursive domain equations by
hand~\citep{TODO-SCOTT}. One potential issue is that we would like to
solve domain equations which are not strictly speaking functorial. In
particular, we want to handle the case where the equation contains
both positive and negative occurences. We handle these by viewing them
as functors
$F : \op{\presheaves{\omega}} \times \presheaves{\omega} \to
\presheaves{\omega}$
in the style of \citet{TODO-MIXED-VARIANCE}. The question then
becomes, for what functors can we construct an object, $I$, so that
$F(I, I) \cong I$, an invariant object. This construction is a
generalization of the theorem of America and
Rutten~\citep{TODO-AMERICA}. The precise accounting the theorem can
be found in \citet{Birkedal:11} allowing for any enriched functor
between categories enriched in sheaves over a complete Heyting algebra
with a well-founded basis. For our purposes though a much less general
theorem suffices.
\begin{defn}
  A functor, $F$, is said to be locally contractive if there is a
  family of contractive morphisms
  \[
    f_{X_1, X_2, Y_1, Y_2} :
    X_1^{X_2} \times Y_2^{Y_1} \to F(X_2, Y_2)^{F(X_1, Y_1)}
  \]
  So that this family is family respects composition and identity and
  so that for each $\pair{g}{h} : 1 \to X_1^{X_2} \times Y_2^{Y_1}$
  then the following equation holds.
  \[
    \transpose{f_{X_1, X_2, Y_1, Y_2} \circ \pair{g}{h}} =
    F(\transpose{g}, \transpose{h})
  \]
  If just the latter condition holds $F$ is said to have a strength
  $f$ and if the family of morphisms is not contractive we shall call
  the functor locally nonexpansive.
\end{defn}
This definition is subtly different than some presentations of this
theorem, which require only that $F$ have a strength comprised of
contractive morphisms. The central theorem,
Theorem~\ref{thm:steps:fixedpoints} is true even in this weakened
version but I am unable to find a proof of this fact and I cannot list
a theorem that I cannot prove myself. The relative strengthening that
this generality provides seems vacuous as well, none of the examples
of the fixed point theorem used either in this section or
Section~\ref{sec:guarded} require it.
\begin{example}
  The functors $\times$, $\to$, and $+$ are all locally
  nonexpansive. The functor $\later$ is locally
  contractive. Additionally, any locally contractive functor is locally
  nonexpansive.
\end{example}
\begin{lem}
  The composition of locally nonexpansive functors is locally
  nonexpansive. The composition of a locally contractive functor with
  a locally nonexpansive functor is locally contractive.
\end{lem}
We may now state the theorem which makes working a
$\presheaves{\omega}$ so appealing.
\begin{thm}\label{thm:steps:fixedpoints}
  For any locally contractive functor $F$ there exists an object $I$,
  unique up to isomorphism, so that $F(I, I) \cong I$.
\end{thm}
This theorem allows us to solve a wide variety of domain equations,
including a small modification of the crucial one for Kripke
worlds. First, let us define $\upred{S}$ as a ``time varying
predicate'' on a set $S$. This will be the analog of the sets of terms
used in step-indexed models.
\[
  \upred{X}(n) \triangleq \{(m, x) \mid x \in X \land m < n\}
\]
There is an evident restriction mapping sending a uniform predicate at
stage $n + 1$ to the uniform predicate containing the entries indexed
by numbers smaller than $n$. Next, the presheaf of partial maps from
assignables to presheaves can be defined as follows:
\[
  (\assignables \pto X)(n) \triangleq
  \{f \in X(n)^{F} \mid F \subseteq \assignables \land F\ \mathsf{finite}\}
\]
Importantly, it is not hard to see that $\assignables \pto -$ defines
a functor which is locally nonexpansive. The final piece necessary is
a subobject of the exponential $Y^{\assignables \pto X}$ which
isolates those functions which are monotone. In order to define this,
we must assume that $Y$ comes equipped with a partial order. Viewed
from the external perspective this is a family of relations indexed by
natural numbers which respects reindexing. If one views the partial
order internally, however, it is just a normal partial order. The
subobject desired can be defined internally in a way which is
obviously correct using the internal logic of $\presheaves{\omega}$ as
follows.
\begin{align*}
  (\assignables \pto X) \mto Y &\triangleq \\
  \{f : Y^{\assignables \pto X} \mid{}&
   \forall w_1 w_2 : \assignables \pto X.\ w_1 \reach w_2 \implies f(w_1) \le f(w_2)\}
\end{align*}
Constructed externally this presheaf is somewhat more difficult to
construct, but still quite possible. It's
\begin{align*}
  ((\assignables \pto X) \mto Y)(n) &= \\
  \{f : (Y^{\assignables \pto X})(n) \mid{}&
   \forall m \le n.\ \forall w_1 w_2 : (\assignables \pto X)(n).\\
  &\qquad w_1 \reach_n w_2 \implies f(m)(\star, w_1) \le_n f(m)(\star, w_2)\}
\end{align*}
This external definition is neither more precise nor more intuitive,
so it is more helpful to read the internal version and understand that
all the quantifiers in the predicate implicitly handle the steps
correctly.

At the end of these constructions, we may define the following
contractive functor.
\[
  X \mapsto (\assignables \pto \later X) \mto \upred{\term \times \term}
\]
Solving this for a fixed point gives the desired presheaf of Kripke worlds.

%%% Local Variables:
%%% mode: latex
%%% TeX-master: "../main"
%%% End:
