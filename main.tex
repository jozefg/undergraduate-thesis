\documentclass{amsart}
\usepackage[hyperfootnotes=false]{hyperref}
\usepackage[sort&compress,square,comma,numbers]{natbib}
\usepackage{amsthm}
\usepackage{amsmath}
\usepackage{amssymb}
\usepackage{mathpartir}
\usepackage{stmaryrd}
\usepackage{xifthen}
\usepackage{tikz-cd}
\usepackage{xcolor}

% Formatting tools
\newcommand{\declareJudgement}[1]{\framebox{$\displaystyle{}{#1}$}}
\newcommand{\different}[1]{{\color{red} #1}}
% Theorems
\newtheorem{thm}{Theorem}[section]
\newtheorem{cor}[thm]{Corollary}
\newtheorem{lem}[thm]{Lemma}
\newtheorem{remark}[thm]{Remark}
\newtheorem{example}[thm]{Example}
\newtheorem{defn}[thm]{Definition}

% Categories stuff
\newcommand{\Ccat}{\ensuremath{\mathbb{C}}}
\newcommand{\Dcat}{\ensuremath{\mathbb{D}}}
\newcommand{\op}[1]{\ensuremath{#1^{\mathsf{op}}}}
\newcommand{\SET}{\ensuremath{\mathbf{Set}}}
\newcommand{\DCPO}{\ensuremath{\mathbf{Dcpo}}}
\newcommand{\presheaves}[1]{\ensuremath{\widehat{#1}}}
\newcommand{\transpose}[1]{\ensuremath{\widehat{#1}}}
\newcommand{\pair}[2]{\ensuremath{\left\langle #1, #2 \right\rangle}}
\newcommand{\univ}{\ensuremath{\mathcal{U}}}
\newcommand{\mono}{\ensuremath{rightarrowtail}}
\newcommand{\yoneda}{\ensuremath{\mathsf{y}}}

% Math stuffs
\newcommand{\upred}[1]{\ensuremath{\mathsf{UPred}(#1)}}
\newcommand{\reach}{\ensuremath{\mathrel{\sqsubseteq}}}
\newcommand{\breach}{\ensuremath{\mathrel{\sqsupseteq}}}
\newcommand{\mto}{\ensuremath{\overset{\mathsf{mon}}{\to}}}
\newcommand{\pto}{\ensuremath{\rightharpoonup}}
\newcommand{\pow}[1]{\ensuremath{\mathcal{P}(#1)}}
\newcommand{\powfin}[1]{\ensuremath{\mathcal{P_{\mathrm{fin}}}(#1)}}
\newcommand{\card}[1]{\ensuremath{\left\vert #1 \right\vert}}
\newcommand{\real}{\ensuremath{\mathbb{R}}}
\newcommand{\nat}{\ensuremath{\mathbb{N}}}
\newcommand{\cbult}{\ensuremath{\mathrm{CBUlt}}}
\newcommand{\cle}{\ensuremath{\lesssim}}
\newcommand{\ceq}{\ensuremath{\cong}}
\newcommand{\relR}{\ensuremath{\mathrel{\mathcal{R}}}}
\newcommand{\relS}{\ensuremath{\mathrel{\mathcal{S}}}}
\newcommand{\AND}{\ensuremath{\mathrel{\wedge}}}
\newcommand{\OR}{\ensuremath{\mathrel{\vee}}}
\newcommand{\den}[1]{\ensuremath{\llbracket #1 \rrbracket}}
\newcommand{\definitely}[1]{\ensuremath{\lceil #1 \rceil}}
\newcommand{\possibly}[1]{\ensuremath{\lfloor #1 \rfloor}}
\newcommand{\defs}{\ensuremath{\mathrel{\triangleq}}}
\newcommand{\bifix}{\ensuremath{\mathsf{bifix}}}
\newcommand{\disjoint}{\ensuremath{\mathop{\#}}}
\newcommand{\hole}{\ensuremath{\square}}
\newcommand{\sincl}[1]{\ensuremath{\mathsf{succ}(#1)}}
\newcommand{\join}{\ensuremath{\bigvee}}
\newcommand{\cless}{\ensuremath{\lessapprox}}
\newcommand{\aless}{\ensuremath{\lesssim}}

\DeclareMathOperator{\dom}{Dom}

% Sets
\newcommand{\states}{\ensuremath{\mathrm{State}}}
\newcommand{\worlds}{\ensuremath{\mathrm{World}}}
\newcommand{\assignables}{\ensuremath{\mathrm{Assignable}}}
\newcommand{\nrel}[1]{\ensuremath{\mathbb{R}_{#1}}}
\newcommand{\semtypes}{\ensuremath{\mathbb{T}}}
\newcommand{\types}{\ensuremath{\mathrm{Type}}}
\newcommand{\typesEnv}{\ensuremath{\mathrm{TypeEnv}}}
\newcommand{\term}{\ensuremath{\mathrm{Term}}}
\newcommand{\urel}{\ensuremath{\mathrm{URel}}}

% judgments
\newcommand{\guardJ}[2]{\ensuremath{\text{$#1.\,#2$\ \textsf{guarded}}}}
\newcommand{\hasE}[2]{\ensuremath{#1 \mathrel{:} #2}}
\newcommand{\hasM}[2]{\ensuremath{#1 \mathrel{\div} #2}}
\newcommand{\hasKJ}[2]{\ensuremath{#1 \vdash \hasE{#2}{\mathrm{\kind}}}}
\newcommand{\hasTJ}[4]{\ifthenelse{\isempty{#1}}%
  {\ensuremath{#2 \vdash \hasE{#3}{#4}}}%
  {\ensuremath{#1;#2 \vdash \hasE{#3}{#4}}}}
\newcommand{\hasEJ}[5]{\ifthenelse{\isempty{#1}}%
  {\ensuremath{#2; #3 \vdash \hasE{#4}{#5}}}%
  {\ensuremath{#1; #2; #3 \vdash \hasE{#4}{#5}}}}
\newcommand{\hasESigJ}[5]{\ensuremath{#1; #2 \vdash_{#3} \hasE{#4}{#5}}}
\newcommand{\hasMJ}[5]{\ensuremath{#1; #2 \vdash_{#3} \hasM{#4}{#5}}}
\newcommand{\hasCEJ}[9]{\ensuremath{#5 : (#1; #2 \vdash_{#3} #4) \rightsquigarrow (#6; #7 \vdash_{#8} #9)}}

\newcommand{\subKJ}[3]{\ifthenelse{\isempty{#1}}%
{\ensuremath{{#2} \leq {#3}}}%
{\ensuremath{{#1} \vdash {#2} \leq {#3}}}}

\newcommand{\equivESigJ}[6]{\ensuremath{#1; #2 \vdash_{#3} \hasE{#4 \cong #5}{#6}}}
\newcommand{\approxESigJ}[6]{\ensuremath{#1; #2 \vdash_{#3} \hasE{#4 \aless #5}{#6}}}
\newcommand{\equivMJ}[6]{\ensuremath{#1; #2 \vdash_{#3} \hasM{#4 \cong #5}{#6}}}

\newcommand{\step}[2]{\ensuremath{#1 \mapsto #2}}
\newcommand{\steps}[2]{\ensuremath{#1 \mapsto^* #2}}
\newcommand{\stepM}[4]{\ensuremath{(#1, #2) \mapsto (#3, #4)}}
\newcommand{\stepsM}[5][*]{\ensuremath{(#2, #3) \mapsto^{#1} (#4, #5)}}

\newcommand{\dec}[4][i]{\ensuremath{#2 \rhd^{#1} #3 : #4}}
\newcommand{\valueJ}[1]{\ensuremath{#1\ \mathsf{value}}}
\newcommand{\finalJ}[2]{\ensuremath{(#1, #2)\ \mathsf{final}}}

% language
\newcommand{\kind}{\ensuremath{\mathsf{kind}}}
\newcommand{\later}{\ensuremath{{\blacktriangleright}}}
\newcommand{\ilater}{\ensuremath{{\triangleright}}}
\newcommand{\rec}[2]{\ensuremath{\mu #1.\, #2}}
\newcommand{\fn}[2]{\ensuremath{#1 \to #2}}
\newcommand{\tp}{\ensuremath{\mathsf{T}}}
\newcommand{\unit}{\ensuremath{\mathsf{unit}}}
\newcommand{\cmd}[1]{\ensuremath{\mathsf{cmd}(#1)}}

\newcommand{\ap}[2]{\ensuremath{#1\ #2}}
\newcommand{\lam}[3]{\ensuremath{\lambda #1 {:} #2.\, #3}}
\newcommand{\into}[1]{\ifthenelse{\isempty{#1}}%
  {\ensuremath{\mathsf{in}}}%
  {\ensuremath{\mathsf{in}\,#1}}}
\newcommand{\out}[1]{\ifthenelse{\isempty{#1}}%
  {\ensuremath{\mathsf{out}}}%
  {\ensuremath{\mathsf{out}\,#1}}}
\newcommand{\delay}{\mathsf{next}}
\newcommand{\fix}{\mathsf{fix}}
\newcommand{\letdelay}[3]{\ensuremath{\mathsf{let\ next\ } #1 = #2 \mathsf{\ in\ } #3}}
\newcommand{\all}[3]{\ensuremath{\forall #1 {:} #2.\, #3}}
\newcommand{\allNoKind}[2]{\ensuremath{\forall #1.\, #2}}

\newcommand{\Ap}[2]{\ensuremath{#1[#2]}}
\newcommand{\Lam}[3]{\ensuremath{\Lambda #1 {:} #2.\, #3}}

\newcommand{\LamNoKind}[2]{\ensuremath{\Lambda #1.\, #2}}
\newcommand{\ret}[1]{\ensuremath{\mathsf{ret}(#1)}}
\newcommand{\get}[1]{\ensuremath{\mathsf{get}[#1]}}
\newcommand{\set}[2]{\ensuremath{\mathsf{set}[#1](#2)}}
\newcommand{\dcl}[3]{\ensuremath{\mathsf{dcl}\ #1 := #2\ \mathsf{in}\ #3}}
\newcommand{\bnd}[3]{\ensuremath{\mathsf{bnd}\ #1 \gets #2;\ #3}}

\newcommand{\trunc}{\ensuremath{\mathsf{trunc}}}

\newcommand{\littrue}{\ensuremath{\mathsf{true}}}
\newcommand{\litfalse}{\ensuremath{\mathsf{false}}}

\newcommand{\zap}{\ensuremath{\circledast}}


\title{The Next 700 Failed Step-Index-Free Logical Relations}
\author{Daniel Gratzer}
\date{\today}

\begin{document}
\begin{abstract}
  An important question in programming languages is the study of
  program equivalence. This is done typically with the construction of
  a relational denotational model or a syntactic analogue, called a
  logical relation. Logical relations have proven to be an effective
  tool for analyzing programs and lending formal weight to ideas like
  data abstraction and information hiding. A central difficulty with
  logical relations is their fragility; it has proven to be a
  challenge to scale logical relations to more realistic languages. In
  this thesis, we discuss a number of methods for the construction of
  a logical relation for a language with general references. None of
  these methods are sufficient to define a logical relation without
  resorting to step-indexing, demonstrating the difficulty of
  expressing the recursive structure of higher-order references.
\end{abstract}
\maketitle

In the study of programming languages a great number of important
questions can be reduced to questions about the equality of
programs. The verification of a compiler pass is nothing but a
question of equality of the naive program and its optimized version,
an optimized data structure may be shown to be correct relative to a
much simpler reference solution, a type theory hinges on the
construction of definitional equality and so on. Given the role that
equality plays in programming languages, a great variety of
mathematical tools have been developed to analyze the various notions
of equality possible in a programming language.

When in this thesis equality is discussed, it is generally meant to be
\emph{contextual equivalence}. Contextual equivalence is formally
defined later (see
Definition~\ref{defn:language:contextualEquivalence}) but informally
contextual equivalence of two programs, $e_1 \cong e_2$, expresses
that $e_1$ and $e_2$ are internally indistinguishable. This means that
there is no program containing $e_1$ so that if $e_1$ is replaced with
$e_2$ then the program returns a different result. This notion of
equivalence is appealing because there is no aspect of a program we
care about other than how it computes. Contextual equivalence allow to
ignore unimportant differences in the precise way that the answer of a
program is computed and focus instead on the answer itself. This means
that two well-behaved implementations of a data structure, for
instance, are contextually equivalent even if one is far more complex
and efficient than the other.

For all the appeal of contextual equivalence, it is very difficult to
establish that two programs are contextually equivalent. In order to
establish that $e_1 \cong e_2$ it is necessary to quantify over all possible
programs using $e_1$ and $e_2$ and reason about their behavior. This
includes programs which do not really make use of $e_1$ or $e_2$ such
as $\ap{(\lam{x}{\fn{\unit}{\tau}}{1})}{\lam{x}{\unit}{e_1}}$. This
lack of constraints on how $e_1$ or $e_2$ is used is precisely what
makes contextual equivalence so useful but it makes even the most
basic proofs involved affairs. In order to compensate for this, a
variety of tools have been developed to simply the process of
establishing when $e_1 \cong e_2$.

Broadly speaking, there are two classes of these tools. One may
consider denotational approaches, where in the equality of programs is
reduced to the question of the equality of normal mathematical
objects. This line of study begins with Dana Scott's investigations of
the lambda calculus~\citep{TODO-CITE-SCOTT}. These tools have been
immensely effective when they may be applied but they are difficult to
use. Complex programming languages often make use of extremely
sophisticated mathematical objects and using the model requires
understanding them. This has meant that denotational semantics is
traditionally out of reach for the verification of programs by an
average programmer or anyone besides a domain expert.

On the other hand, there are syntactic approaches to equality. Some of
these date back to the original study of the lambda calculus and its
reduction properties. Syntactic tools tend to be simpler to use,
relying only one elementary mathematics and an understanding of the
syntax itself. For equality, the tool of choice when working
syntactically is a logical relation~\citep{TODO-TAIT-AMAL}. Dating
back to~\citet{TODO-TAIT} logical relations crucially define a
type-indexed notion of equality. Logical relations have proven
important for validating certain obviously desirable properties such
as function extensionality. They have also given rise to a powerful
theory of data abstraction called
parametricity~\citep{TODO-REYNOLDS}. Parametric reasoning as inspired
by logical relations has been so important that there have even been
attempts to reconstruct it in a purely abstract denotational
sense~\citep{TODO-FIBRATIONAL}.

To be precise, a logical relation in for us is a family of sets
indexed by the types of the language, $(R_\tau)_\tau$. It is
constructed by induction over the types indexing it so that
$R_{\fn{\tau_1}{\tau_2}}$ is defined in terms of $R_{\tau_1}$ and
$R_{\tau_2}$. For each $\tau$, the following property is expected to
hold:
\[
  (e_1, e_2) \in R_\tau \implies
  \hasE{\cdot}{e_1, e_2}{\tau} \land e_1 \cong e_2
\]
This property expresses the soundness of the logical relation with
regards to contextual equivalence: logical equivalence implies
contextual equivalence. The reverse property, completeness, is
desirable but often unachievable\footnote{Instead a variety of
  techniques for \emph{forcing} completeness to hold are
  employed. These amount to adding all the missing identifications to
  the logical relation. While theoretically desirable, this is useless
  for the problem of actually establishing an
  equivalence.}. Completeness is usually unnecessary for establishing
certain concrete equivalences so in this context it is less
important. Finally, an important property of logical relations which
is difficult to capture formally is that it is much easier to prove
that $(e_1, e_2) \in R_\tau$ than to directly show that
$e_1 \cong e_2$. Typically, $R_\tau$ is meant to capture the logical
action of $\tau$; it expresses the precise set of observations
possible to make of expressions of type $\tau$ without all the
duplication of contextual equivalence. For instance, in order to show
that $(e_1, e_2) \in R_{\fn{\tau_1}{\tau_2}}$ it suffices to show the
following.
\[
  \forall (a_1, a_2) \in R_{\tau_1}.
  \ (\ap{e_1}{a_1}, \ap{e_2}{a_2}) \in R_{\tau_2}
\]
Given the obvious appeal of having a logical relation defined for a
language, it would seem that the construction of a logical relation
characterizing equality is a natural first step in the study of a new
programming language. The central stumbling block to this goal is that
logical relations are difficult to extend to new programming languages
and especially to new types and computational effects.

For an instance of where trouble might arise, consider a language with
polymorphic types: $\allNoKind{\alpha}{\tau}$. What should the
definition of $R_{\allNoKind{\alpha}{\tau}}$ be? Phrased different,
the question is when are two polymorphic functions
indistinguishable. One might expect a definition like the following.
\[
  (e_1, e_2) \in R_{\allNoKind{\alpha}{\tau}} \defs
  \forall \tau'.\ (\Ap{e_1}{\tau'}, \Ap{e_2}{\tau'}) \in R_{[\tau'/\alpha]\tau}
\]
This does correctly characterize contextual equivalence but it is
ill-suited as a definition in a logical relation because it will not
be well-founded. A logical relation is constructed by induction on the
types and with impredicative polymorphism there is absolutely no
reason why $\tau'$ should be smaller than
$\allNoKind{\alpha}{\tau}$. This is not a minor issue, defining a
correct logical relation for a language with impredicative
polymorphism requires the novel method of
candidates~\citep{TODO-GIRARD}. This problem has been a recurring
issue: features in programming languages tend to have some recursive
structure which prevents them from easily fitting into the inductive
definition of a logical relation. For many features clever
constructions have been found which circumvent well-foundedness
issues: for parametric polymorphism~\citep{TODO-GIRARD}, for
first-order state~\citep{TODO-PITTS-AND-STARK}, simple
exceptions~\citep{TODO-I-DUNNO-SURELY-NOT-NUPRL}, and for recursive
types~\citep{TODO-CRARY-AND-HARPER}. Despite this work logical
relations are still a long way away from being able to cope with a
full programming language. The state of the art for dealing with
features like higher-order references, nontrivial control passing, or
concurrency is to use step-indexing.

Step-indexing is a technique first proposed by \citet{TODO-APPEL}. It
is based on a simple but ingenious idea: if you the logical relation
is not well-defined by induction on the type just add a number which
decreases and induct on that. This idea means that a logical relation
is no longer a type-indexed relation but a relation indexed by a
number (called a \emph{step}) and a type, $(R_\tau^i)_{i, \tau}$. The
meaning of a relation at step $i$ and type $\tau$ is that two programs
are related if they are indistinguishable at type $\tau$ for $i$
steps. Two programs are related for $i$ steps intuitively if it will
take at least $i$ steps to tell them apart. For instance, if a program
$e_1$ runs to $\tt$ in 10 steps while $e_2$ runs to false in 20, it
takes 20 steps to distinguish them since we must wait for $e_2$ to
finish running before they can be compared. This idea of steps has
proven to be incredibly robust and easily extend to many different
settings~\citep{TODO-SUCCESSES-OF-STEP-INDEXING}. There is a price to
be payed for this extra flexibility. Firstly, steps will now pervade
the definition of the logical relation and any proof now must contain
pointless bookkeeping activity in order to keep track of
them. Secondly, the evaluation behavior of the programs under
consideration has begun to matter again. The goal of contextual
equivalence was to erase it from consideration but now there are
instances where in order to use a logical relation a program must be
modified with spurious NO-OP statements in order to make it take extra
steps. These issues are even present when just using the logical
relation to validate contextual
equivalences~\citep{TODO-TRANSFINITE}. In practice, this has meant
that while step-indexing is incredibly widely used it is almost
universally disliked\footnote{Citation: I walked around at POPL and
  the 5 people I asked seemed like REALLY bummed about it.}.

The structure of this thesis as follows. In Section~\ref{sec:language}
an ML-like language is described that will serve as target for the
logical relation. In Section~\ref{sec:step-indexing} a step-indexed
logical relation for this language is constructed and discussed,
especially to highlight its short-comings. In
Section~\ref{sec:domains} the first failed step-index-free logical
relation is described, the most naive approach centered around using
domain theory. In Section~\ref{sec:handedness} a variety a number of
step-index-free logical relations are discussed for languages powerful
enough to encode general references, largely built around (guarded)
recursive kinds.

%%% Local Variables:
%%% mode: latex
%%% TeX-master: "../main"
%%% End:

\section{An ML-like Language with General References}

\section{A Step-Indexed Logical Relation}

Before diving into the various approaches for constructing a logical
relation without step-indexing, it is well worth the time to see how
a logical relation can be done with it. The purpose of this section is
to sketch the complication intrinsic to any logical relation and show
how step-indexing obliterates them, though at a high cost.

Our logical to begin with a mapping from types to semantic types
(merely sets of terms). In order to handle impredicative polymorphism
Girard's method~\citep{Girard:71,Girard:72}, see \citet{TODO-PFPL} for
a comprehensive explanation of the technique. This means that our
logical relation is of the form
\[
  \den{-}_{-} : \types \to \typesEnv \to \pow{\term \times \term}
\]

The central challenge is of course the meaning of $\den{\cmd{\tau}}$:
the action of the logical relation at commands. At an intuitive level,
for two commands are rather like (partial) functions: they map heaps to heaps
and a return value. Drawing inspiration from how logical relations for
functions are defined, we might write the following for the definition
the logical relation.
\begin{align*}
  \den{\cmd{\tau}}_\eta& \triangleq \{(e_1, e_2) \mid\\
  &\exists m_1, m_2.\ \steps{e_i}{\cmd{m_i}} \land{}\\
  &\forall h_1 \sim h_2.
  \ (m_1, h_1) \Downarrow \iff (m_2, h_2) \Downarrow \land{}\\
  &\quad \forall v_1, h_1', v_2, h_2'.
  \ (\stepsM{m_1}{h_1}{\ret{v_1}}{h_1'} \land \stepsM{m_2}{h_2}{\ret{v_2}}{h_2'})\\
  &\qquad \implies (h_1' \sim h_2' \land (v_1, v_2) \in \den{\tau}_\eta)
\end{align*}
Here left undefined is the definition of $\sim$ between two
heaps. This is in fact a major issue because there appears to be no
good way to identify when two heaps ought to be equal. The first issue
here is that semantic equality of terms (be it contextual or logical)
is type-indexed. This means that in order to compare heaps pointwise
for equality (a reasonable though still wrong idea) requires that we
at least know the types of the entries. Furthermore, we shouldn't
compare these heaps for equality at all locations necessarily, two
heaps should only need to agree on the cells that the programs are
going to use. This is a significant concept if we want to prove
programs to be equivalent which do not use the heap identically. For
instance, consider the two programs:
\[
  \dcl{\alpha}{1}{\ret{\cmd{\get{\alpha}}}} \qquad\qquad
  \ret{\cmd{\ret{1}}}
\]
These are contextually equivalent (the assignable of the first program
is hidden from external manipulation) and yet they allocate in
different ways. So $\sim$ must not be \emph{merely} pointwise equality
in the most general case. Additionally, proving that these two
programs are equal requires showing that $h_1 ~ h_2$ if and only if
$h_1(\alpha) = 1$. That is, this program doesn't merely require that
heap cells contain values of some syntactic type, but they may need to
belong to an arbitrary semantic type. In order to reconcile these
constraints, one thing is clear: the logical relation must somehow
vary depending on the state that the heap is supposed to be in. It is
simply not the case that programs that are equivalent in a heap where
no cells are required to exist if and only if they're equivalent in a
heap where one cell is required to exist.

%%% Local Variables:
%%% mode: latex
%%% TeX-master: "../main"
%%% End:

\section{Tying the Knot Using Domains}\label{sec:domains}

This section describes the first and most obvious approach to building
a step-index-free logical relation for state. It would seem that the
real crux of the issue was this Kripke world. Nothing past the
construction of the Kripke world in Section~\ref{sec:steps}
was particularly challenging, especially when a logical approach is
taken. So a natural approach is to find a different setting in which
to solve the following equation
\[
  \worlds \cong \assignables \pto (\worlds \mto \pow{\term \times \term})
\]
Traditionally, domains have been used for exactly this purpose. The
impetus of domain theory as an investigation separate from order
theory in general was the observation by Dana Scott that domains
possessed solutions to equations that sets did not. For instance, a
model of the lambda calculus may be constructed by solving the
equation
\[
  D \cong D \to D
\]
This equation is not possible to solve in sets without trivializing
$D$. In fact, Theorem~\ref{thm:steps:fixed-points} uses a construction
derived originally from domain theory itself.

\subsection{Domain Theory, Briefly}

A domain is a particular sort of partial order which comes equipped
with enough structure to accurately describe a notion of continuous
functions upon it.
\begin{defn}\label{def:domains:directed-set}
  In a partial order $P$, a directed set $D \supseteq P$ is a set so
  that $D \neq \emptyset$ and if $a, b \in D$ then there is a $c \in
  D$ so that $a, b \le c$.
\end{defn}
\begin{defn}\label{def:domains:domain}
  A domain\footnote{In this text domains are directed complete partial
    orders as opposed to any of the myriad varierties of domains one
    might have chosen.} is partial order possessing joins (least upper
  bounds) of all directed subsets. Given a directed subset $D$ we
  write this join as $\join D$.
\end{defn}
Intuitively, elements of a domain represent a (partial) piece of
knowledge. Moving to a larger element represents moving to a piece of
knowledge that extends what was already known. The ability to
construct joins of directed sets fits into this intuition as the
ability to take a great many fragments of compatible knowledge and
glue them together.

This intuition primarily comes from the example domain of a partial
function. The function is not everywhere defined so it is not a
complete piece of knowledge, but given a collection of partial
functions which agree on the intersection, they can be glued
together. More precisely, fix our underlying set of $S \pto D$ to be
partial functions from some set $S$ to a domain $D$. The ordering on
these functions is given by the following:
\[
  f_1 \reach f_2 \defs \forall s \in S.\ f_1(s) \Downarrow \implies f_1(s) = f_2(s)
\]
Then, joins are constructed on directed sets as follows:
\[
  (\join_i f_i)(s) = \join \{d \mid \exists i.\ f_i(s) = d\}
\]
Showing that this is a partial function relies crucially on the
directedness of $(f_i)_i$.

The extra structure on domains gives rise to a new class of maps that
preserve this structure. These maps are called (Scott) continuous in
that they preserve joins.
\begin{defn}[Scott continuity]
  A map between domains is (Scott) continuous if it preserves the
  joins of directed sets. That is, if $f : D \to E$ and
  $I \subseteq D$ is directed, then
  $f(\join I) = \join \{f(i) \mid i \in I\}$.
\end{defn}
Calling this definition continuity can be justified by considering the
Scott topology on domains, in which continuous functions are precisely
those which preserve upper bounds. For a complete account of the
theory of domains, see for instance \citet{Girez:03,Abramsky:94}.
\begin{thm}
  Domains form an a category, $\DCPO$, where morphisms are continuous
  functions. This category is bicartesian closed, in particular the
  set of continuous maps between two domains is a domain.
\end{thm}
The interesting portion of domain theory for these purposes is the
ability to solve domain equations. Specifically, locally continuous
functors of domains again have solutions.
\begin{defn}
  A functor, $F : \DCPO \to \DCPO$ is locally continuous if its action
  on morphisms is a continuous function. That is, there is a
  continuous map of domains $B^A \to F(B)^{F(A)}$. This definition
  generalizes readily to a mixed variance n-ary functor.
\end{defn}
\begin{thm}\label{thm:domains:fixed-points}
  Any mixed variance locally continuous functor
  $F : \op{\DCPO} \times \DCPO \to \DCPO$ has an invariant object $I$
  so that $F(I, I) \cong I$.
\end{thm}

\subsection{Domains for State}

The domains involved in expression the desired equation are quite
straightforward. Power sets are always a complete lattice and
therefore a domain. Scott continuous functions are always monotone so
instead of considering monotone functions it's more natural to pick
Scott continuous functions to get a domain. Finally, partial maps form
a domain as shown above with the small complication that they cannot
be limited to finite partial maps.

This gives us a proper functor $F : \op{\DCPO} \to \DCPO$ defined by
\begin{align*}
  D \mapsto& \assignables \pto (D \mto \pow{\term \times \term})\\
  f \mapsto& \lambda x.\ \lambda \alpha.\ \lambda d.\ x(\alpha)(f(d))
\end{align*}
There is only one exceptional feature of all of this, the ordering on
the partial map aspect of this domain is nonstandard. A Kripke world
is supposed to map an assignable to a semantic type. The ordering on
these worlds should let us add new mappings, but old mappings should
stay the same. They must in fact, if a location is allocated so that
it can be either $\littrue$ or $\litfalse$, it would be a real
problem if suddenly it could also be a natural number. This ordering,
explicitly, is then defined by the following formula.
\[
  w_1 \reach w_2 \iff \forall \alpha \in \dom(w_1).\ w_1(\alpha) \simeq w_2(\alpha)
\]
This ordering, however natural, means that $F$ is not locally
continuous and therefore there is no way to construct a solution to
it.

Supposing for a moment that a solution to $F$ did exist. There would
still seemingly be a problem because one of these worlds could have
full support, making modeling allocation impossible. This can be
avoided by defining the logical relation on compact
elements~\citep{Girez:03} of the world domain, precisely those worlds
enjoying only finite support.

By defining the logical relation on these worlds, it is trivial to
extend it to an arbitrary world by taking $\den{\tau}(w)$ on such a
world to be the join of the results of $\den{\tau}$ on all the compact
elements below $w$. This is automatically continuous and recovers
exactly $\den{\tau}$ on compact elements which are, after all, the
worlds of interest.

%%% Local Variables:
%%% mode: latex
%%% TeX-master: "../main"
%%% End:

\section{Understanding State through (Syntactic) Minimal Invariance}\label{sec:smi}

Now that domains are seen to be unhelpful, there is some question
about where to turn next. There are not many well-studied
algebraically complete categories~\citep{Freyd:70} which support
interesting denotational semantics. The semantic structures for
step-indexing that support solving recursive domain equation require
more than continuity or approximation, they require a notion of
closeness for two semantic types. In $\presheaves{\omega}$ for
instance, there was a natural notion of when two points
$p, q : 1 \to P$ were close by checking at what stage they became
equal, eg when $p_n = q_n : \{\star\} \to P$.

Crucially, in order for this to make sense for semantic types every
equality was judged relative to a step. Two semantic types are equal
at $n$ if they agree on equalities that hold at stage $n$ or earlier,
eg, equalities which hold if the two sides are only allowed to run for
$n$ steps. Without step-indexing what could this be replaced by?

The answer is suggested by an idea in domain theory to resolve a
seemingly unrelated problem. The observant reader may have noticed
that there is a crucial strengthening that
Theorem~\ref{thm:steps:fixed-points} provides over
Theorem~\ref{thm:domains:fixed-points}, the former provides unique
solutions. In fact, there may be many nonisomorphic domains which
satisfy a given equation. This state of affairs can prove troublesome
for semantics in domains because it means that not all the properties
of the domain are determined by the equation. There may be ``exotic''
elements which are not required to satisfy the domain equation but are
present none the less which impedes reasoning about recursive domain
equations by induction or similar. \citet{Pitts:96} proposes a
solution to this by imposing an additional requirement on the solution
to a domain equation: minimal invariance.

Minimal invariance, precisely, is the statement that given a $D$ so
that $D \cong F(D, D)$ for a particular family of functions $\pi_i : D
\to D$ the following equation holds.
\[
  \join_i \pi_i = 1
\]
The function $\pi_i$ is determined by induction on $i$.
\begin{align*}
  \pi_i &: D \to D\\
  \pi_0 &\defs \bot\\
  \pi_{i + 1} &\defs F(\pi_i, \pi_i)
\end{align*}
The intuition behind this definition is that if $\pi_i$ converges to
$1$ then every element in $D$ must have arisen as the limit of
elements computed by the inclusion $F^n(\bot, \bot) \to D$ which can
be viewed as semantic requirement that elements of $D$ are
finite\footnote{Or rather, determined a the limits of ``finite''
  elements. This is not to be confused here with \emph{compact} elements and
  algebraic domains which share a similar intuition but require
  technically different properties since there is no reason to assume
  that $F$ preserves algebraicness}. It is not difficult to prove that
if $D \cong F(D)$ and $E \cong F(E)$ with $D$ and $E$ both being
minimally invariant then $E \cong D$ and the standard $D_\infty$
construction produces a minimally invariant domain. Therefore, domain
equations have a unique minimally invariant solution and these satisfy
a form of induction through minimal invariance.

An interesting observation made originally by
\citet{Birkedal:99} and extend by
\citet{Crary:07} is that for the class of domains arising
naturally in semantics (so called ``universal domains'') the
projection functions, $\pi_i$, are computable
syntactically. Additionally, since contextual equivalence can be
generalized to contextual approximation in a natural way, minimal
invariance can be stated as a syntactic property of the programming
language itself. In this setting, projections are naturally
type-indexed and this is denoted $\pi_\tau^i : \fn{\tau}{\tau}$. The
definition of $\pi_\tau^i$ behaves almost as an $\eta$-expansion where
after $i$ $\eta$ expansion the function simply diverges.

It is well worth it to take a moment to be precise about the
definitions involved.
\begin{defn}\label{def:smi:contextual-approximation}
  Two terms $\hasESigJ{\Delta}{\Gamma}{\Sigma}{e_1, e_2}{\tau}$ are
  said to be contextually approximate at type $\tau$, written
  $\approxESigJ{\Delta}{\Gamma}{\Sigma}{e_1}{e_2}{\tau}$ if for all
  contexts
  $\hasCEJ{\Delta}{\Gamma}{\Sigma}{\tau}{C}{\cdot}{\cdot}{\Sigma'}{\cmd{\tau'}}$,
  $\steps{C[e_1]}{\cmd{m_1}}$ and $\steps{C[e_2]}{\cmd{m_2}}$ such
  that for any heap $h : \Sigma$ then
  $(m_1, h) \Downarrow \implies (m_2, h) \Downarrow$.
\end{defn}
We will adopt the traditional abuse of abandoning the context
annotations and just writing $e_1 \cless e_2$ when the contexts and
$\tau$ are obvious.
\begin{defn}\label{def:smi:contextual-limits}
  A family of terms $(e_i)_{i \in I}$ is said to have a limit $e$ if
  for all $i \in I$ $e_i \cless e$ and $e$ is the smallest element
  with this property.
\end{defn}
Since limits are only unique up to contextual equivalence, it is not
necessarily well defined to write $\join_i e_i$ as if it were a term
rather than an equivalence class of terms. We will instead often make
use of the notation $\join_i e_i = e$ to signify that $e$ belongs to
this equivalence class. The corresponding syntactic version of minimal
invariance can now be precisely stated.
\begin{thm}\label{thm:smi:smi}
  For all $\hasTJ{}{\Gamma}{\tau}{\tp}$ and $e : \tau$,
  $\join_i \ap{\pi_\tau^i}{e} = e$
\end{thm}
The idea behind using syntactic minimal invariance is that instead of
defining a single logical relation $\den{-}_{-}$ a pair of logical
relations is defined instead: $\definitely{-}_{-}$ and
$\possibly{-}_{-}$. The definition of these two logical relations are
mutually recursive and provide a decomposition of the logical relation
two monotone halves. The idea is that one should think of
$\definitely{\tau}_{\eta}$ containing equalities that are certainly
true but it does not contain all of them. On the other hand
$\possibly{\tau}_\eta$ contains all equalities that are true as well
as a few that might not be. This suggests a natural relationship
between these two sets where
$\definitely{\tau}_\eta \subseteq \possibly{\tau}_\eta$. The crucial
move that syntactic minimal invariance provides is to show the
following.
\[
  \forall i.\ (e_1, e_2) \in \definitely{\tau}_\eta \implies
  (\ap{\pi_\tau^i}{e_1}, \ap{\pi_\tau^i}{e_2}) \in \possibly{\tau}_\eta
\]
Then, by showing the comparatively easy lemma that
$\possibly{\tau}_\eta$ is always closed under limits when $\eta$ is it
immediately follows that
$\definitely{\tau}_\eta = \possibly{\tau}_\eta$. This maneuver is
useful because these two definitions are naturally monotone therefore
its easy to express a natural definition of recursive types, as is
done in \citet{Crary:07} for instance. The only complication
is that environments must generalize from relations to birelations, a
map to a pair of relations rather than a map to a single
relation. Define $\op{\eta}$ to be the environment mapping $\alpha$ to
the relations $\eta(\alpha)$ with the components swapped. The crucial
insight is that when the variance in the definition of the logical
relation swaps, we switch from $\definitely{-}$ to $\possibly{-}$ and
vice-versa, for instance:
\begin{align*}
  \definitely{\fn{\tau_1}{\tau_2}}_\eta &=
  \{(e_1, e_2) \mid \forall (a_1, a_2) \in \possibly{\tau_1}_\eta.
  \ (\ap{e_1}{a_1}, \ap{e_2}{a_2}) \in \definitely{\tau_2}_\eta\}
\end{align*}
This approach seems to naturally fit recursive types and one might
wonder then if it can be scaled to support state just as step-indexing
does\footnote{If this sentence leave the reader in suspense I
  encourage them to glance at the title}.

The key insight is that while the distinction of
$\definitely{-}$/$\possibly{-}$ is not useful, projection functions
give us a notion akin to step-indexing. We can say that two semantic
types are equal at stage $i$ if they are equal upon truncating all the
programs in the equalities by $\pi^i$. It remains then to define
$\pi_\tau^i$ for our language. In order to do this we need a
divergence at the level of expressions, this can be easily
accomplished by adding fixed points or just adding a formal term
$\bot$ so that $\step{\bot}{\bot}$.
\begin{align*}
  \pi_\tau^0 &= \lam{x}{\tau}{\bot}\\
  \pi_{\fn{\tau_1}{\tau_2}}^{i + 1} &=
  \lam{f}{\fn{\tau_1}{\tau_2}}{\lam{x}{\tau_1}{\ap{\pi_{\tau_2}^i}{(\ap{f}{(\ap{\pi_{\tau_1}^i}{x})})}}}\\
  \pi_{\allNoKind{\alpha}{\tau}}^{i + 1} &=
  \lam{f}{\allNoKind{\alpha}{\tau}}{\LamNoKind{\alpha}{\ap{\pi_\tau^i}{(\Ap{f}{\alpha})}}}\\
  \pi_{\cmd{\tau}}^{i + 1} &=
  \lam{x}{\cmd{\tau}}{\cmd{\bnd{x'}{x}{\ret{\ap{\pi_\tau^i}{x'}}}}}
\end{align*}
This approach, while effective for recursive types, is not quite what
is needed to handle state. After all, the issue in
Section~\ref{sec:steps} was not the definition of
$den{\cmd{\tau}}_\eta(w)$, it was the construction of the set in which
$w$ lives. On the other hand, the projections themselves are
interesting, they provide a way to construct an indexing of semantic
types without actually counting steps. This approach was explored in
\citet{Birkedal:domain:10} and successfully used to construct a
logical relation for a similar language. This work, however, made use
of the normal domain theoretic projections rather than their syntactic
counterparts. The question of producing a logical relation for our
language is then reduced to the question of producing a syntactic
account of \citet{Birkedal:domain:10}.

In this account, monadic computations are represented by
store-passing.
\begin{align*}
  \states &= \assignables \pto D\\
  D &\cong (D \to D) \oplus (D \times D) \oplus ... \oplus \states \to (\states \times D)
\end{align*}
Projections, $\pi^i$, work by casing on which branch of the domain
equation the argument is in and proceed accordingly. This is analogous
to how with syntactic minimal invariance the projection function is
split according the type of the argument. Most of the clauses are
standard with the interesting one being the last one for
commands. Denote a value in of the domain in the last branch as
$\cmd{f}$ where $f : \states \to (\states \times D)$. Then
$\pi^{i + 1}$ on this satisfies the following formula.
\begin{align*}
  \pi^{i + 1}(\cmd{f}) &=
  \cmd{\lambda s. (\trunc^i(\pi_1 f(s)), \pi^i(\pi_2 f(s)))}
  \trunc^i(s) = \pi^i \circ s
\end{align*}
In words: truncation on a command runs the command a point-wise
truncated heap and then truncates the return value as well as the
resulting heap.

In order to implement this syntactically, there's an issue. How can it
be possible to replicate in a setting where accessing the whole heap
is forbidden. In fact, this cannot be implemented as is. This presents
a serious issue in scaling up syntactic minimal invariance and there
is no clear solution.

If, for instance, a new command $\trunc_i \div \mathrm{unit}$ is
added to the language it is possible to express the projections. With
the other definition left unchanged the new case,
$\pi_{\cmd{\tau}}^{i + 1}$, is as follows.
\begin{align*}
  \lam{c}{\cmd{\tau}}
  {\cmd{& \bnd{\_}{\cmd{\trunc_i}}{\\
        & \bnd{x}{c}{\\
        & \bnd{\_}{\cmd{\trunc_i}}{\\
        & \ret{\ap{\pi_\tau^i}{x}}}}}}}
\end{align*}
If this is added then Theorem~\ref{thm:smi:smi} holds when we add the
following rule for evaluating $\trunc^i$.
\begin{mathpar}
  \inferrule{
  }{\stepsM{\trunc^0}{h}{\trunc^0}{h}}\and
  \inferrule{
    h(\alpha) = \ap{\pi_\tau^i}{\alpha}
  }{\stepsM{\trunc^{i + 1}}{h}{\ret{()}}{h'}}
\end{mathpar}
It is worth noting the strangeness of the last rule which is morally
``truncating the full heap''. This strangeness is twofold. It bakes in
the $\pi_\tau$ functions into the very operational semantics despite
them just being chunks of user-defined code. It also requires that the
heap be in some way typed so that when projecting at some particular
cell it is possible to determine the type to truncate at.

Beyond the strangeness of the rule, however, it works. A logical
relation can be defined using Kripke worlds which index semantic types
by truncation. This is problematic, however, because it changes the
notion of contextual equivalence. By adding new constructs to the
language new contexts are added and this impacts contextual
equivalence. In this extended language, programs that ought to be
equal are no longer. For instance, the following program should be
equivalent to $\cmd{\cmd{\ret{\lam{x}{\tau}{x}}}}$ but is not.
\begin{align*}
  \cmd{& \dcl{\alpha}{\lam{x}{\tau}{x}}{\\
       & \ret{\cmd{\get{\alpha}}}}}
\end{align*}
The issue is that another part of the program could truncate the heap,
replacing $\lam{x}{\tau}{x}$ with a truncation of itself. This
truncation would diverge strictly more than the original and thus the
second piece of code only contextually approximates the first. This
weakening is discussed more in subsequent work to
\citet{Birkedal:domain:10}, for instance in
\citet{Birkedal:adts:12}. In our case, however, this restriction is
damaging. It makes it impossible to prove any information-hiding
results using our logical relation and this makes it difficult to
prove that certain effects are benevolent\footnote{So actually this is
  a key motivation that I forgot about until now. TODO mention it
  earlier.}.

This puts a stop to naively extending syntactic minimal invariance to
our language with state. An idea with briefly considering is to build
a logical relation on a language with $\trunc_i$ and then attempt to
excise the terms that should not have been in there in the first place
after the fact. This approach has proven to be too complicated to
feasibly manage. Instead, a better approach seems to be to isolate a
more type-theoretic characterization of the difficulty of a logical
relation for state.

%%% Local Variables:
%%% mode: latex
%%% TeX-master: "../main"
%%% End:

\subsection{A Logical Relation in Ultrametric Spaces}\label{sec:guarded}

%%% Local Variables:
%%% mode: latex
%%% TeX-master: "../main"
%%% End:

\section{Conclusions}

This thesis has illustrated the difficulty of constructing a logical
relation for a language with higher-order state. The current set of
techniques which work are complex and despite this they only provide
sound but not complete reasoning principles. Even with this complexity
logical relations are among the best known techniques for handling
effects. They strike a combination between using simple mathematics
and providing useful reasoning principles.

The difficulty and complexity of the attempts to remove step-indexing
from the logical relation raises a question: is it worth it? The
motivations for step-index-free logical relations where basically the
following:
\begin{itemize}
\item Logical relations for step-indexing where complicated and
  painful to use and construct correctly.
\item Step-indexing is ad-hoc and difficult to justify.
\item Step-indexing is ill-suited for capturing properties beyond
  safety.
\end{itemize}
The first two issues are addressed in the same way: a generalization
in the way we view logical relations. If logical relations are just
structure preserving relations, there's no reason to limit ourselves
to traditional set-theoretic relations. Already with Kripke logical
relations there is the same indexing structure observed in Kripke
logical relations without step-indexing. Rather than viewing relations
as holding or not holding it is quite natural to replace them with a
more internsional, local notion of truth as is done in both
step-indexing and Kripke logical relations. If we make use of more
sophisticated tools for working with these relations, either by
working internally to a type theory, a higher-order logic, or a topos
much of the day-to-day issues of step-indexing can be resolved.

The last issue is the real problem. The nature of step-indexing is
inherently limiting because it is only is suited for properties that
are sound with respect to finite prefixes. Consider for instance, the
set of programs producing a stream of natural numbers. The program
outputting $1$, $2$, $3$, .. for any finite step $n$ produces a
maximum but it is not the case that it ever actually outputs a number
globally! This is a classic issue in mathematics: global truth is not
truth in every localization and step-indexing only allows us to
discuss local truth.

In order to fix this, one can either remove the indexing structure and
never work with local truth or add enough locales so that local truth
is strong enough. For instance, if we work with $\omega^2$ in the
example above the problematic program is ruled out: it does not
satisfy the program at $(\omega, 0)$. A fuller theory of logical
relation in this setting is given by \citet{Svendsen:16}.

Further investigation is needed in both directions. There are many
unstudied generalizations of relations beyond merely considering sets
of truth values. On the other hand, there is an undeniable appeal to
just being able to work with sets and to find some new clever
technique to avoid the indexing all together. At the moment though,
there seems to be far more unexplored and accessible space in the
first direction than the second; constructing a step-index-free
logical relation seems to require some new approach beyond
applications of classic tools as was done in this thesis.

%%% Local Variables:
%%% mode: latex
%%% TeX-master: "../main"
%%% End:

\section*{Acknowledgments}

I would like to thank the huge number of people who have been
instrumental in writing this thesis and the research of the last four
years that went into it. I am greatly indebted to the POP group at CMU
for a unique environment to study. A special thanks to Carlo Angiuli,
Evan Cavallo, Adrien Guatto, Anders M{\"o}rtberg, Jonathan Sterling,
and Joseph Tassarotti for countless hours of discussions on
programming languages. In particular, I must thank Jonathan Sterling and
Carlo Angiuli for valuable feedback on a draft of this thesis. Thanks
to Lars Birkedal and Robert Harper for feedback and advice on this
work. Finally, I must thank Karl Crary for advice, encouragement, and
much wisdom into how to do research in type theory.

%%% Local Variables:
%%% mode: latex
%%% TeX-master: "../main"
%%% End:


\bibliographystyle{plainnat}
\bibliography{citations}
\end{document}
