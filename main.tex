\documentclass{article}
\usepackage[hyperfootnotes=false]{hyperref}
\usepackage[sort&compress,square,comma,numbers]{natbib}
\usepackage{amsthm}
\usepackage{amsmath}
\usepackage{amssymb}
\usepackage{mathpartir}
\usepackage{stmaryrd}
\usepackage{xifthen}
\usepackage{tikz-cd}
\usepackage{xcolor}

% Formatting tools
\newcommand{\declareJudgement}[1]{\framebox{$\displaystyle{}{#1}$}}
\newcommand{\different}[1]{{\color{red} #1}}
% Theorems
\newtheorem{thm}{Theorem}[section]
\newtheorem{cor}[thm]{Corollary}
\newtheorem{lem}[thm]{Lemma}
\newtheorem{remark}[thm]{Remark}
\newtheorem{example}[thm]{Example}
\newtheorem{defn}[thm]{Definition}

% Categories stuff
\newcommand{\Ccat}{\ensuremath{\mathbb{C}}}
\newcommand{\Dcat}{\ensuremath{\mathbb{D}}}
\newcommand{\op}[1]{\ensuremath{#1^{\mathsf{op}}}}
\newcommand{\SET}{\ensuremath{\mathbf{Set}}}
\newcommand{\DCPO}{\ensuremath{\mathbf{Dcpo}}}
\newcommand{\presheaves}[1]{\ensuremath{\widehat{#1}}}
\newcommand{\transpose}[1]{\ensuremath{\widehat{#1}}}
\newcommand{\pair}[2]{\ensuremath{\left\langle #1, #2 \right\rangle}}
\newcommand{\univ}{\ensuremath{\mathcal{U}}}
\newcommand{\mono}{\ensuremath{\rightarrowtail}}
\newcommand{\yoneda}{\ensuremath{\mathsf{y}}}

% Math stuffs
\newcommand{\upred}[1]{\ensuremath{\mathsf{UPred}(#1)}}
\newcommand{\reach}{\ensuremath{\mathrel{\sqsubseteq}}}
\newcommand{\breach}{\ensuremath{\mathrel{\sqsupseteq}}}
\newcommand{\mto}{\ensuremath{\xrightarrow{\mathsf{mon}}}}
\newcommand{\pto}{\ensuremath{\rightharpoonup}}
\newcommand{\pow}[1]{\ensuremath{\mathcal{P}(#1)}}
\newcommand{\powfin}[1]{\ensuremath{\mathcal{P_{\mathrm{fin}}}(#1)}}
\newcommand{\card}[1]{\ensuremath{\left\vert #1 \right\vert}}
\newcommand{\real}{\ensuremath{\mathbb{R}}}
\newcommand{\nat}{\ensuremath{\mathbb{N}}}
\newcommand{\cbult}{\ensuremath{\mathrm{CBUlt}}}
\newcommand{\cle}{\ensuremath{\lesssim}}
\newcommand{\ceq}{\ensuremath{\cong}}
\newcommand{\relR}{\ensuremath{\mathrel{\mathcal{R}}}}
\newcommand{\relS}{\ensuremath{\mathrel{\mathcal{S}}}}
\newcommand{\AND}{\ensuremath{\mathrel{\wedge}}}
\newcommand{\OR}{\ensuremath{\mathrel{\vee}}}
\newcommand{\den}[1]{\ensuremath{\llbracket #1 \rrbracket}}
\newcommand{\definitely}[1]{\ensuremath{\lceil #1 \rceil}}
\newcommand{\possibly}[1]{\ensuremath{\lfloor #1 \rfloor}}
\newcommand{\defs}{\ensuremath{\mathrel{\triangleq}}}
\newcommand{\bifix}{\ensuremath{\mathsf{bifix}}}
\newcommand{\disjoint}{\ensuremath{\mathop{\#}}}
\newcommand{\hole}{\ensuremath{\square}}
\newcommand{\sincl}[1]{\ensuremath{\mathsf{succ}(#1)}}
\newcommand{\join}{\ensuremath{\bigvee}}
\newcommand{\cless}{\ensuremath{\lessapprox}}
\newcommand{\aless}{\ensuremath{\lesssim}}

\DeclareMathOperator{\dom}{Dom}

% Sets
\newcommand{\states}{\ensuremath{\mathrm{State}}}
\newcommand{\worlds}{\ensuremath{\mathrm{World}}}
\newcommand{\assignables}{\ensuremath{\mathrm{Assignable}}}
\newcommand{\nrel}[1]{\ensuremath{\mathbb{R}_{#1}}}
\newcommand{\semtypes}{\ensuremath{\mathbb{T}}}
\newcommand{\types}{\ensuremath{\mathrm{Type}}}
\newcommand{\typesEnv}{\ensuremath{\mathrm{TypeEnv}}}
\newcommand{\term}{\ensuremath{\mathrm{Term}}}
\newcommand{\urel}{\ensuremath{\mathrm{URel}}}

% judgments
\newcommand{\guardJ}[2]{\ensuremath{\text{$#1.\,#2$\ \textsf{guarded}}}}
\newcommand{\hasE}[2]{\ensuremath{#1 \mathrel{:} #2}}
\newcommand{\hasM}[2]{\ensuremath{#1 \mathrel{\div} #2}}
\newcommand{\hasKJ}[2]{\ensuremath{#1 \vdash \hasE{#2}{\mathrm{\kind}}}}
\newcommand{\hasTJ}[4]{\ifthenelse{\isempty{#1}}%
  {\ensuremath{#2 \vdash \hasE{#3}{#4}}}%
  {\ensuremath{#1;#2 \vdash \hasE{#3}{#4}}}}
\newcommand{\hasEJ}[5]{\ifthenelse{\isempty{#1}}%
  {\ensuremath{#2; #3 \vdash \hasE{#4}{#5}}}%
  {\ensuremath{#1; #2; #3 \vdash \hasE{#4}{#5}}}}
\newcommand{\hasESigJ}[5]{\ensuremath{#1; #2 \vdash_{#3} \hasE{#4}{#5}}}
\newcommand{\hasMJ}[5]{\ensuremath{#1; #2 \vdash_{#3} \hasM{#4}{#5}}}
\newcommand{\hasCEJ}[9]{\ensuremath{#5 : (#1; #2 \vdash_{#3} #4) \rightsquigarrow (#6; #7 \vdash_{#8} #9)}}

\newcommand{\subKJ}[3]{\ifthenelse{\isempty{#1}}%
{\ensuremath{{#2} \leq {#3}}}%
{\ensuremath{{#1} \vdash {#2} \leq {#3}}}}

\newcommand{\equivESigJ}[6]{\ensuremath{#1; #2 \vdash_{#3} \hasE{#4 \cong #5}{#6}}}
\newcommand{\approxESigJ}[6]{\ensuremath{#1; #2 \vdash_{#3} \hasE{#4 \aless #5}{#6}}}
\newcommand{\equivMJ}[6]{\ensuremath{#1; #2 \vdash_{#3} \hasM{#4 \cong #5}{#6}}}

\newcommand{\step}[2]{\ensuremath{#1 \mapsto #2}}
\newcommand{\steps}[2]{\ensuremath{#1 \mapsto^* #2}}
\newcommand{\stepM}[4]{\ensuremath{(#1, #2) \mapsto (#3, #4)}}
\newcommand{\stepsM}[5][*]{\ensuremath{(#2, #3) \mapsto^{#1} (#4, #5)}}

\newcommand{\dec}[4][i]{\ensuremath{#2 \rhd^{#1} #3 : #4}}
\newcommand{\valueJ}[1]{\ensuremath{#1\ \mathsf{value}}}
\newcommand{\finalJ}[2]{\ensuremath{(#1, #2)\ \mathsf{final}}}

% language
\newcommand{\kind}{\ensuremath{\mathsf{kind}}}
\newcommand{\later}{\ensuremath{{\blacktriangleright}}}
\newcommand{\ilater}{\ensuremath{{\triangleright}}}
\newcommand{\rec}[2]{\ensuremath{\mu #1.\, #2}}
\newcommand{\fn}[2]{\ensuremath{#1 \to #2}}
\newcommand{\tp}{\ensuremath{\mathsf{T}}}
\newcommand{\unit}{\ensuremath{\mathsf{unit}}}
\newcommand{\cmd}[1]{\ensuremath{\mathsf{cmd}(#1)}}

\newcommand{\ap}[2]{\ensuremath{#1\ #2}}
\newcommand{\lam}[3]{\ensuremath{\lambda #1 {:} #2.\, #3}}
\newcommand{\into}[1]{\ifthenelse{\isempty{#1}}%
  {\ensuremath{\mathsf{in}}}%
  {\ensuremath{\mathsf{in}\,#1}}}
\newcommand{\out}[1]{\ifthenelse{\isempty{#1}}%
  {\ensuremath{\mathsf{out}}}%
  {\ensuremath{\mathsf{out}\,#1}}}
\newcommand{\delay}{\mathsf{next}}
\newcommand{\fix}{\mathsf{fix}}
\newcommand{\letdelay}[3]{\ensuremath{\mathsf{let\ next\ } #1 = #2 \mathsf{\ in\ } #3}}
\newcommand{\all}[3]{\ensuremath{\forall #1 {:} #2.\, #3}}
\newcommand{\allNoKind}[2]{\ensuremath{\forall #1.\, #2}}

\newcommand{\Ap}[2]{\ensuremath{#1[#2]}}
\newcommand{\Lam}[3]{\ensuremath{\Lambda #1 {:} #2.\, #3}}

\newcommand{\LamNoKind}[2]{\ensuremath{\Lambda #1.\, #2}}
\newcommand{\ret}[1]{\ensuremath{\mathsf{ret}(#1)}}
\newcommand{\get}[1]{\ensuremath{\mathsf{get}[#1]}}
\newcommand{\set}[2]{\ensuremath{\mathsf{set}[#1](#2)}}
\newcommand{\dcl}[3]{\ensuremath{\mathsf{dcl}\ #1 := #2\ \mathsf{in}\ #3}}
\newcommand{\bnd}[3]{\ensuremath{\mathsf{bnd}\ #1 \gets #2;\ #3}}

\newcommand{\trunc}{\ensuremath{\mathsf{trunc}}}

\newcommand{\littrue}{\ensuremath{\mathsf{true}}}
\newcommand{\litfalse}{\ensuremath{\mathsf{false}}}

\newcommand{\zap}{\ensuremath{\circledast}}


\title{The Next 700 Failed Step-Index-Free Logical Relations}
\author{Daniel Gratzer}
\date{\today}

\begin{document}
\begin{titlepage}
  \makeatletter
  \centering
  {\scshape\Large Senior Thesis\par}
  \vspace{1.5cm}
  {\huge\bfseries \@title\par}
  \vspace{2cm}
  {\Large\itshape \@author\par}
  \vfill
  supervised by\par
  Professor Karl \textsc{Crary}

  \vfill

  % Bottom of the page
  {\large \@date\par}
  \makeatother
\end{titlepage}

\begin{abstract}
  An important topic in programming languages is the study of
  program equivalence. This is done typically with the construction of
  a relational denotational model or a syntactic analogue, called a
  logical relation. Logical relations have proven to be an effective
  tool for analyzing programs and lending formal weight to ideas like
  data abstraction and information hiding.

  A central difficulty with logical relations is their fragility; it
  has proven to be a challenge to scale logical relations to more
  realistic languages. A common technique for accomplishing this is
  \emph{step-indexing}. Step-indexing may be an effective tool for
  defining logical relations but it results in often frustrating
  technical details and limitations. Replacing it with more
  traditional logical relations is desirable but so far has only been
  achieved for recursive types. In this work we consider extending
  traditional logical relations to higher-order references, a common
  feature in modern languages.

  The central challenge with constructing a logical relation for
  higher-order references is that it must be a Kripke logical
  relation. The construction of the Kripke worlds has been a
  persistent challenge because they do not exist as mere sets. The
  semantic types must be indexed by a Kripke world and the Kripke
  world must mention those semantic types, a recursive equation
  which has no solutions naively. This can be solved by using
  step-indexing to navigate the recursive equation but other
  techniques may be possible.

  Three methods are explored in this thesis: a domain theoretic
  approach, a naive application syntactic minimal invariance, and an
  application of syntactic minimal invariance to step-indexing. None
  are sufficient to solve the problem. This demonstrates the
  difficulty of expressing the recursive structure of higher-order
  references.
\end{abstract}

\newpage

In the study of programming languages a great number of important
questions hinge on the equality of programs.
\begin{itemize}
\item The verification of a compiler pass is nothing but a
  question of equality of the naive program and its optimized
  version.
\item An optimized data structure may be shown to be correct relative to a
  much simpler reference solution
\item Much of the research in dependent type theory centers around
  what terms should be judged equal.
\end{itemize}
Given the role that equality plays in programming languages it is
unsurprising that a great variety of mathematical tools have been
developed to study it.

Before anything can be said about tools for analyzing equality,
however, a precise characterization of equality must be given. There
are many reasonable notions of equality between programs and all of
them are suitable for different circumstances. For instance, one can
compare programs up to renaming of variables, $\alpha$-equivalence, in
compilers and other applications where it is important that no
information is lost. In type-checkers for dependently typed languages,
it is natural to be more flexible and regard programs as the same if
they share a normal form, some sort of $\beta\eta$-equivalence at
least. In this work, we are interested in studying program behavior
and verifying properties about it so an even more relax notion of
equality is called for: \emph{contextual equivalence}.

Contextual equivalence is formally defined later (see
Definition~\ref{def:language:cxt}) but informally contextual
equivalence of two programs, $e_1 \cong e_2$, expresses that $e_1$ and
$e_2$ are internally indistinguishable. This means that there is no
program containing $e_1$ so that if $e_1$ is replaced with $e_2$ then
the program returns a different result. This notion of equivalence is
appealing because there is no aspect of a program we care about other
than what it computes to. Contextual equivalence allows us to ignore
unimportant differences in the precise way that the answer of a
program is computed and focus instead on the answer itself. This means
that two implementations of a data structure, for instance, are
contextually equivalent even if one is far more complex and efficient
than the other.

For all the appeal of contextual equivalence, it is very difficult to
establish any instances of it. In order to show that
$e_1 \cong e_2$ it is necessary to quantify over all possible programs
using $e_1$ and $e_2$ and reason about their behavior. This includes
programs which do not really make use of $e_1$ or $e_2$ such as
$\ap{(\lam{x}{\fn{\unit}{\tau}}{1})}{(\lam{x}{\unit}{e_1})}$. This lack
of constraints on how $e_1$ or $e_2$ is used is precisely what makes
contextual equivalence so useful but it makes even the most basic
proofs involved affairs. In order to compensate for this, a variety of
tools have been specialized to simplify the process of establishing when
$e_1 \cong e_2$.

Broadly speaking, there are two classes of these tools. One may
consider denotational approaches, where in the equality of programs is
reduced to the question of the equality of normal mathematical
objects. This line of study begins with Dana Scott's investigations of
the lambda calculus~\citep{Scott:76}. Such tools have been
immensely effective when they may be applied but they are difficult to
use. Complex programming languages often make use of extremely
sophisticated mathematical objects and using the model requires
understanding them. This has meant that denotational semantics is
traditionally out of reach for the verification of programs by an
average programmer or anyone besides a domain expert.

On the other hand, there are syntactic approaches to equality. Some of
these date back to the original study of the lambda calculus and its
reduction properties. Syntactic tools tend to be simpler to use,
relying only one elementary mathematics and an understanding of the
syntax itself. For equality, the tool of choice when working
syntactically is a logical
relation~\citep{Tait:67,Girard:72,Pitts:98,Appel:01,Ahmed:04,Ahmed:06,Appel:07,Dreyer:09,Dreyer:10}.
Dating back to~\citet{Tait:67} logical relations generalize
homomorphisms and can be thought of as \emph{structure preserving
  relations}\footnote{This viewpoint is more liberating than the
  traditional explanation that they are a type-indexed inductively
  defined family of relations. Logical relations arise for general
  mathematical structure. A programming language is merely one example
  of such structure.}. Logical relations are used for a wide variety
of tasks but the most important one for us is as a method for
demonstrating contextual equivalences. The study of logical relations
has given rise to a powerful generalization of the theory of data
abstraction called parametricity~\citep{Reynolds:83}. Parametric
reasoning has been so important that there have even been attempts to
reconstruct it in a denotational
sense~\citep{Bainbridge:90,Abadi:90,Ma:91,Birkedal:05,Dunphy:04}. This
work is important for crystallizing the connections between
naturality and parametricity as well as cementing parametricity in the
broader tradition of logical relations. At the present time this
program is still incomplete however. A summary of the state of the art
in 2013 and the fundamental challenges may be found in
\citet{Hermida:14}.

To be precise, a logical relation for us is a family of binary
relations indexed by the types of the language,
$(R_\tau)_{\tau \in \types}$. It is constructed by induction over the
types indexing it so that $R_{\fn{\tau_1}{\tau_2}}$ is defined in
terms of $R_{\tau_1}$ and $R_{\tau_2}$. For each $\tau$, the following
property is expected to hold:
\[
  (e_1, e_2) \in R_\tau \implies
  \hasE{e_1, e_2}{\tau} \land e_1 \cong e_2
\]
This property expresses the soundness of the logical relation with
respect to contextual equivalence. The reverse property, completeness,
is desirable but often unachievable\footnote{Instead a variety of
  techniques for \emph{forcing} completeness to hold are
  employed. These amount to adding all the missing identifications to
  the logical relation. While theoretically desirable, this is useless
  for the problem of actually establishing an
  equivalence.}. Completeness is usually unnecessary for establishing
certain concrete equivalences so in this context it is less
important. Finally, an important property of logical relations which
is difficult to capture formally is that it is much easier to prove
that $(e_1, e_2) \in R_\tau$ than to directly show that
$e_1 \cong e_2$. Typically, $R_\tau$ is meant to capture the logical
action of $\tau$; it expresses the precise set of observations
possible to make of expressions of type $\tau$ without all the
duplication of contextual equivalence. For instance, in order to show
that $(e_1, e_2) \in R_{\fn{\tau_1}{\tau_2}}$ it suffices to show the
following.
\[
  \forall (a_1, a_2) \in R_{\tau_1}.
  \ (\ap{e_1}{a_1}, \ap{e_2}{a_2}) \in R_{\tau_2}
\]
Given the obvious appeal of having a logical relation defined for a
language, it would seem that the construction of a logical relation
characterizing equality is a natural first step in the study of a new
programming language. The central stumbling block to this goal is that
logical relations are difficult to extend to new programming languages
and especially to new types and computational effects.

To see where trouble might arise, consider a language with polymorphic
types: $\allNoKind{\alpha}{\tau}$. What should the definition of
$R_{\allNoKind{\alpha}{\tau}}$ be? Phrased different, when are two
polymorphic functions indistinguishable? One might expect a definition
like the following:
\[
  (e_1, e_2) \in R_{\allNoKind{\alpha}{\tau}} \defs
  \forall \tau'.\ (\Ap{e_1}{\tau'}, \Ap{e_2}{\tau'}) \in R_{[\tau'/\alpha]\tau}
\]
This does correctly characterize contextual equivalence but it is
ill-suited as a definition in a logical relation because it will not
be well-founded. A logical relation is constructed by induction on the
types and with impredicative polymorphism there is absolutely no
reason why $\tau'$ should be smaller than
$\allNoKind{\alpha}{\tau}$. This is not a minor issue, defining a
correct logical relation for a language with impredicative
polymorphism requires the novel method of
candidates~\citep{Girard:71,Girard:72}. This situation has been a
recurring issue: features in programming languages tend to have some
recursive structure which prevents them from easily fitting into the
inductive definition of a logical relation. For many features, clever
constructions have been found which circumvent well-foundedness
issues: for instance parametric
polymorphism~\citep{Girard:71,Girard:72}, first-order
state~\citep{Pitts:98}, simple exceptions, and recursive
types~\citep{Birkedal:99,Crary:07}. Despite this work logical
relations are still a long way away from being able to cope with a
full programming language. The state of the art for dealing with
features like higher-order references, nontrivial control passing, or
concurrency is to use step-indexing.

Step-indexing is a technique first proposed by \citet{Appel:01}. It is
based on a simple but ingenious idea: if you the logical relation is
not well-defined by induction on the type just add a number which
decreases and induct on that. This idea means that a logical relation
is no longer a type-indexed relation but a relation indexed by a
number (called a \emph{step}) and a type, $(R_\tau^i)_{i, \tau}$.

The meaning of two programs related at step $i$ and type $\tau$ they
are indistinguishable at type $\tau$ for $i$ steps. Two programs are
related for $i$ steps intuitively if it will take at least $i$ steps
in order to make an observation which distinguishes them. For
instance, if a program $e_1$ runs to $\tt$ in 10 steps while $e_2$
runs to false in 20, it takes 20 steps to distinguish them since we
must wait for $e_2$ to finish running before they can be compared.

This idea of steps has proven to be incredibly robust and easily
extend to many different
settings~\citep{Appel:01,Ahmed:04,Ahmed:06,Appel:07,Dreyer:09,Dreyer:10,Birkedal:steps:11,Turon:13,Svendsen:16}.
There is a price to be payed for this extra flexibility. Firstly,
steps will now pervade the definition of the logical relation and any
proof now must contain pointless bookkeeping activity in order to keep
track of them. Secondly, the evaluation behavior of the programs under
consideration has begun to matter again. The goal of contextual
equivalence was to erase it from consideration. Now there are
instances where in order to use a logical relation a program must be
modified with spurious NO-OP statements in order to make it take extra
steps, a clear violation of this principle. These issues are even
present when just using the logical relation to validate contextual
equivalences~\citep{Svendsen:16}. In practice, this has meant that
while step-indexing is incredibly widely used it is almost universally
disliked\footnote{Citation: I walked around at POPL and the 5 people I
  asked seemed like REALLY bummed about it.}.

The structure of this thesis as follows. In Section~\ref{sec:language}
an ML-like language is described that will serve as target for the
logical relation. In Section~\ref{sec:steps} a step-indexed logical
relation for this language is constructed and discussed, especially to
highlight its short-comings. In Section~\ref{sec:domains} the first
failed step-index-free logical relation is described, the most na\"ive
approach centered around using domain theory. In
Sections~\ref{sec:smi} and~\ref{sec:guarded} a variety of attempts at
step-index-free logical relations are discussed for languages powerful
enough to encode general references, largely built around (guarded)
recursive kinds.

%%% Local Variables:
%%% mode: latex
%%% TeX-master: "../main"
%%% End:

\section{An ML-like Language with General References}\label{sec:language}

In order to make the discussion of logical relations more concrete, a
particular language is necessary. In this section we develop a core
calculus suitable for studying the effects of general references on
reasoning.

The language under consideration here is heavily influenced by the
Modernized Algol discussed by Harper~\citep{Harper:16}. It features a
syntactic separation between commands and expressions. Expressions are
characterized by being unable to depend on in any way on the
heap. Commands may modify the heap using assignables, a mutable
``variable'' that makes clear the binding structure (nominal rather
than substitutive). A crucial component of the system is a modality
for internalizing and suspending commands to treat them as
expressions. Arguably this language is heavier-weight than the ML-like
languages usually discussed which just allow references in any
location. It does confer that advantage that handling state is now a
question of defining the logical action of the command modality and
its complexity is not littered throughout the full logical
relation. Indeed, the definitions of the logical relation at $\to$ or
$\forall$ are almost completely unchanged.

The syntax of our language has three sorts: commands, expressions, and
types defined by the following grammar.
\[
  \begin{array}{lcl}
    \tau & ::= & \alpha \mid \fn{\tau}{\tau} \mid \allNoKind{\alpha}{\tau}
    \mid \cmd{\tau}\\
    c & ::= & \ret{e} \mid \get{\alpha} \mid \set{\alpha}{e} \\
      & \mid & \dcl{\alpha}{e}{c} \mid \bnd{x}{e}{c}\\
    e & ::= & x \mid \ap{e}{e} \mid \lam{x}{\tau}{e} \mid
              \LamNoKind{\alpha}{e} \mid \cmd{e}
  \end{array}
\]
Only one point of this syntax must be clarified, which is the
unfortunate coincidence of $\alpha$s. In an attempt to maintain
consistency with the standard literature on System F and
Harper~\citep{Harper:16} $\alpha$ here refers either to an assignable
(a symbol) or a type variable. Crucially, assignables are \emph{not}
variables. Assignables are bound by the operators of our language and
they do $\alpha$-vary as a variable might but they may not be
substituted for. This justifies comparing bound symbols for equality;
they are placeholders and contain an intrinsic notion of identity in
the form of a binding site. It is nonsensical to talk about a rule
like the following.
\[
  \inferrule{ }{\step{\dcl{\alpha}{v}{c}}{[v/\alpha]c}}
\]
In some sense, this is similar to the confusion that many programmers
have when discussing languages with variables: C-like languages do not
permit substitution because they do not possess variables but rather
assignables. In C a rule like the above is clearly false and leads to
statements such as \verb+1 = 2+. One can understand the difference
between $\bnd{x}{e}{c}$ and $\dcl{x}{e}{c}$ as the difference between
\emph{variables}, defined through substitution, and \emph{assignables}
which are defined through binding identity. The former is like a
let-binding while the latter is closer to a declaration in C. In our
language this separation exists which is why the operators for reading
and writing a mutable cell are indexed by symbol rather than taking an
arbitrary term. There is much to be said on the subject of symbols
and, more generally, nominal binding but it is sadly out-of-scope for
this discussion. The interested reader is referred
to~\citep{Pitts:13}.

The static semantics of the language are divided into three judgments,
the first of which is the judgment ensuring a type is a
well-formed. Informally, a type is well formed in a context $\Delta$
if $\Delta$ contains all the free variables of the type.
\begin{mathpar}
  \declareJudgement{\hasTJ{}{\Delta}{\tau}{\tp}}\\
  \inferrule{
    \alpha \in \Delta
  }{\hasTJ{}{\Delta}{\alpha}{\tp}}\and
  \inferrule{
    \hasTJ{}{\Delta}{\tau_1}{\tp}\\
    \hasTJ{}{\Delta}{\tau_2}{\tp}
  }{\hasTJ{}{\Delta}{\fn{\tau_1}{\tau_2}}{\tp}}\and
  \inferrule{
    \hasTJ{}{\Delta, \alpha}{\tau}{\tp}
  }{\hasTJ{}{\Delta}{\allNoKind{\alpha}{\tau}}{\tp}}\and
  \inferrule{
    \hasTJ{}{\Delta}{\tau}{\tp}
  }{\hasTJ{}{\Delta}{\cmd{\tau}}{\tp}}
\end{mathpar}
In order to explain the statics of expressions and commands two
judgments are necessary and they must depend on each other. This
mutual dependence is a result of the $\cmd{-}$ modality which
internalizes the command judgment. First the expression judgment is
given, it is completely standard except that it must also be fibered
over a specification of the available assignables. This extra context
is necessary in order to make sense of the binding done in
$\mathsf{dcl}$.
\begin{mathpar}
  \declareJudgement{\hasESigJ{\Delta}{\Gamma}{\Sigma}{e}{\tau}}\\
  \inferrule{
    x : \tau \in \Gamma
  }{\hasESigJ{\Delta}{\Gamma}{\Sigma}{x}{\tau}}\and
  \inferrule{
    \hasESigJ{\Delta}{\Gamma, x : \tau_1}{\Sigma}{e}{\tau_2}
  }{\hasESigJ{\Delta}{\Gamma}{\Sigma}{\lam{x}{\tau_1}{e}}{\fn{\tau_1}{\tau_2}}}\and
  \inferrule{
    \hasESigJ{\Delta}{\Gamma}{\Sigma}{e_1}{\fn{\tau_1}{\tau_2}}\\
    \hasESigJ{\Delta}{\Gamma}{\Sigma}{e_2}{\tau_1}\\
  }{\hasESigJ{\Delta}{\Gamma}{\Sigma}{\ap{e_1}{e_2}}{\tau_2}}\and
  \inferrule{
    \hasESigJ{\Delta, \alpha}{\Gamma}{\Sigma}{e}{\tau}
  }{\hasESigJ{\Delta}{\Gamma}{\Sigma}{\LamNoKind{\alpha}{e}}{\allNoKind{\alpha}{\tau}}}\and
  \inferrule{
    \hasESigJ{\Delta}{\Gamma}{\Sigma}{e}{\allNoKind{\alpha}{\tau_1}}\\
    \hasTJ{}{\Delta}{\tau_2}{\tp}
  }{\hasESigJ{\Delta}{\Gamma}{\Sigma}{\Ap{e}{\tau_2}}{[\tau_2/\alpha]\tau_1}}\and
  \inferrule{
    \hasMJ{\Delta}{\Gamma}{\Sigma}{m}{\tau}
  }{\hasESigJ{\Delta}{\Gamma}{\Sigma}{e}{\tau}}
\end{mathpar}
These are the rules of System F with the exception of the final
one. This makes use of the judgment for commands, where
$\hasM{m}{\tau}$ signifies that $m$ is a command which (when executed
on the appropriate heap) will run to a $\ret{v}$ for some
$\hasE{v}{\tau}$. It is worth noting that $\Sigma$ in this judgment is
unimportant except in this rule as well. It is only necessary to track
the available assignables because without this information it is not
possible to determine if a command is well-typed. The rules for
commands are as follows.
\begin{mathpar}
  \inferrule{
    \hasESigJ{\Delta}{\Gamma}{\Sigma}{e}{\tau}
  }{\hasMJ{\Delta}{\Gamma}{\Sigma}{\ret{e}}{\tau}}\and
  \inferrule{
    \alpha \div \tau \in \Sigma
  }{\hasMJ{\Delta}{\Gamma}{\Sigma}{\get{\alpha}}{\tau}}\and
  \inferrule{
    \alpha \div \tau \in \Sigma\\
    \hasESigJ{\Delta}{\Gamma}{\Sigma}{e}{\tau}
  }{\hasMJ{\Delta}{\Gamma}{\Sigma}{\set{\alpha}{e}}{\tau}}\and
  \inferrule{
    \hasESigJ{\Delta}{\Gamma}{\Sigma}{e}{\tau_2}\\
    \hasMJ{\Delta}{\Gamma}{\Sigma, \alpha : \tau_2}{m}{\tau_1}
  }{\hasMJ{\Delta}{\Gamma}{\Sigma}{\dcl{\alpha}{e}{m}}{\tau_1}}\and
  \inferrule{
    \hasESigJ{\Delta}{\Gamma}{\Sigma}{e}{\cmd{\tau_1}}\\
    \hasMJ{\Delta}{\Gamma, x : \tau_1}{\Sigma}{m}{\tau_2}
  }{\hasMJ{\Delta}{\Gamma}{\Sigma}{\bnd{x}{e}{m}}{\tau_2}}\and
\end{mathpar}
The operational semantics of this language can be presented in two
distinct ways differing in the way that the allocation of a fresh heap
cell is done. In classical literature, an expression is evaluated
using traditional small step semantics. Commands in this framework are
represented as a pair $(m, h)$, a command and the heap it is running
on, and there is another stepping relation defined between these
configurations. In \citet{Harper:16} the binding structure of symbols
is preserved in the operational semantics so that a running command is
represented as a configuration $\nu \Sigma.\ (m, h)$ where $\nu$ is a
collection of symbols bound in both $m$ and $h$. The advantage of this
presentation is that it eliminates the choice of how to pick a fresh
symbol during allocation. In the more traditional set up the stepping
relation nondeterministic, allowing for any fresh location to be
chosen. This nondeterminism complicates many metatheoretic properties
and obscures what is really happening since the language is
fundamentally not nondeterministic! The formal presentation has just
obscured the precise meaning of a construct and approximated it with
nondeterminism. For the problems discussed in this thesis, however,
this advantage is not significant and it proves to be (while not
impossible) onerous to convert from semantic objects to $\Sigma$ when
running programs. Indeed a slightly different presentation from Harper
would be required in any case since programs will be run on heaps
containing locations varying over not syntactic but semantic
types. This means for expository purposes the traditional operational
semantics are chosen.
\begin{mathparpagebreakable}
  \declareJudgement{\valueJ{v}}\\
  \inferrule{ }{\valueJ{\lam{x}{\tau}{e}}}\and
  \inferrule{ }{\valueJ{\LamNoKind{\alpha}{e}}}\and
  \inferrule{ }{\valueJ{\cmd{m}}}\\
  \declareJudgement{\step{e}{e'}}\\
  \inferrule{
    \step{e_1}{e_1'}
  }{\step{\ap{e_1}{e_2}}{\ap{e_1'}{e_2}}}\and
  \inferrule{
    \valueJ{v}\\
    \step{e_2}{e_2'}
  }{\step{\ap{v}{e_2}}{\ap{v}{e_2'}}}\and
  \inferrule{
    \valueJ{v}
  }{\step{\ap{(\lam{x}{\tau}{e})}{v}}{[v/x]e}}\and
  \inferrule{
    \steps{e}{e'}
  }{\steps{\Ap{e}{c}}{\Ap{e'}{c}}}\and
  \inferrule{ }{\step{\ap{(\LamNoKind{\alpha}{e})}{c}}{[c/\alpha]e}}\\
  \declareJudgement{\finalJ{m}{h}}\\
  \inferrule{
    \valueJ{v}
  }{\finalJ{m}{h}}\\
  \declareJudgement{\stepsM{m}{h}{m'}{h'}}\\
  \inferrule{
    \step{e}{e'}
  }{\stepM{\ret{e}}{h}{\ret{e'}}{h}}\and
  \inferrule{
    \step{e}{e'}
  }{\stepM{\bnd{x}{e}{m}}{h}{\bnd{x}{e'}{m}}{h}}\and
  \inferrule{
    \stepM{m_1}{h}{m_1'}{h'}
  }{\stepM{\bnd{x}{\cmd{m_1}}{m_2}}{h}{\bnd{x}{\cmd{m_1'}}{m_2}}{h'}}\and
  \inferrule{
    \valueJ{v}
  }{\stepM{\bnd{x}{\cmd{\ret{v}}}{m}}{h}{[v/x]m}{h}}\and
  \inferrule{
    \step{e}{e'}
  }{\stepM{\dcl{\alpha}{e}{m}}{h}{\dcl{\alpha}{e'}{m}}{h}}\and
  \inferrule{
    \valueJ{v}\\
    \alpha \disjoint h
  }{\stepM{\dcl{\alpha}{v}{m}}{h}{m}{h[\alpha \mapsto v]}}\and
  \inferrule{ }{\stepsM{\get{\alpha}}{h}{\ret{h(\alpha)}}{h}}\and
  \inferrule{
    \step{e}{e'}
  }{\stepsM{\set{\alpha}{e}}{h}{\set{\alpha}{e'}}{h}}\and
  \inferrule{
    \valueJ{v}
  }{\stepsM{\set{\alpha}{v}}{h}{\ret{v}}{h[\alpha \mapsto v]}}\and
\end{mathparpagebreakable}
This gives a precise account of the language which we want to
construct a logical relation for. Still undefined, however, is the
notion of equivalence that this logical relation should capture. It
was informally sketched in the introduction as the ability to replace
one program with the other in an arbitrary context. It is now possible
to make this definition formal. First, a context formally is a term
with a distinguished hole in it, written $\hole$. They are generated
from the following grammar.
\[
  \begin{array}{lcl}
    C &::=& \hole \mid \ap{C}{e} \mid \ap{e}{C} \mid \lam{x}{\tau}{C} \mid
            \LamNoKind{\alpha}{C} \mid \Ap{C}{c} \mid \cmd{C_m}\\
    C_m &::=& \ret{C} \mid \bnd{x}{C}{m} \mid \bnd{x}{e}{C_m}\\
      & \mid & \dcl{\alpha}{C}{m} \mid \dcl{\alpha}{e}{C_m} \mid \set{\alpha}{C}\\
    C^m &::=& \ap{C^m}{e} \mid \ap{e}{C^m} \mid \lam{x}{\tau}{C^m} \mid
    \LamNoKind{\alpha}{C^m} \mid \Ap{C^m}{c} \mid \cmd{C_m^m}\\
    C_m^m &::=& \hole \mid \ret{C^m} \mid \bnd{x}{C^m}{m} \mid \bnd{x}{e}{C_m^m}\\
 & \mid & \dcl{\alpha}{C^m}{m} \mid \dcl{\alpha}{e}{C_m^m} \mid \set{\alpha}{C^m}\\
  \end{array}
\]
Two of varieties contexts, $C$ and $C_m$, come equipped with a natural
operation replacing $\hole$ with an expression, denoted $C[e]$, while
the other two possess an identical operation for commands. Notice that
crucially this operation is capturing; $e$ is allowed to make use of
the variables bound in $C$. This is crucial to allowing contextual
equivalence to work with open terms. A small point of interest is that
the fact that we have two sorts which may mention each other leads to
four different varieties of contexts. This quadratic growth continues
which makes contextual equivalence difficult to define for complex
languages. This has been addressed in Crary~\citep{Crary:17} by
defining contextual equivalence as the unique relation satisfying
Theorem~\ref{thm:language:cxt}.
n
Finally, contexts can be equipped with a notion of typing. This typing
relation for $C$ for instance expresses that when filled with
a program $\hasESigJ{\Delta}{\Gamma}{\Sigma}{e}{\tau}$ they produce a
term $\hasESigJ{\Delta'}{\Gamma'}{\Sigma'}{C[e]}{\tau'}$. This typing
relation is unsurprising and therefore omitted.

With this, we are in a position to define contextual equivalence for
our language.
\begin{defn}\label{def:language:heapmatch}
  A heap, $h$, is said to match $\Sigma$, written $h : \Sigma$, if for
  all $\alpha \div \tau \in \Sigma$,
  $\hasESigJ{\cdot}{\cdot}{\Sigma}{h(l)}{\tau}$.
\end{defn}
\begin{defn}\label{def:language:kleene}
  To programs are said to be Kleene equal at $\Sigma$, written
  $e_1 \simeq_\Sigma e_2$ if for all $h : \Sigma$ and $m_1, m_2$ so
  that $\steps{e_i}{m_i}$ then $(m_1, h) \Downarrow$ and
  $(m_2, h) \Downarrow$ or $(m_1, h) \Uparrow$ and
  $(m_2, h) \Uparrow$.
\end{defn}
This definition of Kleene equality is slightly non-standard in that it
requires that the expressions are commands and that it only checks if
the coterminate, not that they terminate at the same value. This
deviation is due to the fact that there is no base observable type in
this language and so there is no observation to by made besides
termination. Furthermore, it is not hard to prove that expressions
always terminate (they behave as System F with a new base type) so
\emph{no} interesting observations may be made of just expressions. We
will also make use of the notation $(m_1, h_1) \simeq (m_2, h_2)$ to
signify just the final part of this definition. Finally, we can lift
Kleene equality to a definition of contextual equivalence.
\begin{defn}\label{def:language:cxt}
  To programs are contextually equivalent,
  $\equivESigJ{\Delta}{\Gamma}{\Sigma}{e_1}{e_2}{\tau}$ if for all
  contexts
  $\hasCEJ{\Delta}{\Gamma}{\Sigma}{\tau}{C}{\cdot}{\cdot}{\Sigma'}{\cmd{\tau'}}$,
  $C[e_1] \simeq_{\Sigma'} C[e_2]$.

  Similarly, two commands are said to contextually equivalent if
  $\hasCEJ{\Delta}{\Gamma}{\Sigma}{\tau}{C^m}{\cdot}{\cdot}{\Sigma'}{\cmd{\tau'}}$,
  $C^m[m_1] \simeq_{\Sigma'} C^m[m_2]$.
\end{defn}
This definition formalizes our earlier intuitions and it makes obvious
the issues that were mentioned with contextual equivalence previously.
In order to establish contextual equivalence we must write a proof
which handles an arbitrary choice of $C$, clearly a difficult
task.

Most proofs of contextual equivalence make use of a coinductive
characterization of it. Contextual equivalence is the coarsest
consistent congruence relation on terms. A consistent relation is one
where if two terms are related by the relation than they
coterminate. A congruence relation is an equivalence relation
(reflexive, symmetric, transitive) which respects the structure of the
terms. To make this precise, a pair of relations $(\relR, \relS)$ is
said to respect the structure of terms if the following inferences are
valid.
\begin{mathparpagebreakable}
  \inferrule{
    e_1 \relR e_2
  }{\lam{x}{\tau}{e_1} \relR \lam{x}{\tau}{e_2}}\and
  \inferrule{
    e_1 \relR e_1'\\
    e_2 \relR e_2'
  }{\ap{e_1}{e_2} \relR \ap{e_1'}{e_2'}}\and
  \inferrule{
    e_1 \relR e_2
  }{\LamNoKind{\alpha}{e_1} \relR \LamNoKind{\alpha}{e_2}}\and
  \inferrule{
    e_1 \relR e_2
  }{\Ap{e_1}{c} \relR \Ap{e_2}{c}}\and
  \inferrule{
    m_1 \relS m_2
  }{\cmd{m_1} \relR \cmd{m_2}}\and
  \inferrule{
    e_1 \relR e_2
  }{\ret{e_1} \relS \ret{e_2}}\and
  \inferrule{
    e_1 \relR e_2\\
    m_1 \relS m_2
  }{\bnd{x}{e_1}{m_1} \relS \bnd{x}{e_2}{m_2}}\and
  \inferrule{
    e_1 \relR e_2\\
    m_1 \relS m_2
  }{\dcl{\alpha}{e_1}{m_1} \relS \dcl{\alpha}{e_2}{m_2}}\and
  \inferrule{
    e_1 \relR e_2\\
  }{\set{\alpha}{e_1} \relS \set{\alpha}{e_2}}\and
  \inferrule{
  }{\get{\alpha} \relS \get{\alpha}}
\end{mathparpagebreakable}
The standard result is then captured by the following theorem.
\begin{thm}\label{thm:language:cxt}
  Contextual equivalence on expressions and commands is the coarsest
  consistent congruence.
\end{thm}
This concludes the discussion of the language under consideration. We
now turn to discussing logical relations for this language.

%%% Local Variables:
%%% mode: latex
%%% TeX-master: "../main"
%%% End:

\section{A Step-Indexed Logical Relation}\label{sec:steps}

Before diving into the various approaches for constructing a logical
relation without step-indexing, it is well worth the time to see how
a logical relation can be done with it. The purpose of this section is
to sketch the complication intrinsic to any logical relation and show
how step-indexing obliterates them, though at a high cost.

Our logical to begin with a mapping from types to semantic types
(merely sets of terms). In order to handle impredicative polymorphism
Girard's method~\citep{Girard:71,Girard:72}, see \citet{TODO-PFPL} for
a comprehensive explanation of the technique. This means that our
logical relation is of the form
\[
  \den{-}_{-} : \types \to \typesEnv \to \pow{\term \times \term}
\]
The central challenge is of course the meaning of $\den{\cmd{\tau}}$:
the action of the logical relation at commands. At an intuitive level,
for two commands are rather like (partial) functions: they map heaps to heaps
and a return value. Drawing inspiration from how logical relations for
functions are defined, we might write the following for the definition
the logical relation.
\begin{align*}
  \den{\cmd{\tau}}_\eta& \triangleq \{(e_1, e_2) \mid\\
  &\exists m_1, m_2.\ \steps{e_i}{\cmd{m_i}} \land{}\\
  &\forall h_1 \sim h_2.
  \ (m_1, h_1) \simeq (m_2, h_2) \land{}\\
  &\quad \forall v_1, h_1', v_2, h_2'.
  \ (\stepsM{m_1}{h_1}{\ret{v_1}}{h_1'} \land \stepsM{m_2}{h_2}{\ret{v_2}}{h_2'})\\
  &\qquad \implies (h_1' \sim h_2' \land (v_1, v_2) \in \den{\tau}_\eta)
\end{align*}
Here left undefined is the definition of $\sim$ between two
heaps. This is in fact a major issue because there appears to be no
good way to identify when two heaps ought to be equal. The first issue
here is that semantic equality of terms (be it contextual or logical)
is type-indexed. This means that in order to compare heaps pointwise
for equality (a reasonable though still wrong idea) requires that we
at least know the types of the entries. Furthermore, we shouldn't
compare these heaps for equality at all locations necessarily, two
heaps should only need to agree on the cells that the programs are
going to use. This is a significant concept if we want to prove
programs to be equivalent which do not use the heap identically. For
instance, consider the two programs:
\[
  \dcl{\alpha}{1}{\ret{\cmd{\get{\alpha}}}} \qquad\qquad
  \ret{\cmd{\ret{1}}}
\]
These are contextually equivalent (the assignable of the first program
is hidden from external manipulation) and yet they allocate in
different ways. So $\sim$ must not be \emph{merely} pointwise equality
in the most general case. Additionally, proving that these two
programs are equal requires showing that $h_1 ~ h_2$ if and only if
$h_1(\alpha) = 1$. That is, this program doesn't merely require that
heap cells contain values of some syntactic type, but they may need to
belong to an arbitrary semantic type. In order to reconcile these
constraints, one thing is clear: the logical relation must somehow
vary depending on the state that the heap is supposed to be in. It is
simply not the case that programs that are equivalent in a heap where
no cells are required to exist if and only if they're equivalent in a
heap where one cell is required to exist.

The solution to this is called a Kripke logical
relation~\citep{TODO-KRIPKE}. Kripke logical relations, more generally
sheaf models, are a recurring phenomenon in computer
science~\citep{TODO-KRIPKE-STUFF}. The recurring theme of Kripke
logical relations, or sheaf models, is to abandon the notion of a
single global truth and judge truth relative to a current state of the
world. Rather than considering $(e_1, e_2) \in \den{\tau}_\eta$, we
should consider at some world $w$: $(e_1, e_2)
\den{\tau}_\eta(w)$. Now the collection of worlds should be
\emph{ordered} by a reachability relation $\reach$. Importantly,
Kripke logical relations should be monotone in relation to $\reach$:
\[
  \forall w_1, w_2.
  \ w_1 \reach w_2 \land (e_1, e_2) \in \den{\tau}_{w_1}
  \implies (e_1, e_2) \in \den{\tau}_{w_2}
\]
Intuitively, if we know that some fact holds at a world, $w_1$, and we
add more knowledge to $w_1$ to reach $w_2 \breach w_1$ it should not
reduce what we know to be true.

In this case, the choice of Kripke world is meant to be express the
current state of the heap that programs are being compared at, or at
least, what is known about it. What is this world concretely however?
As a first cut, one could consider a simple collection of symbols and
types. That is,
$\worlds \triangleq \powfin{\assignables \times \types}$. We can change
our clause for the logical relation to take these worlds into
account.
\begin{align*}
  \den{\cmd{\tau}}_\eta\different{(w_1)}& \triangleq \{(e_1, e_2) \mid\\
  &\exists m_1, m_2.\ \steps{e_i}{\cmd{m_i}} \land{}\\
  &\forall \different{w_2 \breach w_1}.\ \forall h_1 \different{\sim_{w_2}} h_2.
  \ (m_1, h_1) \simeq (m_2, h_2) \land{}\\
  &\quad \forall v_1, h_1', v_2, h_2'.
  \ (\stepsM{m_1}{h_1}{\ret{v_1}}{h_1'} \land \stepsM{m_2}{h_2}{\ret{v_2}}{h_2'})\\
  &\qquad \implies (\different{\exists w_3 \breach w_2}.
    \ h_1' \different{\sim_{w_3}} h_2' \land (v_1, v_2) \in \den{\tau}_\eta\different{(w_3)})
\end{align*}
This addition of the Kripke worlds is largely forced. We must quantify
over all possible $w_2$ extending $w_1$ at the beginning: if this was
elided then $\den{\cmd{\tau}}_\eta$ would not be monotone. The
extension at the end, $w_3 \breach w_2$, is so that the world may be
updated to reflect the changes that were caused by allocating new
cells or updating existing ones. Still unexplained is
$h_1 \sim_w h_2$. At this point it can be defined in a slightly more
refined way since $w$ at least specifies what cells we ought to
compare for equality. The notion of equality that we want is
problematic though.
\begin{itemize}
\item If some stronger notion of equality than logical equivalence,
  such as $\alpha$-equivalence, is used the fundamental theorem will
  fail in the clause for $\set{\alpha}{e}$.
\item If a weaker equality than logical equivalence is used then the
  fundamental theorem will fail in the clause for $\get{\alpha}$.
\item If logical equivalence itself is used, the definition is will
  become ill-founded. This is because the heap may contain cells with
  a type larger than $\cmd{\tau}$.
\end{itemize}
What is needed is an judo throw in the vein of Girard's
method. Instead of attempting to decide what the equality for a
particular heap location should be in the definition of $\sim_w$, it
should be told to us already by $w$. This idea, originating with
\citet{TODO-PITTS-AND-STARK}, means that our Kripke worlds should
instead satisfy the relation:
\[
  \worlds = \assignables \pto \pow{\term \times \term}
\]
Now the world extension relation is defined by the following.
\[
  w_1 \reach w_2 \triangleq
  \dom(w_1) \subseteq \dom(w_2) \land
  \forall \alpha.\ w_1(\alpha) = w_2(\alpha)
\]
This version is much more plausible. With the definition of the clause
of the logical relation described previously together with the
following definition of $\sim_w$ the logical relation is well-defined.
\[
  h_1 \sim_w h_2 \triangleq \forall \alpha \in \dom(w).
  \  (h_1(\alpha), h_2(\alpha)) \in w(\alpha)
\]
The issue here is more subtle and causes the fundamental theorem to
fail in the rule for allocation: what relation should we pick when a
fresh cell is allocated? It seems that the only choice when allocating
a cell of type $\tau$ at world $w$ is to extend our world with
$\den{\tau}_\eta(w)$. The complication arises when we allocate more
cells later and move to a fresh world. The relation at $\alpha$ is now
stale, it refers to an outdated world and doesn't allow for
equivalences which are true at this new world but were previously
false. To concretely see this, consider the program:
\begin{align*}
  \cmd{
    &\dcl{\alpha}{\lam{x}{\nat}{\cmd{\ret{1}}}}{\\
      &\quad \ret{\cmd{\set{\alpha}{\lam{x}{\nat}{
          \bnd{x}{\cmd{\get{\alpha}}}{\\ & \qquad\qquad\qquad\qquad\ret{x + 1}}
        }}}}
    }
  }
\end{align*}
This style of program is used to encode recursion in this language in
general. In this case, however, the central point of interest is that
$\alpha$ is updated to contain a command which mentions $\alpha$. In
logical relation, $\alpha$ could only ever contain terms in
$\den{\fn{\nat}{\cmd{\nat}}}(\emptyset)$. In particular, this never
includes a command which mentions $\alpha$. This prevents this
perfectly type safe program from being included in the logical
relation. This will mean that the fundamental theorem fails.

What is to be done here? The root of the issue is that when we
allocate a cell it is impossible to determine precisely what programs
will occupy it because programs in the cell may mention cells that are
yet to be allocated at the time of the construction. What is needed is
for the semantic type stored in a heap cell to vary according to the
world. This fixes an asymmetry between the Kripke world and the
logical relation: the Kripke world supposedly maps locations to
semantic types but the semantics types (as determined by the logical
relation) vary of the Kripke worlds. This leads us to the final form
of the definition of Kripke worlds.
\[
  \worlds = \assignables \pto (\worlds \to \pow{\term \times \term})
\]
Herein lies the rub, this definition of the set of worlds is precisely
what is required for this logical relation. It is not a set
though. A simple cardinality argument shows that there can be no such
set since it would have to be larger than its own power set. This is
not an easily avoided problem. It is unknown how to simply avoid this
using ingenuity in the choice of the Kripke world.

This is where step-indexing enters the picture. Step-indexing after
all was introduced to break precisely these sort of circularities and
``solve'' recursive equations up to an approximation.

With step-indexing, the space of semantic types becomes indexed by
natural numbers and we will likewise index the world by these same
natural numbers. The idea is that a world at stage $n$ maps to
assignables to semantic types which only vary at the previous $n - 1$
stages, past $n - 1$ they are simply constant. The issue is that this
approach requires a number of complex definitions which obscure the
underlying intent: to solve this recursive equation.

Instead of slogging through the classical step-indexed definitions, we
will make use of a more modern categorical approach. Instead of
working with sets, from the beginning we will work with sets varying
over natural numbers: presheaves over $\omega$.
\begin{defn}
  A presheaf over a category $\Ccat$ is a functor from $\op{\Ccat}$ to
  $\SET$. Presheaves form a category with morphisms being natural
  transformations. This category is written $\presheaves{\Ccat}$.
\end{defn}
\begin{example}
  A presheaf over $\omega$ is a family of sets $(X_i)_{i \in \nat}$
  with a map $r_n : X_{n + 1} \to X_n$ for all $n$ called restriction
  maps. A map between presheaves is then a family of maps
  $f_n : X_n \to Y_n$ so that the following commutes for all $n$.
  \[
    \begin{tikzcd}
      X_{n + 1} \ar[r, "f_{n + 1}"] \ar[d, swap, "r_n"] & Y_{n + 1} \ar[d, "r_n"]\\
      X_n \ar[r, swap, "f_n"] & Y_n
    \end{tikzcd}
  \]
\end{example}
Rather than working directly with sets then, we can instead work with
presheaves synthetically. For instance, we can define the exponential
of two presheaves rather than mucking about to create an implication
which interacts properly with the step.
\begin{lem}
  The (co)limit (a generalized version of products and sums) of
  presheaves is determined pointwise.
\end{lem}
The above theorem tells us, for instance, that if we want to form the
product of two presheaves, $X$ and $Y$, over $\omega$, at time $n$ it's just the
product of the $X(n)$ and $Y(n)$. More generally, this lemma gives a
wide-variety of ways to construct complex presheaves from simple ones
without ever having to deal with the step manually. The main missing
element is the ability to form exponentials, the categorical analog of
functions. For presheaves these are slightly more complicated since
the function must respect the indexing structure.
\begin{lem}
  Given two presheaves $X, Y : \presheaves{\Ccat}$, the exponential
  between them is
  \[
    (Y^X)(c) \triangleq \hom(\yoneda(c) \times X, Y)
  \]
  where $\yoneda$ is the Yoneda embedding, defined by the following.
  \[
    \yoneda(c) = hom_\Ccat(-, c)
  \]
\end{lem}
This definition may seem abstract but it is, importantly,
monotone and so determines a valid presheaf. The real power of this
approach is that \emph{it does not matter} that this definition is
complex. The point is that this definition determines a function of
presheaves in that it contains elements we can apply and only elements
we can apply and beyond that its construction is entirely
irrelevant. The categorical methodology of only caring about how an
object relates through morphisms to the rest of the category is
designed to let us avoid having to think too hard about any precise
construction of an object. % TASTE IT.

The constructions so far have given us a wide variety of constructible
presheaves but nothing thus far has increased our expressive power of
what we had in $\SET$. For that we need the ability to solve certain
recursive equations. In fact, it will turn out that we can solve
\emph{guarded} domain equations in $\presheaves{\omega}$. In order to
see this, first let us define a functor on $\presheaves{\omega}$.
\begin{defn}
  The later functor,
  $\later : \presheaves{\omega} \to \presheaves{\omega}$, is defined
  on objects as follows.
  \begin{align*}
    (\later X)(n + 1) &= X(n)\\
    (\later X)(0) &= \{\star\}
  \end{align*}
  This family of sets is a presheaf in the obvious way:
  $r_0 = \lambda x.\ \star$ and $r_{n + 1}$ is the $r_n$ of $X$. The
  action on morphisms is as follows.
  \begin{align*}
    (\later f)(n + 1) &= f_n\\
    (\later f)(0) &= \lambda x.\ \star
  \end{align*}

  The later functor comes equipped with a natural transformation
  $\delay : 1 \to \later$. Explicitly, this means $\delay$ is a family
  of maps $\delay_X : X \to \later X$ so that
  $\delay_Y \circ f = f \circ \delay_X$ for any $f : X \to Y$.
\end{defn}
The later modality is the logical essence of step-indexing. The key
idea of step-indexing, after all, is not merely that it suffices to
ensure that two programs are related for all steps in order to
conclude that they're related. The second, subtler, key idea is that
in order for two programs to be related for $n$ steps, a subcomponent
of these programs needs to be related for only $n - 1$ steps. If this
was not the case, then it would never be possible to decrement the
step-index and the exercise would have been largely moot. The later
modality internalizes the idea of ``$n - 1$ steps'' and so it lets us
talk about (uniformly) constructing an element at stage $n$ from an
element at stage $n - 1$. The $\delay$ operation even crystallizes the
monotonously of the logical relation: if we have an element at stage
$n$ we can always construct one at stage $n - 1$.

A crucial point about using natural numbers for step-indexing is that
natural numbers are well-founded. There's no way to pick a natural
number that we can decrement forever. This justifies reasoning by
induction on the natural numbers. In particular, it justifies
constructing an object at any stage provided we can take a
construction of it at stage $n - 1$ and construct it at stage
$n$. This principle may be expressed with the later modality: it's
precisely the existence of a family of morphisms:
\[
  \fix_A : A^{\later A} \to A
\]
These morphisms allow us to take a particular sort of fixed-point at
any level of maps $\later A \to A$. These morphisms belong to the
class of contractive morphisms: morphisms $X \to Y$ that can be
factored as $X \to \later X \to Y$. In particular, suppose we have
some $f : B \times \later A \to A$. Then the following equality holds:
\[
  \fix_A \circ \transpose{f} = f \circ \pair{1}{\delay_A \circ \fix_A \circ \transpose{f}}
\]
Written out using more type-theoretic notation this is easier to
process.
\[
  \forall a, b.\ \fix_A(f(b, -))(a) = f (b, \delay_A(\fix_A(f(b))(a)))
\]
This fixed-point construction allows us to build many interesting
structures inside this category. Through the inclusion of
universes~\citep{TODO-UNIVERSES}, one can even construct solutions to
certain (small) domain equations entirely internally to the
category~\citep{TODO-QUACK-FIXED-POINTS}. The basic idea is that
$\presheaves{\omega}$ can be easily made to support an object,
$\mathcal{U}$, for which global sections classify other smaller
objects. This is the categorical analog of a Grothendieck
universe~\citep{TODO-GROTHENDIECK} and proves to be an important
concept in the semantics of dependent type theory. Then, by using
$\fix_\univ$ it is possible to construct the recursively defined
presheaves necessary to build logical relations. This approach, while
not explicit, is latent in the work done in constructing logical
relations in guarded type theory and guarded
logics~\citep{TODO-LR-IN-GUARDED}.

We will consider the more traditional approach of using proper domain
equations rather than just small ones. In this case, the traditional
domain theoretic approach is to use functors and to construct fixed
points of them. This approach in domain theory originates from
\citet{TODO-PLOTKIN-SMYTH} and subsumed prior work by Scott
constructing recursive domain equations by
hand~\citep{TODO-SCOTT}. One potential issue is that we would like to
solve domain equations which are not strictly speaking functorial. In
particular, we want to handle the case where the equation contains
both positive and negative occurences. We handle these by viewing them
as functors
$F : \op{\presheaves{\omega}} \times \presheaves{\omega} \to
\presheaves{\omega}$
in the style of \citet{TODO-MIXED-VARIANCE}. The question then
becomes, for what functors can we construct an object, $I$, so that
$F(I, I) \cong I$, an invariant object. This construction is a
generalization of the theorem of America and
Rutten~\citep{TODO-AMERICA}. The precise accounting the theorem can
be found in \citet{Birkedal:11} allowing for any enriched functor
between categories enriched in sheaves over a complete Heyting algebra
with a well-founded basis. For our purposes though a much less general
theorem suffices.
\begin{defn}
  A functor, $F$, is said to be locally contractive if there is a
  family of contractive morphisms
  \[
    f_{X_1, X_2, Y_1, Y_2} :
    X_1^{X_2} \times Y_2^{Y_1} \to F(X_2, Y_2)^{F(X_1, Y_1)}
  \]
  So that this family is family respects composition and identity and
  so that for each $\pair{g}{h} : 1 \to X_1^{X_2} \times Y_2^{Y_1}$
  then the following equation holds.
  \[
    \transpose{f_{X_1, X_2, Y_1, Y_2} \circ \pair{g}{h}} =
    F(\transpose{g}, \transpose{h})
  \]
  If just the latter condition holds $F$ is said to have a strength
  $f$ and if the family of morphisms is not contractive we shall call
  the functor locally nonexpansive.
\end{defn}
This definition is subtly different than some presentations of this
theorem, which require only that $F$ have a strength comprised of
contractive morphisms. This version essentially states that $F$ must
be an enriched functor for the category $\presheaves{\omega}$ enriched
over itself. The central theorem, Theorem~\ref{thm:steps:fixed-points}
is true even in this weakened version but I am unable to find a proof
of this fact and I cannot list a theorem that I cannot prove
myself. The relative strengthening that this generality provides seems
vacuous as well, none of the examples of the fixed point theorem used
either in this section or Section~\ref{sec:guarded} require it.
\begin{example}
  The functors $\times$, $\to$, and $+$ are all locally
  nonexpansive. The functor $\later$ is locally
  contractive. Additionally, any locally contractive functor is locally
  nonexpansive.
\end{example}
\begin{lem}
  The composition of locally nonexpansive functors is locally
  nonexpansive. The composition of a locally contractive functor with
  a locally nonexpansive functor is locally contractive.
\end{lem}
We may now state the theorem which makes working a
$\presheaves{\omega}$ so appealing.
\begin{thm}\label{thm:steps:fixed-points}
  For any locally contractive functor $F$ there exists an object $I$,
  unique up to isomorphism, so that $F(I, I) \cong I$.
\end{thm}
This theorem allows us to solve a wide variety of domain equations,
including a small modification of the crucial one for Kripke
worlds. First, let us define $\upred{S}$ as a ``time varying
predicate'' on a set $S$. This will be the analog of the sets of terms
used in step-indexed models.
\[
  \upred{X}(n) \triangleq \{(m, x) \mid x \in X \land m < n\}
\]
There is an evident restriction mapping sending a uniform predicate at
stage $n + 1$ to the uniform predicate containing the entries indexed
by numbers smaller than $n$. Next, the presheaf of partial maps from
assignables to presheaves can be defined as follows:
\[
  (\assignables \pto X)(n) \triangleq
  \{f \in X(n)^{F} \mid F \subseteq \assignables \land F\ \mathsf{finite}\}
\]
Importantly, it is not hard to see that $\assignables \pto -$ defines
a functor which is locally nonexpansive. The final piece necessary is
a subobject of the exponential $Y^{\assignables \pto X}$ which
isolates those functions which are monotone. In order to define this,
we must assume that $Y$ comes equipped with a partial order. Viewed
from the external perspective this is a family of relations indexed by
natural numbers which respects reindexing. If one views the partial
order internally, however, it is just a normal partial order. The
subobject desired can be defined internally in a way which is
obviously correct using the internal logic of $\presheaves{\omega}$ as
follows.
\begin{align*}
  (\assignables \pto X) \mto Y &\triangleq \\
  \{f : Y^{\assignables \pto X} \mid{}&
   \forall w_1 w_2 : \assignables \pto X.\ w_1 \reach w_2 \implies f(w_1) \le f(w_2)\}
\end{align*}
Constructed externally this presheaf is somewhat more difficult to
construct, but still quite possible. It's
\begin{align*}
  ((\assignables \pto X) \mto Y)(n) &= \\
  \{f : (Y^{\assignables \pto X})(n) \mid{}&
   \forall m \le n.\ \forall w_1 w_2 : (\assignables \pto X)(n).\\
  &\qquad w_1 \reach_n w_2 \implies f(m)(\star, w_1) \le_n f(m)(\star, w_2)\}
\end{align*}
This external definition is neither more precise nor more intuitive,
so it is more helpful to read the internal version and understand that
all the quantifiers in the predicate implicitly handle the steps
correctly.

At the end of these constructions, we may define the following
contractive functor.
\[
  X \mapsto \later (\assignables \pto X \mto \upred{\term \times \term})
\]
Solving this for a fixed point gives the desired presheaf of Kripke
worlds. Defining the rest of the logical relation proves to be quite
straightforward now that the right definition of Kripke worlds is in
place. The type of our logical relation is slightly different now,
it's defined as a monotone map:
\[
  \den{-}_{-} : \types \to \typesEnv \to \worlds \mto \upred{\term \times \term}
\]
Here $\mto$ is an subobject of the exponential in the presheaf
category so it is indexed by natural numbers and subject to a
naturality condition. Then the crucial clause of our logical relation
can be defined as follows.
\begin{align*}
  \den{\cmd{\tau}}_\eta(n)(w_1)& \triangleq \{(k, e_1, e_2) \mid
  k_1 \le n \land \exists m_1, m_2.\ \steps{e_i}{\cmd{m_i}} \land{}\\
  &\forall w_2 \breach_{k_1} \different{r_{k_1}^n(w_1)}.\ \forall h_1 \different{\sim_{w_2}^{k_1}} h_2.
  \ (m_1, h_1) \simeq (m_2, h_2) \land{}\\
  &\quad \different{\forall d.}\ \forall v_1, h_1', v_2, h_2'.
  \ (\stepsM[\different{d}]{m_1}{h_1}{\ret{v_1}}{h_1'} \land \stepsM{m_2}{h_2}{\ret{v_2}}{h_2'})\\
  &\qquad \implies (\different{\exists w_3 \breach r^{k_2 - d}w_2}.
    \ h_1' \different{\sim_{w_3}} h_2' \land (v_1, v_2) \in \den{\tau}_\eta(k_2 - d)\different{(w_3)})
\end{align*}
We can now also define $\sim$ making use of the following isomorphism.
\[
  \iota : \worlds \cong
  \later (\assignables \pto \worlds \mto \upred{\term \times \term})
\]
The following definition is then quite naturally expressed
internally. The external version merely involves a preponderance of
indices. Let us suppose that $w$ is a world at step $n$
\begin{align*}
  h_1 &{}\sim_w h_2 \defs\\
  &\begin{cases}
    \forall \alpha \in \dom(\iota(w)).\ (n - 1, h_1(\alpha), h_2(\alpha)) \in \iota(w)(\alpha)(r^{n - 1}(w))
    & n > 1\\
    \text{true} & \text{otherwise}
  \end{cases}
\end{align*}
The essence of the definition is the self-application. A heap is
related to another heap at a world if all of the points are related
relative to the same world. It was this circularity and
self-application which drove us to step-indexing in the first place.

The rest of the logical relation is entirely standard step-indexing
and the curious reader is referred to any of the following excellent
resources~\citet{TODO-STEP-INDEXING}.

A final consideration here is that the version of step-indexing
presented in this paper is ``half-baked''. We have eschewed explicit
indexing in the entire construction of the semantic domains and made
use of $\later$ and presheaves, but then chosen to define the logical
relation with explicit indexing. All the same issues that applied to
the construction of the domain of discourse for the logical relation
apply the actual definition of the logical relations itself. Since our
definition of the logical relation is just constructing a subpresheaf
pointwise using standard logical formula and passing around the index,
can we isolate the handling of the index and abstract away from it?

A powerful approach to handling this comes from topos theory. Topos
theory is a subfield of category theory concerned with studying
categories which are similar to $\SET$. It is a standard result in
topos theory that all presheaf categories are toposes and so all the
tools of topos theory are applicable to $\presheaves{\omega}$. For
instance, the explicit definition of $\upred{A}$ can be replaced with
the topos theoretic definition of power sets. Of particular interest
is the ``internal logic\footnote{In fact, the internal logic of a
  topos can be generalized to a full dependent type theory with a
  distinguished impredicative universe. This type theory provides even
  more flexibility than the logic. In it, we could have defined the
  entirety of our construction of Kripke worlds in a setting just like
  programming in type theory.}'' of the topos. It turns out that
$\presheaves{\omega}$ supports a version of intuitionistic
higher-order logic whose interpretation is precisely the standard
set-theoretic interpretation with a little extra work to pass around
the index. Importantly, one can take a presheaf, $A$, and a predicate,
$\phi$, on it and form a subpresheaf $\{ a \in A \mid \phi(a) \}$ so
that every element of this comprehension satisfies $\phi$ as one would
hope.

The internal logic of $\presheaves{\omega}$ includes everything we
need to define our logic including even version of L\"ob induction and
a logical version of later, $\ilater$. Working this way sweeps the
step-index entirely under the rug and obliterates the requirement to
perform explicit index math anywhere. Instead, the frustrating
mismatch of having a fact at $n - 1$ and needing it at $n$ manifests
through the more explicit having $\later \phi$ and needing
$\phi$. Proper index discipline is replaced by the proper handling of
the later modality. For instance, here is the definition of world
satisfaction working in the internal logic.
\[
  h_1 \sim_w h_2 \triangleq
  \forall \alpha \in \dom(w).
  \ \sincl{\iota(w) \zap \delay(w) \zap (h_1, h_2)}
\]
No explicit steps are present only a handful of logical operations to
handle the new type-constructor later. This approach to step-indexing,
whether explicitly topos theoretic or not is the subject of a long
line of work~\citep{TODO-LOGICAL-STEPS}. Importantly, this logical
approach to step-indexing enables abstraction from the natural numbers
themselves and suggests more general notions of
step-indexing~\citep{Svendsen:16}. The most approachable introduction to
this approach is \citet{Dreyer:09} and \citet{Birkedal:steps:11} provides a
thorough topos-theoretic grounding explicitly using Kripke-Joyal
semantics.

%%% Local Variables:
%%% mode: latex
%%% TeX-master: "../main"
%%% End:

\section{Tying the Knot Using Domains}\label{sec:domains}

This section describes the first and most obvious approach to building
a step-index-free logical relation for state. It would seem that the
real crux of the issue was this Kripke world. Nothing past the
construction of the Kripke world in Section~\ref{sec:step-indexing}
was particularly challenging, especially when a logical approach is
taken. So a natural approach is to find a different setting in which
to solve the following equation
\[
  \worlds \cong \assignables \pto (\worlds \mto \pow{\term \times \term})
\]
Traditionally, domains have been used for exactly this purpose. The
impetus of domain theory as an investigation separate from order
theory in general was the observation by Dana Scott that domains
possessed solutions to equations that sets did not. For instance, a
model of the lambda calculus may be constructed by solving the
equation
\[
  D \cong D \to D
\]
This equation is not possible to solve in sets without trivializing
$D$. In fact, Theorem~\ref{thm:steps:fixed-points} uses a construction
derived originally from domain theory itself.

\subsection{Domain Theory, Briefly}

A domain is a particular sort of partial order which comes equipped
with enough structure to accurately describe a notion of continuous
functions upon it.
\begin{defn}\label{def:domains:directed-set}
  In a partial order $P$, a directed set $D \supseteq P$ is a set so
  that $D \neq \emptyset$ and if $a, b \in D$ then there is a $c \in
  D$ so that $a, b \le c$.
\end{defn}
\begin{defn}\label{def:domains:domain}
  A domain\footnote{In this text domains are directed complete partial
    orders as opposed to any of the myriad varierties of domains one
    might have chosen.} is partial order possessing joins (least upper
  bounds) of all directed subsets. Given a directed subset $D$ we
  write this join as $\join D$.
\end{defn}
Intuitively, elements of a domain represent a (partial) piece of
knowledge. Moving to a larger element means moving to a piece of
compatible knowledge that extends what was already known. The ability
to construct joins of directed sets fits into this intuition as the
ability to take a great many fragments of compatible knowledge and
glue them together. This intuition primarily comes from the example
domain of a partial function. The function is not everywhere defined
so it is not a complete piece of knowledge, but given a collection of
partial functions which agree on the intersection, they can be glued
together.

The extra structure on domains gives rise to a new class of maps that
preserve this structure. These maps are called (Scott) continuous in
that they preserve joins.
\begin{defn}[Scott continuity]
  A map between domains is (Scott) continuous if it preserves the
  joins of directed sets. That is, if $f : D \to E$ and
  $I \subseteq D$ is directed, then
  $f(\join I) = \join \{f(i) \mid i \in I\}$.
\end{defn}
Calling this definition continuity can be justified by considering the
Scott topology on domains, in which continuous functions are precisely
those which preserve upper bounds. For a complete account of the
theory of domains, see for instance \citet{TODO-COMPEDIUM}.
\begin{thm}
  Domains form an a category where morphisms are continuous
  functions. This category is bicartesian closed, in particular the
  set of continuous maps between two domains is a domain.
\end{thm}
The interesting portion of domain theory for these purposes is the
ability to solve domain equations. Specifically, locally continuous
functors of domains again have solutions.
\begin{defn}
  A functor, $F : \DCPO \to \DCPO$ is locally continuous if its action
  on morphisms is a continuous function. That is, there is a
  continuous map of domains $B^A \to F(B)^{F(A)}$. This definition
  generalizes readily to a mixed variance n-ary functor.
\end{defn}
\begin{thm}\label{thm:domains:fixed-points}
  Any mixed variance locally continuous functor
  $F : \op{\DCPO} \times \DCPO \to \DCPO$ has an invariant object $I$
  so that $F(I, I) \cong I$.
\end{thm}

\subsection{Domains for State}

The domains involved in expression the desired equation are quite
straightforward. Power sets are always a complete lattice and
therefore a domain. Scott continuous functions are always monotone so
instead of considering monotone functions it's more natural to pick
Scott continuous functions to get a domain. Finally, partial maps are
another classic example of domains, with the small complication that
they cannot be limited to finite partial maps.

This gives us a proper functor $F : \op{\DCPO} \to \DCPO$ defined by
\begin{align*}
  D \mapsto& \assignables \pto (D \mto \pow{\term \times \term})\\
  f \mapsto& \lambda x.\ \lambda \alpha.\ \lambda d.\ x(\alpha)(f(d))
\end{align*}
There is only one exceptional feature of all of this, the ordering on
the partial map aspect of this domain is nonstandard. A Kripke world
is supposed to map an assignable to a semantic type. The ordering on
these worlds should let us add new mappings, but old mappings should
stay the same. They must in fact, if a location is allocated so that
it can be either $\littrue$ or $\litfalse$, it would be a real
problem if suddenly it could also be a natural number. This ordering,
explicitly, is then defined by the following formula.
\[
  w_1 \reach w_2 \iff \forall \alpha \in \dom(w_1).\ w_1(\alpha) \simeq w_2(\alpha)
\]
This ordering though, however natural, means that $F$ is not locally
continuous and therefore there is no way to construct a solution to
it.

There is only one insight worth mentioning from supposing that a
solution to $F$ did exist. There would still seemingly be a problem
because one of these worlds could have full support, making modeling
allocation impossible. The trick to avoiding this it to define the
logical relation on compact elements~\citep{TODO-COMPEDIUM} of the
world, precisely those worlds enjoying only finite support.

By defining the logical relation on these worlds, it is trivial to
extend it to an arbitrary world by taking $\den{\tau}(w)$ on such a
world to be the join of the results of $\den{\tau}$ on all the compact
elements below $w$. This is automatically continuous and recovers
exactly $\den{\tau}$ on compact elements which are, after all, the
worlds of interest.

%%% Local Variables:
%%% mode: latex
%%% TeX-master: "../main"
%%% End:

\section{Understanding State through (Syntactic) Minimal Invariance}\label{sec:smi}

Now that domains are seen to be unhelpful, there is some question
about where to turn next. There are not many well-studied
algebraically complete categories~\citep{TODO-FREYD} which support
interesting denotational semantics. The semantic structures for
step-indexing that support solving recursive domain equation require
more than continuity or approximation, they require a notion of
closeness for two semantic types. In $\presheaves{\omega}$ for
instance, there was a natural notion of when two points
$p, q : 1 \to P$ were close by checking at what stage they became
equal, eg when $p_n = q_n : \{\star\} \to P$.

Crucially, in order for this to make sense for semantic types every
equality was judged relative to a step. Two semantic types are equal
at $n$ if they agree on equalities that hold at stage $n$ or earlier,
eg, equalities which hold if the two sides are only allowed to run for
$n$ steps. Without step-indexing what could this be replaced by?

The answer is suggested by an idea in domain theory to resolve a
seemingly unrelated problem. The observant reader may have noticed
that there is a crucial strengthening that
Theorem~\ref{thm:step-indexing:fixed-points} provides over
Theorem~\ref{thm:domains:fixed-points}, the former provides unique
solutions. In fact, there may be many nonisomorphic domains which
satisfy a given equation. This state of affairs can prove troublesome
for semantics in domains because it means that not all the properties
of the domain are determined by the equation. There may be ``exotic''
elements which are not required to satisfy the domain equation but are
present none the less which impedes reasoning about recursive domain
equations by induction or similar. \citet{TODO-PITTS} proposes a
solution to this by imposing an additional requirement on the solution
to a domain equation: minimal invariance.

Minimal invariance, precisely, is the statement that given a $D$ so
that $D \cong F(D, D)$ for a particular family of functions $\pi_i : D
\to D$ the following equation holds.
\[
  \join_i \pi_i = 1
\]
The function $\pi_i$ is determined by induction on $i$.
\begin{align*}
  \pi_i &: D \to D\\
  \pi_0 &\defs \bot\\
  \pi_{i + 1} &\defs F(\pi_i, \pi_i)
\end{align*}
The intuition behind this definition is that if $\pi_i$ converges to
$1$ then every element in $D$ must have arisen as the limit of
elements computed by the inclusion $F^n(\bot, \bot) \to D$ which can
be viewed as semantic requirement that elements of $D$ are
finite\footnote{Or rather, determined a the limits of ``finite''
  elements. This is not to be confused here with \emph{compact} elements and
  algebraic domains which share a similar intuition but require
  technically different properties since there is no reason to assume
  that $F$ preserves algebraicness}. It is not difficult to prove that
if $D \cong F(D)$ and $E \cong F(E)$ with $D$ and $E$ both being
minimally invariant then $E \cong D$ and the standard $D_\infty$
construction produces a minimally invariant domain. Therefore, domain
equations have a unique minimally invariant solution and these satisfy
a form of induction through minimal invariance.

An interesting observation made originally by
\citet{TODO-BIRKEDAL-AND-HARPER} and extend by
\citet{TODO-CRARY-AND-HARPER} is that for the class of domains arising
naturally in semantics (so called ``universal domains'') the
projection functions, $\pi_i$, are computable
syntactically. Additionally, since contextual equivalence can be
generalized to contextual approximation in a natural way, minimal
invariance can be stated as a syntactic property of the programming
language itself. In this setting, projections are naturally
type-indexed and this is denoted $\pi_\tau^i : \fn{\tau}{\tau}$. The
definition of $\pi_\tau^i$ behaves almost as an $\eta$-expansion where
after $i$ $\eta$ expansion the function simply diverges.

It is well worth it to take a moment to be precise about the
definitions involved.
\begin{defn}\label{def:smi:contextual-approximation}
  Two terms $\hasESigJ{\Delta}{\Gamma}{\Sigma}{e_1, e_2}{\tau}$ are
  said to be contextually approximate at type $\tau$, written
  $\approxESigJ{\Delta}{\Gamma}{\Sigma}{e_1}{e_2}{\tau}$ if for all
  contexts
  $\hasCEJ{\Delta}{\Gamma}{\Sigma}{\tau}{C}{\cdot}{\cdot}{\Sigma'}{\cmd{\tau'}}$,
  $\steps{C[e_1]}{\cmd{m_1}}$ and $\steps{C[e_2]}{\cmd{m_2}}$ such
  that for any heap $h : \Sigma$ then
  $(m_1, h) \Downarrow \implies (m_2, h) \Downarrow$.
\end{defn}
We will adopt the traditional abuse of abandoning the context
annotations and just writing $e_1 \cless e_2$ when the contexts and
$\tau$ are obvious.
\begin{defn}\label{def:smi:contextual-limits}
  A family of terms $(e_i)_{i \in I}$ is said to have a limit $e$ if
  for all $i \in I$ $e_i \cless e$ and $e$ is the smallest element
  with this property.
\end{defn}
Since limits are only unique up to contextual equivalence, it is not
necessarily well defined to write $\join_i e_i$ as if it were a term
rather than an equivalence class of terms. We will instead often make
use of the notation $\join_i e_i = e$ to signify that $e$ belongs to
this equivalence class.

The idea behind using syntactic minimal invariance is that instead of
defining a single logical relation $\den{-}_{-}$ a pair of logical
relations is defined instead: $\definitely{-}_{-}$ and
$\possibly{-}_{-}$. The definition of these two logical relations are
mutually recursive and provide a decomposition of the logical relation
two monotone halves. The idea is that one should think of
$\definitely{\tau}_{\eta}$ containing equalities that are certainly
true but it does not contain all of them. On the other hand
$\possibly{\tau}_\eta$ contains all equalities that are true as well
as a few that might not be. This suggests a natural relationship
between these two sets where
$\definitely{\tau}_\eta \subseteq \possibly{\tau}_\eta$. The crucial
move that syntactic minimal invariance provides is to show the
following.
\[
  \forall i.\ (e_1, e_2) \in \definitely{\tau}_\eta \implies
  (\ap{\pi_\tau^i}{e_1}, \ap{\pi_\tau^i}{e_2}) \in \possibly{\tau}_\eta
\]
Then, by showing the comparatively easy lemma that
$\possibly{\tau}_\eta$ is always closed under limits when $\eta$ is it
immediately follows that
$\definitely{\tau}_\eta = \possibly{\tau}_\eta$. This maneuver is
useful because these two definitions are naturally monotone therefore
its easy to express a natural definition of recursive types, as is
done in \citet{TODO-CRARY-HARPER} for instance. The only complication
is that environments must generalize from relations to birelations, a
map to a pair of relations rather than a map to a single
relation. Define $\op{\eta}$ to be the environment mapping $\alpha$ to
the relations $\eta(\alpha)$ with the components swapped. The crucial
insight is that when the variance in the definition of the logical
relation swaps, we switch from $\definitely{-}$ to $\possibly{-}$ and
vice-versa, for instance:
\begin{align*}
  \definitely{\fn{\tau_1}{\tau_2}}_\eta &=
  \{(e_1, e_2) \mid \forall (a_1, a_2) \in \possibly{\tau_1}_\eta.
  \ (\ap{e_1}{a_1}, \ap{e_2}{a_2}) \in \definitely{\tau_2}_\eta\}
\end{align*}
This approach seems to naturally fit recursive types and one might
wonder then if it can be scaled to support state just as step-indexing
does\footnote{If this sentence leave the reader in suspense I
  encourage them to glance at the title}.

The key insight is that while the distinction of
$\definitely{-}$/$\possibly{-}$ is not useful, projection functions
give us a notion akin to step-indexing. We can say that two semantic
types are equal at stage $i$ if they are equal upon truncating all the
programs in the equalities by $\pi^i$. It remains then to define
$\pi_\tau^i$ for our language. In order to do this we need a
divergence at the level of expressions, this can be easily
accomplished by adding fixed points or just adding a formal term
$\bot$ so that $\step{\bot}{\bot}$.
\begin{align*}
  \pi_\tau^0 &= \lam{x}{\tau}{\bot}\\
  \pi_{\fn{\tau_1}{\tau_2}}^{i + 1} &=
  \lam{f}{\fn{\tau_1}{\tau_2}}{\lam{x}{\tau_1}{\ap{\pi_{\tau_2}^i}{(\ap{f}{(\ap{\pi_{\tau_1}^i}{x})})}}}\\
  \pi_{\allNoKind{\alpha}{\tau}}^{i + 1} &=
  \lam{f}{\allNoKind{\alpha}{\tau}}{\LamNoKind{\alpha}{\ap{\pi_\tau^i}{(\Ap{f}{\alpha})}}}\\
  \pi_{\cmd{\tau}}^{i + 1} &=
  \lam{x}{\cmd{\tau}}{\cmd{\bnd{x'}{x}{\ret{\ap{\pi_\tau^i}{x'}}}}}
\end{align*}

%%% Local Variables:
%%% mode: latex
%%% TeX-master: "../main"
%%% End:

\section{A Logical Relation in $\presheaves{\omega}$ with Syntactic Minimal Invariance}\label{sec:guarded}

The final proposed approach to handling logical relations for state is
to instead study a language which allows us to merely encode
state. This is not sufficient to solve many of the original goals of
this work but it's a simpler problem and a natural stop along the way
to a proper logical relation. The natural candidate for a logical
relation is to consider (guarded) recursive kinds as described by
\citet{Pottier:11}. The motivation for doing this is that a language
with guarded recursive kinds is quite complex but in a very different
way than a language with state. It does not require Kripke worlds in
particular, the recursion is instead present in the kind structure. By
being present in the structure of types itself it is easier to handle
with type-based methods.

Recursive kinds require us to switch to a richer language where the
types themselves support a form of computation. For our purposes, the
language is given by the following grammar.
\[
  \begin{array}{lcl}
    e & ::= & x \mid \lam{x}{\tau}{e} \mid \ap{e}{e} \mid \Lam{\alpha}{\kappa}{e} \mid \Ap{e}{c}\\
    c, \tau & ::= & \alpha \mid \fn{\tau}{\tau} \mid \all{\alpha}{\kappa}{\tau} \\
      & \mid &  \lam{\alpha}{\kappa}{c} \mid \ap{c}{c} \mid \into{c}
               \mid \out{c}\\
    \kappa & ::= & j \mid \tp \mid \fn{\kappa}{\kappa} \mid \rec{j}{\kappa}
  \end{array}
\]
This language features a simple term language but a much richer type
language, here called \emph{constructors}. Constructors allow for
computation which produces types. This is formally captured by two
judgments: $\hasTJ{\Omega}{\Delta}{c}{\kappa}$ and
$\hasTJ{\Omega}{\Delta}{c_1 \equiv c_2}{\kappa}$. The first judgment
specifies that a constructor $c$ has a kind $\kappa$ in a context of
kind variables $\Omega$ and constructor variables $\Delta$. The second
specifies when two constructors are equal. This latter judgment is
related to the actual expressions of the language by the following
rule.
\[
  \inferrule{
    \hasEJ{\Omega}{\Delta}{\Gamma}{e}{\tau_1}\\
    \hasTJ{\Omega}{\Delta}{\tau_1 \equiv \tau_2}{\tp}
  }{\hasEJ{\Omega}{\Delta}{\Gamma}{e}{\tau_2}}
\]
The constructor language typing judgments are standard for F$\omega$
(see \citet{Barendregt:13} for an explanation) except for the
following rules.
\begin{mathparpagebreakable}
  \inferrule{
    \hasKJ{\Omega, j}{\kappa}
  }{\hasKJ{\Omega}{\rec{j}{\kappa}}}\and
  \inferrule{
    \hasTJ{\Omega}{\Delta}{c}{\rec{j}{\kappa}}
  }{\hasTJ{\Omega}{\Delta}{\out{c}}{[\rec{j}{\kappa}/j]\kappa}}\and
  \inferrule{
    \hasTJ{\Omega}{\Delta}{c}{\rec{j}{\kappa}}
  }{\hasTJ{\Omega}{\Delta}{\out{c}}{[\rec{j}{\kappa}/j]\kappa}}
\end{mathparpagebreakable}
The calculus described so far is a version of F$\omega$ with recursive
kinds. This language is quite expressive because recursive kinds allow
for recursive constructor definitions of arbitrary sort through
encoding the Y-combinator at the constructor level. It is
exceptionally poorly behaved, however, because recursive kinds allow
for constructors which do not have a weak head normal form. This
forces choices in the language that seem to have no good answers. For
instance, what terms belong to a constructor of kind $\tp$ that has no
weak head normal form.

There is a technique for potentially approaching a logical relation
for even this language with general recursive types. It draws on two
key ideas, one from Section~\ref{sec:domains} and one from
Section~\ref{sec:smi}. We define a pair of logical relations
$\definitely{-}$ and $\possibly{-}$ as in Section~\ref{sec:smi}. This
logical relation is defined on the \emph{normal forms} of
constructors however. In order to handle the cases where a term has an
infinite normal form or simply diverges, we actually work of finite
fragments of the B\"ohm tree of a term~\citep{Barendregt:13}. Two terms
are related at a type with an infinite normal form if and only if they
are in all finite approximations of that type. In this case, there is
actually a meaningful difference between $\definitely{-}$ and
$\possibly{-}$: if we reach a node in the B\"ohm tree which indicates
that more information exists but we are not allowed access to it in
this approximation, say ?, $\definitely{?} = \{(\bot, \bot)\}$ while
$\possibly{?} = \term \times \term$.

This approach for logical relations works in that the logical relation
is well-defined and if both sides of the birelation $\eta$ are
well-defined, then $\definitely{\tau}_\eta =
\possibly{\tau}_\eta$. The issue is that this fact does not scale to
general case of $c : \kappa$. In particular, it is not the case that
for an $\eta = \op{\eta}$ that
$\definitely{\lam{\alpha}{\kappa}{c}}_\eta = \possibly{\lam{\alpha}{x}{\kappa}}_\eta$.
The issue is that these must be equal as functions on pairs of
relations $(R_1, R_2)$ and not merely on a single relation $R$. If
$R_1 = R_2$ they are equal but this information is not sufficient to
get the case for $\forall$ to go through in the fundamental
theorem. This failure will be instructive for the coming issue:
$\forall$ is the only case where the higher-typed constructors factor
in to the expression language. This interaction is crucial for
\citet{Pottier:11} so it cannot be ignored, but it will be at the
heart of the failure of the next attempt.

Instead of considering full recursive kinds, we restrict our attention
to a particular class of them: the class of guarded recursive
kinds. Guarded recursion in general originates with
\citet{Nakano:00}. The idea is to isolate which recursive
definitions are \emph{productive}. Productivity is informally the idea
that after a finite number of steps something must be produced by a
program. The program need not compute to a normal form but it should
produce some observable output. The classic example of productivity is
a stream: an infinite stream should never evaluate to a normal form
but we would like to be able to calculate the $n$th element of the
stream in finite time. Productive programming is an extremely natural
paradigm for many programs which are never expected to terminate but
still expected to produce work as they go.

The question is how to ensure that a definition is productive. One
line of work is to consider a syntactic condition on programs. This is
used by some proof assistants~\citep{coq} and is simple and
sound. The idea is to check that no recursive calls occur without
being underneath a constructor. In the stream example, for instance,
this means that an element of the stream has to be produced before the
function is entitled to recurse. The issue is that a syntactic check
is too brittle. It does not, for instance, work if a productive
definition has been factored into two different parts since one of
these parts may not be syntactically unguarded. It is also ill-suited
for handling higher-order functions where the productivity may rely on
the supplied function being productive.

A more robust solution is to follow the approach of
\citet{Nakano:00} and equip the type theory with a modality
indicating that something is only available to use ``at a later point
in time''. Then, for instance, in a language with streams the
constructor for a stream would have the type
$\fn{\tau}{\later \mathrm{Stream}_\tau}{\mathrm{Stream}_\tau}$
indicating that it is sufficient to provide the tail of a list so that
it is only available later. In order to make this modality useful,
$\later$ comes equipped with a number of operations. Two of the basic
ones ensure that $\later$ is an applicative functor:
\begin{align*}
  \delay{-} &: \fn{\tau}{\later \tau}\\
  - \zap - &: \later (\fn{\tau_1}{\tau_2}) \to \fn{\later \tau_1}{\later \tau_2}
\end{align*}
Finally, $\later$ can be used to construct fixed points.
\begin{align*}
  \fix{} : \fn{(\fn{\later \tau}{\tau})}{\tau}
\end{align*}
All of this is extremely similar to the operations available in
$\presheaves{\omega}$. The first two operations were facts about the
functor $\later$ and the last one is L\"ob induction. Nakano's work on
guarded recursion provides a syntactic account of this modality. In
the original presentation the operations written above were presented
as subtyping relations in the calculus rendering them all silent. In
more recent
presentations~\citep{Birkedal:steps:11,Bizjak:16,Birkedal:16,Bahr:17}
these operations are all made explicit again as they are mediated by
nontrivial isomorphisms. For our purposes, we will continue the
subtyping presentation for consistency with \citet{Pottier:11} which
describes the encoding motivating this construction.

Guarded recursive kinds are slightly different than guarded recursion
in general because the term language contains divergent and
nonproductive terms. The guarantees provided by guardedness apply only
at the level of constructors where it is the case that all
constructors are productive (have a weak head normal form). In this
sense, the language of guarded recursive kinds can be seen as a
particular simply-typed calculus with guarded recursion and a
distinguished base type $\tp$. The term-formers of this type are the
type-formers of our expression language: $\fn{-}{-}$ and
$\all{\alpha}{\kappa}{-}$. A further distinction between this calculus
and other type theories is that our type formers are all
contractive. This means that they take arguments available only later
and produce a result available now. In type theories with a universe,
for instance, the operations on the universe are all nonexpansive:
they preserve laters by taking arguments available later and producing
a result later (the inverse of the $- \zap -$ operation would do this
for instance). This extra strength provided by contractiveness is
exactly what allows for the construction of divergent terms in our
language. This would present a soundness issue for a type theory but
from our perspective it is a natural convenience of a programming
language unconcerned with serving as a logic.

To make this discussion formal, the (selected) rules of our language are the
following.
\begin{mathparpagebreakable}
  \declareJudgement{\hasKJ{\Omega}{\kappa}}\\
  \inferrule{
    j \in \Omega
  }{\hasKJ{\Omega}{j}} \and
  \inferrule{
    \hasKJ{\Omega}{\kappa} \\
    \hasKJ{\Omega}{\kappa'}
  }{\hasKJ{\Omega}{\fn{\kappa}{\kappa'}}}\and
  \inferrule{
    \hasKJ{\Omega}{\kappa}
  }{\hasKJ{\Omega}{\later \kappa}} \and
  \inferrule{ }
  {\hasKJ{\Omega}{\tp}} \and
  \inferrule{
    \hasKJ{\Omega, j : \kind}{\kappa}\\
    \guardJ{j}{\kappa}
  }{\hasKJ{\Omega}{\rec{j}{\kappa}}}
\end{mathparpagebreakable}
\begin{mathparpagebreakable}
  \declareJudgement{\guardJ{j}{\kappa}}\\
  \inferrule{
    \mbox{$j$ not free in $\kappa$}
  }{\guardJ{j}{\kappa}}\and
  \inferrule{
    \guardJ{j}{\kappa_1}\\
    \guardJ{j}{\kappa_2}
  }{\guardJ{j}{\fn{\kappa_1}{\kappa_2}}} \and
  \inferrule{ }{\guardJ{j}{\later \kappa}} \and
  \inferrule{
    \guardJ{j}{\kappa}
  }{\guardJ{j}{(\rec{j'}{\kappa})}}
\end{mathparpagebreakable}
\begin{mathparpagebreakable}
  \declareJudgement{\subKJ{}{\kappa}{\kappa'}}\\
  \inferrule{
    \subKJ{}{\kappa_1'}{\kappa_1} \\
    \subKJ{}{\kappa_2}{\kappa_2'}
  }{\subKJ{}{\fn{\kappa_1}{\kappa_2}}{\fn{\kappa_1'}{\kappa_2'}}} \and
  \inferrule{
    \subKJ{}{\kappa}{\kappa'}
  }{\subKJ{}{\later \kappa}{\later \kappa'}} \and
  \inferrule{
  }{\subKJ{}{\kappa}{\later \kappa}} \and
  \inferrule{
  }{\subKJ{}{\later (\fn{\kappa}{\kappa'})}{\fn{\later \kappa}{\later \kappa'}}} \and
  \inferrule{
  }{\subKJ{}{\fn{\later \kappa}{\later \kappa'}}{\later (\fn{\kappa}{\kappa'})}}
\end{mathparpagebreakable}
\begin{mathparpagebreakable}
  \declareJudgement{\hasTJ{}{\Delta}{c}{\kappa} \qquad \hasEJ{}{\Delta}{\Gamma}{e}{\tau}}\\
  \inferrule{
    \hasTJ{}{\Delta}{\tau_1}{\later \tp}\\
    \hasTJ{}{\Delta}{\tau_2}{\later \tp}
  }{\hasTJ{}{\Delta}{\fn{\tau_1}{\tau_2}}{\tp}}\and
  \inferrule{
    \hasTJ{}{\Delta, \alpha : \kappa}{\tau}{\later \tp}
  }{\hasTJ{}{\Delta}{\all{\alpha}{\kappa}{\tau}}{\tp}}
  \and
  \inferrule{
    \hasTJ{}{\Delta}{\tau}{\later^n\,\tp} \\
    \hasEJ{}{\Delta}{\Gamma, x : \tau}{e}{\tau'}
  }{\hasEJ{}{\Delta}{\Gamma}{\lam{x}{\tau}{e}}{\fn{\tau}{\tau'}}}
  \and
  \inferrule{
    \hasEJ{}{\Delta}{\Gamma}{e}{\forall \alpha : \kappa . \tau}\\
    \hasTJ{}{\Delta}{c}{\later^n \kappa}
  }{\hasEJ{}{\Delta}{\Gamma}{\Ap{e}{c}}{[c / \alpha] \tau}}
\end{mathparpagebreakable}
Turning now to the question of how to construct a logical relation for
this language there are two questions to be answered:
\begin{enumerate}
\item We want to interpret kinds as objects of a category, which
  category should be used?
\item Pairs of related terms should then be points of this object,
  what does it mean to be a point of $\den{\tp}$.
\end{enumerate}
This two-layer approach is typical of a logical relation for languages
with a rich kind structure. We need to interpret kinds as some sort of
semantic object so that $\den{\tp}$ is the collection of semantic
types needed for the logical relation. Traditionally the literature
around logical relations for higher-kinded languages has been
concerned with notions of parametricity at higher
kind~\citep{Hasegawa:94,Robinson:94,Dunphy:04,Vytiniotis:10,Atkey:12},
which extends the object-relational interpretation of
Reynolds~\citep{Reynolds:83} to include natural
notions of candidates at higher kind. These candidates include more
complex conditions that also necessitate the use semantic tools to
handle but it is not the goal of our work. In particular,
parametricity arguments will only be interesting for type variables of
kind $\tp$. At all other kinds the candidates are not relations
between semantic types but rather just semantic types. For instance
the interpretation of a constructor function of kind $\fn{\tp}{\tp}$
is a map between relations, not a relation between maps of semantic
types.

This is an undesirable simplification for a general purpose logical
relation but the only motivation for constructing this logical
relation is to study the semantics of guarded recursive kinds and
their impact on the definition. The technical complications introduced
by studying reflexive graphs would be wasted when we plan to move to a
setting with only $\tp$ and Kripke worlds. It is not clear what it
would mean to consider a model where Kripke worlds were subject to a
relational interpretation as well. To my knowledge this has not been
studied explicitly but it is perhaps latent in work like
\citet{Dreyer:10} where the Kripke world is more relational to allow
for relations between heaps that are not merely point-wise. This,
however, is a much loftier goal than merely constructing any logical
relation for state which is what we are after.

Now the answer to these questions is given by the earlier remark that
our language is the guarded simply-typed lambda calculus with guarded
recursive types and a strange base type. Models of this language are
readily available in $\presheaves{\omega}$~\citep{Birkedal:guarded:10,Birkedal:steps:11} and we
make use of these.

Specifically, we interpret kinds as follows as maps a functor from
environments to presheaves. These environments are mixed-variance.
They send a kind variable to two distinct presheaves, one of which is
used covariantly and one contravariantly. Let us signify these
environments by $\theta$ (to distinguish them with from the
environment used by the interpretation of constructors) and write
$\op{\theta}$ for the operation exchanging all the components of
$\theta$. Finally, by convention let $\theta(\alpha)$ project out the
presheaf being used covariantly.
\begin{align*}
  \den{j}_\theta &= \theta(j)\\
  \den{\later \kappa}_\theta &= \later \den{\kappa}_\theta\\
  \den{\fn{\kappa_1}{\kappa_2}}_\theta &= \den{\kappa_2}_\theta^{\den{\kappa_1}_{\op{\theta}}}\\
  \den{\rec{j}{\kappa}}_\theta &= \fix((P_1, P_2) \mapsto \den{\kappa}_{\theta[j \mapsto (P_1, P_2)]})\\
  \den{\tp}_\theta &= \semtypes\\
\end{align*}
It is a straightforward proof to show that this defines a functor. The
only exceptional fact is that it is necessary to show that taking a
fixed point some arguments of a mixed-variance functor results in a
functor in the remaining arguments. This is proven in Section 7 of
\citet{Birkedal:steps:11} for instance.

The interesting part of this definition is the interpretation of
$\tp$, $\den{\semtypes}$. The other clauses are simply induced by the
logical meaning of the connectives. Interpreting $\to$ as the
exponential, for instance, induces a canonical interpretation of
$\lam{\alpha}{\kappa}{c}$ and $ap{c_1}{c_2}$.

The space of semantic types is where we make crucial use of syntactic
minimal invariance. First we need to define a class of relations
appropriate for semantic types.
\begin{defn}\label{def:guarded:urel}
  A uniform relation is a subset
  $R \subseteq \term_{\tau_1} \times \term_{\tau_2}$ for some $\tau_1$
  and $\tau_2$ so that the following hold.
  \begin{itemize}
  \item $R$ is closed on both sides under contextual equivalence.
  \item $R$ is admissible and in particular contains limits of all
    $\pi$-chains.
  \item If $(e_1, e_2) \in R$ then for all $n$,
    $(\ap{\pi_{\tau_1}^n}{e_1}, \ap{\pi_{\tau_2}^n}{e_2}) \in R$.
  \end{itemize}
  The set of uniform relations is denoted $\urel$.
\end{defn}
Uniform relations are a natural setting for our logical relation since
the assumptions imposed on them are quite natural for anything to be
called a semantic type. Items 1 and 3 require that we respect
observation in and item 2 is a technically necessary and quite
reasonable assumption. The definition of projection in this language
is identical to what is was for the System F-like language considered
previously with the slight proviso that instead of
$\pi_{\alpha}^{n + 1} = \lam{x}{\alpha}{x}$ we must generalize it to
$\pi_{K[\alpha]}^{n + 1} = \lam{x}{\alpha}{x}$ so that any
\emph{neutral type} is impossible to truncate. This definition is
forced, there are no other reasonable moves to make at higher-type
since without knowing anything about $\alpha$ no information can be
deduced about $K[\alpha]$

The presheaf structure on uniform relations is instance of a more
general and well-known embedding of complete 1-bounded ultrametric
spaces into $\presheaves{\omega}$ described for instance in \citet{Birkedal:steps:11}:
\[
  \semtypes(n) = \urel / {\equiv_n}
\]
The equivalence relation at $n$ is defined to equate uniform relations
which agree on $n$ truncations. Formally, this is expressed as
follows.
\begin{align*}
  R \equiv_n &S =\\
  &(\forall (e_1, e_2) \in R.\ (\ap{\pi^i_{\tau_1}}{e_1}, \ap{\pi^i_{\tau_2}}{e_2}) \in S) \land {}\\
  &(\forall (e_1, e_2) \in S.\ (\ap{\pi^i_{\tau_1}}{e_1}, \ap{\pi^i_{\tau_2}}{e_2}) \in R)
\end{align*}
Restriction maps simply send equivalence classes of uniform relations
to even coarser equivalence classes of uniform relations.

This definition implies that a global point: $h : 1 \to \semtypes$
precisely in correspondence to a particular uniform relation. It picks
out a family of equivalences classes so that each is increasingly fine
$h(0)(\star) \supseteq h(1)(\star) \supseteq ... \supseteq h(n)(\star)$
Syntactic minimal invariance and admissiblity of uniform relation
tells us that if two uniform relations are equated by $\equiv_n$ for
all $n$ then they are equal on the nose. This means that there can be
at most one uniform relation in all the equivalence classes selected
by $h$ and $h$ is uniquely induced by the uniform relation. This
result is in fact a corollary of the fact that the aforementioned
embedding preserves finite limits.

Now the logical relation (the constructor interpretation) is really a
standard interpretation of the simply-typed lambda calculus with a
base type who's occupants are relations. It is defined as follows.
\begin{align*}
  \den{\alpha}_\eta(n) &= \eta(\alpha)(n)\\
  \den{\lam{\alpha}{\kappa}{c}}_\eta(n) &= (\star, S) \mapsto \den{c}_{\eta[\alpha \mapsto S]}(n)\\
  \den{\ap{c_1}{c_2}}_\eta(n) &= \den{c_1}_\eta(n)(\star)(\den{c_2}_\eta(n))\\
  \den{\out{c}}_\eta &= \mathsf{out}_n(\den{c}_\eta(n))\\
  \den{\into{c}}_\eta &= \mathsf{in}_n(\den{c}_\eta(n))\\
  \den{\fn{\tau_1}{\tau_2}}_\eta(n) &=\\
  \{(e_1, e_2) \mid&\ e_1 \Downarrow \iff {e_2 \Downarrow} \AND \\
  & \forall (a_1, a_2) \in \den{\tau_1}_\eta(n). \ (\ap{e_1}{a_1}, \ap{e_2}{a_2}) \in \den{\tau_2}_\eta\}\\
  \den{\all{\alpha}{\kappa}{\tau}}_\eta &=\\
  \{(e_1, e_2) \mid&\ e_1 \Downarrow \iff {e_2 \Downarrow} \AND\\
  &\forall c_1, c_2 : \kappa,\ S \in \den{\kappa}.
    \ (\Ap{e_1}{c_1}, \Ap{e_2}{c_2}) \in \den{\tau}_{\eta[\alpha \mapsto S]}\}
\end{align*}
In this definition we are using $\mathsf{out}$ and $\mathsf{in}$ as
the isomorphisms mediating a recursive type and its unfolding. There
is a technical issue with this presentation: it properly should be
done on \emph{derivations} of terms and a coherence theorem must be
proven. This is because the while it is the case that if
$\kappa_1 \le kappa_2$ then there is a canonical
$i : \den{\kappa_1} \mono \den{kappa_2}$, it is not the case that $i$
is an identity. This means that when the subtyping rule is applied in
a derivation the resulting semantic object must be adjusted
explicitly. This issue is obliviated when working in ultrametric
spaces because the subtyping coercions are simply identities
there. This technical detail is not relevant to the issue with this
approach and thus we have suppressed it.

The extra applications of $n$ and $\star$ are results of the fact that
the interpretation is a natural transformation from $\den{\Delta}$,
here written $\eta \in \den{\Delta}$ to $\den{\kappa}$. This is
defined by an indexed family of functions which is what has been done
here. Additionally, since constructor functions are exponentials in
the category at stage $n$ they are natural transformations
$\yoneda(n) \times \den{\kappa_1}_\eta \to \den{\kappa_2}_\eta$ so we
must apply them to an element of $\yoneda(n)(n) = \{\star\}$ in
addition to the actual argument.

The issue arises from the fact that this must be made well-typed and
natural. These two requirements are inexorably tied together by the
interpretation of $\lambda$ which is only a valid natural
transformation if $\den{c}$ is natural in $\eta$. It is unfortunately
the case that these properties fail to hold for the definition of
$\den{\all{\alpha}{\kappa}{\tau}}$. In this case, being well-typed
implies that for at $n$, for every pair $\eta$, $\eta'$ which are
equal at stage $n$ it must be that
$\den{\all{\alpha}{\kappa}{\tau}}_\eta \equiv_n \den{\all{\alpha}{\kappa}{\tau}}_{\eta'}$.
Importantly, equality at stage $n$ is \emph{not} necessarily just
point-wise equality because two semantic types are equal at stage $n$
if they are related by $\equiv_n$. This essentially is a statement
that $\den{\all{\alpha}{\kappa}{\tau}}$ is functional with respect to
$\equiv_n$ and this is simply false.

As a counter example, consider the following instance of $\tau$.
\[
  \tau = \fn{\ap{\alpha}{\beta}}{\ap{\alpha}{\gamma}}
\]
For convenience, let us denote
$\all{\alpha}{\fn{\tp}{\tp}}{\fn{\ap{\alpha}{\beta}}\ap{\alpha}{\gamma}}$
as $\tau'$ In this case we must show that show that
$\den{\tau'}_\eta \equiv_n \den{\tau'}_{\eta'}$ provided that
$\eta(\beta) \equiv_n \eta'(\beta)$ and
$\eta(\gamma) \equiv_n \eta'(\gamma)$. Let us suppose that
$\eta(\beta)$ is a uniform relation $R \subseteq \tau_a \times \tau_a$
and $\eta'(\beta)$ is $S \subseteq \tau_a \times \tau_a$ and both
$\eta(\gamma)$ and $\eta'(\gamma)$ are $R$. All that is needed to
cause a problem is to assume that $R \neq S$. If such an $R$ and $S$
cannot be constructed then $\urel$ is trivial and our logical relation
is degenerate.

We are required to show that $\pi^n$ transports from
$\den{\tau'}_\eta(n)$ to $\den{\tau'}_{\eta'}(n)$ and back again. Proving
the first part of this means showing that if
$(e_1, e_2) \in \den{\tau'}_\eta(n)$ then the following holds.
\[
  (\ap{\pi_{[\tau_a/\beta]\tau'}}{e_1}, \ap{\pi_{[\tau_a/\beta]\tau'}}{e_1})
  \in \den{\tau'}_{\eta'}(n)
\]
However, if $n > 2$ and $e_1$ and $e_2$ do not diverge, then this is
equivalent to showing the following.
\[
  (\Lam{\alpha}{\tp}{\Ap{e_1}{\alpha}}, \Lam{\alpha}{\tp}{\Ap{e_1}{\alpha}})
  \in \den{\tau'}_{\eta'}(n)
\]
This can be simplified further since $\den{\tau'}_{\eta'}(n)$ is a uniform
relation to just the following.
\[
  (e_1, e_2) \in \den{\tau'}_{\eta'}(n)
\]
Which is to say, we are required to show that if $\eta$ and $\eta'$
are only related by $\equiv_n$ then we must show that
$\den{\tau'}_{\eta}$ and $\den{\tau'}_{\eta'}$. However, it is not
hard to show that the following two facts are true.
\[
  (\Lam{\alpha}{\fn{\tp}{\tp}}{\lam{x}{\ap{\alpha}{\tau_a}}{x}},
   \Lam{\alpha}{\fn{\tp}{\tp}}{\lam{x}{\ap{\alpha}{\tau_a}}{x}})
   \in \den{\tau'}_\eta
\]
\[
  (\Lam{\alpha}{\fn{\tp}{\tp}}{\lam{x}{\ap{\alpha}{\tau_a}}{x}},
   \Lam{\alpha}{\fn{\tp}{\tp}}{\lam{x}{\ap{\alpha}{\tau_a}}{x}})
   \not\in \den{\tau'}_{\eta'}
\]
This means that $\den{-}$ is not type-correct and not a well-formed
definition. At the root of the issue is the fact that syntactic
minimal invariance does not seem to scale to higher kinds without some
fundamentally new idea. Without something akin to a type-casing
operation it does not seem possible to define a correct version of
$\pi$ at neutral types. It really seems that it must proceed by
induction on the variable the neutral constructor is stuck on. This
issue does not arise in domain theory because in this setting
projections are untyped and case on the structure of the element of
the domain not the ``type'' that this element may later be shown to
inhabit.

Any cure for this issue seems worse than the illness. Adding type-case
for instance allows us to define a correct projection for instance but
it destroys the parametricity of the system. One could consider a
system like LX~\citep{Crary:99} and define projection with respect to
the reification of a type. This solves the issue for all the
types that we have reifications of but there is a diagonalization
issue here: codes are needed for every single constructor but the
codes themselves are constructors.

Finally, as mentioned previously $\urel$ is inadequate for
representation independence results. Since we have required that
uniform relations are closed under truncations any uniform relation
between different types will relate divergent terms on one side to
nondivergent terms on the other. This puts a dent in using this
technique to construct a logical relation for state. Essentially, we
have removed the explicitly need for $\trunc_i$ but all proofs must
act as though $\trunc_i$ is still in the language. All together, this
means that even fixing this particular logical relation is likely
inadequate for our purposes.

%%% Local Variables:
%%% mode: latex
%%% TeX-master: "../main"
%%% End:

\section{Conclusions}

This thesis has illustrated the difficulty of constructing a logical
relation for a language with higher-order state. The current set of
techniques which work are complex and despite this they only provide
sound but incomplete reasoning principles. Even with this logical
relations are among the best known techniques for handling
effects. They strike a combination between using simpler mathematics
and providing useful reasoning principles.

The difficulty and complexity of the attempts to remove step-indexing
from the logical relation raises a question: is it worth it? The
motivations for step-index-free logical relations were basically the
following:
\begin{itemize}
\item Logical relations for step-indexing were complicated and
  painful to use and construct correctly.
\item Step-indexing is ad-hoc and difficult to justify.
\item Step-indexing is ill-suited for capturing properties beyond
  safety.
\end{itemize}
The first two issues are addressed in the same way: a generalization
in the way we view logical relations. If logical relations are just
structure-preserving relations, there's no reason to limit ourselves
to traditional set-theoretic relations. Already with Kripke logical
relations there is the same indexing structure. Rather than viewing
relations as holding or not holding it is quite natural to replace
them with a more internsional, local notion of truth as is done in
both step-indexing and Kripke logical relations. Mathematically, this
is just a switch from relations in sets to relations in an appropriate
presheaf category. If we make use of more sophisticated tools for
working with these relations, either by working internally to a type
theory, a higher-order logic, or a topos much of the day-to-day issues
of step-indexing can be resolved.

The last issue is a real problem. The nature of step-indexing is
inherently limiting because it is only is suited for properties that
are sound with respect to finite prefixes. Consider for instance, the
set of programs producing a stream of natural numbers. The program
outputting $1$, $2$, $3$, etc. At any finite step this program
produces a maximal number but it is not the case that it ever actually
outputs a maximum globally. This is a classic issue in mathematics:
global truth is not truth in every localization and step-indexing only
allows us to discuss local truth.

In order to fix this, one can either remove the indexing structure and
never work with local truth or add enough locales so that local truth
is strong enough. For instance, if we work with $\omega^2$ in the
example above the problematic program is ruled out: it does not
satisfy the program at $(\omega, 0)$. A fuller theory of logical
relation in this setting is given by \citet{Svendsen:16}.

Further investigation is needed in both directions. There are many
unstudied generalizations of relations beyond merely a set-theoretic
subset of a product. On the other hand, there is an undeniable appeal to
just being able to work with sets and to find some new clever
technique to avoid the indexing all together. At the moment though,
there seems to be far more unexplored and accessible space in the
first direction than the second; constructing a step-index-free
logical relation appears to require some new approach beyond
applications of classic tools as was done in this thesis.

%%% Local Variables:
%%% mode: latex
%%% TeX-master: "../main"
%%% End:

\section*{Acknowledgments}

I would like to thank the huge number of people who have been
instrumental in writing this thesis and the research of the last four
years that went into it. I am greatly indebted to the POP group at CMU
for a unique environment to study. A special thanks to Carlo Angiuli,
Evan Cavallo, Adrien Guatto, Anders M{\"o}rtberg, Jonathan Sterling,
and Joseph Tassarotti for countless hours of discussions on
programming languages. In particular, I must thank Jonathan Sterling and
Carlo Angiuli for valuable feedback on a draft of this thesis. Thanks
to Lars Birkedal and Robert Harper for feedback and advice on this
work. Finally, I must thank Karl Crary for advice, encouragement, and
much wisdom into how to do research in type theory.

%%% Local Variables:
%%% mode: latex
%%% TeX-master: "../main"
%%% End:


\bibliographystyle{plainnat}
\bibliography{citations}
\end{document}
