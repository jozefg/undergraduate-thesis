\documentclass{beamer}
\usepackage{mathpartir, stmaryrd}
\usepackage{wasysym}
\usepackage[normalem]{ulem}
\usepackage{listings}
\usepackage{multicol}
\usepackage{microtype}
\usepackage{color}

\usefonttheme[onlymath]{serif}
\definecolor{JadeGreen}{RGB}{0,168,107}
\definecolor{MunsellPurple}{RGB}{159,0,197}

\definecolor{CobaltBlue}{RGB}{0,71,171}
\definecolor{FireBrick}{RGB}{228,34,23}
\definecolor{Alabaster}{RGB}{250,250,250}

\setbeamertemplate{navigation symbols}{}
\setbeamertemplate{headline}{}
\setbeamercolor{example text}{fg=CobaltBlue}
\setbeamertemplate{itemize items}{$\bullet$}
\setbeamercolor{alert text}{fg=FireBrick}

\lstset{language=ML,
        columns=fullflexible,
        keepspaces=true,
        deletekeywords={let,in},
        basicstyle=\ttfamily,
        tabsize=4,
        escapeinside={"*}{*"},
        moredelim=*[is][\color{FireBrick}]{`}{'},
        literate={->}{{$\shortrightarrow$}}1
                 {->}{{$\gets$}}1
                 {\\}{{$\lambda$}}1
                 {decl}{{$\decl$}}1
                 {set}{{$\set$}}1
                 {get}{{$\get$}}1
                 {cmd}{{$\cmd$}}1
                 {ret}{{$\ret$}}1
                 {times}{{$\times$}}1
                 {alpha}{{$\alpha$}}1
                 {beta}{{$\beta$}}1
                 {gamma}{{$\gamma$}}1
                 {vdots}{{\vdots}}1}

\title{The Next 700 Failed \\Step-Index-Free Logical Relations}
\author{Daniel Gratzer}
\date{\today}

\begin{document}
\begin{frame}
  \titlepage
\end{frame}

\begin{frame}
  \begin{center}
    \bf When are two programs \alert<5>{equal}?
  \end{center}
  \begin{enumerate}
  \item<2-> Checking compiler optimizations
  \item<3-> Formalize data abstraction
  \item<4-> Compositional verification proofs
  \end{enumerate}
\end{frame}

\begin{frame}
  \begin{center}
    \bf What should equality mean?
  \end{center}
  \pause
  \begin{itemize}
  \item Should equality be decidable?
    \pause
  \item Should equality consider computational time?
    \pause
  \item Should free variables be treated as \emph{indeterminates}?
  \end{itemize}
\end{frame}

\begin{frame}
  \begin{center}
    Two programs are equal if you can replace one with another in a
    bigger program without causing the bigger program to diverge.
  \end{center}
\end{frame}

\begin{frame}
  \centering
  To prove that two programs, $e_1$ and $e_2$ are equal...
  \begin{enumerate}
  \item Pick a program with a hole in it, $C$
  \item Test whether $C[e_1]$ terminates
  \item Test whether $C[e_2]$ terminates
  \item If the results disagree, they're not equal
    \pause
  \item Rinse and repeat for every possible $C$
  \end{enumerate}
\end{frame}

\begin{frame}
  Contextual equivalence has some advantages:
  \begin{itemize}
  \item Easily scales to different languages
  \item \emph{Extensional}
  \item Allows us to replace equals by equals
  \end{itemize}
\end{frame}

\begin{frame}
  \begin{center}
    \it Rinse and repeat for every possible $C$
  \end{center}
  \pause
  There are a lot of $C$s.
  \pause
  It's not obvious how to write a proof which handles all of them.
\end{frame}

\begin{frame}
  \centering
  Key idea: use the structure of the types to classify what contexts
  are worth considering.
\end{frame}

\end{document}
